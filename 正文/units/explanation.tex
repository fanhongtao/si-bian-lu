\chapter{说明}

思想,是行动的指南。唯有思想上想清楚、想明白了,才可能做正确的事以及正确的做事。

父亲告诉我:有理就有法,有法就有方。虽然他是从中医的角度来讲的,但我认为这一思路在其他方面也是行得通的。
所以有必要将自己的思想提炼出来,作为行动的指南。于是萌生了将思想记录成文字的念头。

本书的内容安排如下:

\begin{enumerate}[(1), nosep, left=\parindent]
    \item 按照时间的先后顺序进行排序,每一小节对应一个事件(问题、现象)的思考。

    \item 考虑到不同事件之间关联,增加了标记(Tag),在书尾按不同的标记(Tag)对小节进行分类。

    \item 每一个小节写好之后,如果不是错别字或严重的语法行文方面的错误,原则上不再进行修改。

    \item 由于对事物的认识,会随着时间的变化而改变。所以,如果对某一事件的看法发生了变化,则在新的小节中重新阐述,原小节的内容保留不动,这种安排,是为了更好的了解自己思想的变迁过程。
\end{enumerate}

曾子曰:“吾日三省吾身:为人谋而不忠乎?与朋友交而不信乎?传不习乎?” \footnote{《论语·学而》}
本书的目的,是通过对自我思想的剖析,指导自己的为人处事,力争让自己活成一个明白人。所以本书的目标读者,就是我自己。
对有缘读到本书的其他读者,如果书中的内容对你有所启发,那将是我的荣幸。
