\begin{taged}{家事}
  \section{2024-04-04-我的父亲}
\end{taged}

我的父亲,范德才,生于1946年2月25日,死于2023年12月27日,享年77岁。

父亲是个早产儿,按中医的说法,先天元气不足,自幼体弱多病,长身体时又赶上“三年自然灾害”
\footnote{“三年自然灾害”,或者称“三年困难时期”,指中国从 1959年至1961年期间由于大跃进运动以及牺牲农业发展工业的政策所导致的全国性的粮食和副食品短缺危机。},
所以个子矮小、身体瘦弱。他身高不到1.55米,根据他自已的说法:年青时,才60多斤,多次在走路途中晕倒,醒过来后继续走路。

父亲从小就爱看书学习。小学毕业后,家里没钱让他读初中。
为了争取到读书的机会,他得在假期里想办法自己赚取学费,赚到学费才能读书。
在多次尝试之后,他终于能够在一个篾匠那里做帮工,就这样靠着自己赚的学费读完了初中。
初中毕业后,由于篾匠已经搬走,找不到合适的赚钱方式,因为没钱才没去读高中。
我小时候学习成绩不好,父亲总拿他当年的读书时的情况来教育我。
他说,他当年读书的时候,成绩很好,有时候数学老师不在,就由他给同学们上课。
要不是家里穷,至少能读到高中毕业。读个大学,想来也不是什么难事。

从学校出来后,就面临找工作的问题。
爷爷以前是在“红联社”做装卸工,所以父亲原本是有机会去“子承父业”做装卸工。
但由于他太瘦弱,抗不动一两百斤的包裹,没办法做这个工作,只能作罢。

后经人介绍,父亲去了绵阳第四建筑工程公司。
从搬运沙石、搅拌水泥等体力活开始做起,再到砌墙、盖房等技术活,成了一名泥瓦匠。
由于公司并不是每天都有活,没有活就没有收入,故他也经常去接一些修补灶台、烟囱及修整屋顶等活以补贴家用。
到了80年代,国家的政策允许大家考取证书,他通过自学,考取了施工技术员的证书,开始以技术员的身份给一些工地的项目设计图纸和指导现场施工。
有一次,父亲接了一个县医院大楼设计加施工的活,在楼差不多盖好时,县里来了一个刚毕业的大学生,说这楼设计得有问题,有坍塌的风险。
这个说法把甲方吓坏了,工程被立刻叫停,父亲也被软禁起来,不允许离开工地。
于是,他和那个大学生打笔墨官司,他们将各自的理由写下来,邮寄到北京的一个设计所,由设计所进行仲裁。
等到北京的回复说,父亲的设计没有问题,父亲才得以恢复自由,工程也才继续进行,直到最后建造完毕,通过验收。

建筑公司的效益不好,需要员工自行去找项目,公司在仅负责提供发票的情况下,还要收取多半的利润。
到了90年代,在找不到项目的情况下,父亲只能自谋出路,走上了做生意这条路。
当时,中国正处于从计划经济到市场经济的转型期,不仅各种物质相对匮乏,而且供需信息还不通畅,
有货的企业找不到买主,而有需要企业又采购不到自己所需的货物。
因为存在这种信息差,于是就有了一群靠撮合买卖双方谋生的人,父亲就是其中的一员。

那段时间,时常都有不同的人到家里来,和父亲“谈生意”。
现在我还有一点印象的人有:脸上长有麻子的“魏叔叔”、
江油\footnote{江油市,四川省辖县级市,由绵阳市代管。}的“王叔叔”
和九院\footnote{即,\href{https://baike.baidu.com/item/中国工程物理研究院}{中国工程物理研究院}。}的一位采购员。
当各种叔叔来家里时,父亲常常会叫我去给客人泡茶。
当时我还在读小学,还不太理解这里面的难处,只是觉得这些人有事没事就往我家跑,还要让我去给泡茶,太烦了。
因为父亲是之前是搞建筑出身,对钢材比较熟悉,所以他做的生意也是以各种钢材为主,
经常听到他和来访的叔叔们谈论各种型号的“螺纹钢”、“不锈钢”、“马口铁”。

在所有做成的生意中,父亲对他替九院购买到材料比较自豪,曾多次给我讲述。
当时,九院要做核试验,需要某种特殊的钢材,他们以核工业部的名义给全国的物资局发了调令,
可各物资局都说没有,怎么都买不到,已经影响到试验进度。为此,九院的采购员们被多次责骂。
在九院的采购员们四处寻找的过程中,其中的一位采购员找到父亲。
父亲告诉他说,我知道那里有材料,但我自己没钱买,你要是真想要,就先把钱打给我。
先前他害怕打钱后拿不到货,是不同意这种做法的,后来估计是被逼得没办法了,说服了他的领导,将钱打到父亲的个人银行账户。
父亲拿到钱后,真的替他们买到到所需的材料。
可以说,父亲通过他的生意,为中国的核工业做出了自己的贡献。

父亲给我聊天时说到他做生意的事情。\footnote{参考“录音”目录下的:范德才@20200825085631.amr}
他说,他没钱没资源,能做成很多单生意,也没有什么别的诀窍,
不过是赚了钱之后,将大部分利润分给为他提供信息的那些叔叔们。
就是凭着让别人拿大头,他拿小头,所以承蒙各位朋友赏脸,有消息都主动通知他,他才得以做成这些生意。
到90年代中后期,随着我国市场经济的一步步建立,信息差变得越来越少,他才停止了做生意。

父亲自幼身体不好,所以他从很早就开始自学中医。不做生意之后,更是将大部分时间都用在钻研中医上。
遇到身体不适,就自己写个药方,让母亲去药店抓药。
得宜于他懂医,会自己保养,所以在他的弟兄四人中\footnote{父亲有一个大哥、一个姐姐、还有一个弟弟。},
他虽然是身体最差的一个,但却是活得最长的,最后一个走。

\contentsep

\textbf{轶事1}:小鬼报恩

在我上幼儿园期间,有一次母亲的单位组织去旅游。
父亲、母亲和我一起,先去乐山看乐山大佛,然后又去了峨眉山。
在峨眉山上逛某一个寺庙时,父亲看见一个脚踩小鬼的菩萨像。
父亲指着那个小鬼说:他(指那个小鬼)犯了多大的罪?在这里被脚踩了上千年,有罪也应该消得差不多了,应该把他放出来。
旁边的人连忙制止父亲这样说。大家原本以为这就算结束了。

过了两三年,婆婆\footnote{父亲的母亲,四川话里叫“婆婆”。}走路时,因为路滑摔跤,摔断了腿骨。
以前通讯不发达,不像现在,不管多远,打个电话就可以找到人。
当时父亲在外地施工,家里人就托人带口信给他,通知他这件事情。
就在婆婆摔倒开始,至口信带到为止,这期间的一个晚上,他在去上厕所的途中,遇到了一件怪事。
在去厕所的路上,要经过一个小桥。当他来到小桥的一头,准备过桥时,忽然发现桥对面亮光一闪,
紧接着看见一个身材巨大、青面獠牙的鬼物急速地向他飘了过来,刹那之间,鬼物就已经飘至跟前。
父亲来不及逃跑,心想今天怕是要交待在这里了。
就在感觉鬼物那锋利的指尖已经快要戳到自己时,身边又是一道亮光,仿佛之间,好像有一个迷你的菩萨出现,
将那鬼物收了过去,最后菩萨消失不见,当晚再无其它事情发生。
没过几天,父亲就收到了口信。

在我小时候,父亲在与人聊天时,曾多次讲述过此事。
说当时来不及细想,等平静下来回想,发现那天晚上见到的鬼物,和峨眉山上被踩的那个小鬼有点像。
大家分析,这多半是峨眉山的小鬼来报恩,想要通知他:家里出事了。
只是因为人鬼殊途,无法直接语言交流,所以才用这种形式。

\textbf{轶事2}: 仗义救人

在我上大学期间,有一次父亲去成都进货。
当他走在成都的一座桥上时,发现桥上有一男子神态可疑,还有点面善?
可他不认识这名男子。这就奇怪了。于是他停了下来,仔细观察该男子。
当他发现该男子有跳河的行动时,及时上前制止了他。

在劝说该男子放弃轻生念头的过程中,得知了他的情况。
原来他本是附近郊县的一个农民,无事可做,打听到有人贩卖活体动物赚了钱,就找人借了一万多块钱,
从国外买了些乌龟、甲鱼之类的动物,到成都来贩卖。
市场里的商贩想要低价收购,给的价格太低,连保本都做不到,被他拒绝后,商贩将他举报给了管理部门。
管理部门以他没有检验检疫为由,没收了他打算贩卖的动物,并且予以销毁。
这事还上了前一天晚上的新闻。父亲感觉他面善,就是因为前一天晚上在新闻中见过他!

该男子说,想到自己借了钱,现在却一无所有,钱没了,动物也没了,干脆死了算了。
父亲心地善良,不忍见此男子还年青就被钱给逼死,于是对那男子说:
你我算是有缘,昨天在电视上看到你,今天又恰好碰到了。
我身上带着两万块钱,今天原本是来成都进货的,现在货也不进了,我把钱都借给你。
你先把之前借的债还了,剩下钱去做点小买卖。我这个钱,等你以后赚了钱再还。

父亲在那男子的千恩万谢中,将随身带的钱借给了他,留下了自己的联系方式后就回家去了。

过了些年,该男子时来运转,跑去给一个有钱的老板做操盘手,赚了钱后,还清了父亲借给他的钱。

父亲在讲述这段经历时说,我当时就只想着要救他一命,没有想过要他还钱。他能还就还,还不起就算了。
以前读《水浒传》这样的小说时,看到“小旋风柴进”、“及时雨宋江”这样仗义疏财的好汉,心生敬仰。
不曾想,父亲也做了一回这样的好汉。要知道,1996~2000年我上大学期间,家里给我的生活费是200元/月。
两万元,相当于我100个月的生活费。这可不是一笔小数目。


\textbf{轶事3}:改方自救\footnote{参考“录音”目录下的:范德才@20220411180211.amr}

父亲懂点医术,经常给自己开方拿药。久而久之,周围的邻居都知道这点。

一天,父亲在小区里散步时,遇到一位阿姨,问我父亲是不是懂中医。父亲说,自己是懂一点。
阿姨说自己患有冠心病,找人给开个了方,但吃了药后没什么效果,想让父亲给改改药方。
父亲说,我没有研究过这个病,暂时还不敢给你改方,你可以先把药方给我看看。

拿到药方后,父亲回家和母亲说起此事。母亲说,那位阿姨的女儿就是这方面的专家。
这下,父亲就更不敢给她改方了。三天后,阿姨就去世了。

虽然那们阿姨去世了,但是父亲并没有停止这方面的研究。
经过几个月的研究,他写了自己第一版的药方。

几年后(2017年12月),孃孃\footnote{父亲的姐姐。}因冠心病住院,住院期间,想让父亲给她开点药。
于是父亲就在上次研究的那一版药方基础上,抓了两副药,煎好后送到医院。
医生看见,说你们要是让病人吃来历不明的药,就不要做院了,出了事医院不负责。
结果带去医院的药孃孃没有喝,被母亲从医院拿回家。而孃孃一两天后就去世了。

带回家的药,父亲舍不得丢掉,将其放置在冰箱里。戏称谁要得了冠心病,就将这药送给他。
结果,他自己很快也就出现了和孃孃相同的心绞痛症状。
此时,冰箱里的药还没有变质,他就拿来自己喝。
喝完后,症状很快缓解,父亲感觉自己的药方起作用了。可还没高兴几天,症状又出现了。

在第一版药方的基础上,父亲结合自己的状况改进了药方。
按新方抓了三副药,吃完后,症状就消失了,至今(录音时间为 2022年4月11日)没有再复发。


\contentsep

俗话说:“人无完人”。父亲当然也有他自己的缺点,但谁都可以评论他,唯独我不能!

因:\zhongdian{子不言父过}。

今天是清明节,又恰逢父亲去世百日,仅以此文纪念我的父亲。
