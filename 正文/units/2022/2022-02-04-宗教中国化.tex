\begin{taged}{宗教}
\section{2022-02-04 宗教中国化}
\end{taged}

《互联网宗教信息服务管理办法》\footnote{原文已收录在“素材”目录中。} 中提到“坚持我国宗教中国化方向,积极引导宗教与社会主义社会相适应,
维护宗教和顺、社会和谐、民族和睦”。那么,怎么样才叫做“宗教中国化”?

我认为,只有做到以下几点,才能算得上是宗教中国化:

首先,传教必须使用中文。在中国却使用外文,那是\textbf{\color{red}殖民},不是传教。
从起源上讲,佛教是外来宗教,但从经文都经过翻译,所以今天我们读佛经,使用的是自己的语言文字。

其次,经文教义要符合中国的传统价值观。“中国化”,应该是以中国的传统为主,对外来宗教进行改造。
如果所传的内容与中国的传统价值观相背离,那又怎么能称得上是“中国化”?
某些教派,禁止信众孝敬父母、祭拜祖先……,这些违背人伦歪理邪说不从教义中删除,怎么能说是“中国化”?

再次,“宗教与社会主义社会相适应”,就要求宗教团体拥护中国共产党的领导,
遵从中国的法律法规,绝对不能搞“教法大于国法”的那种分裂国家的做法。

最后,要做到“维护宗教和顺、社会和谐、民族和睦”,经文教义应该是重点强调“和”,做到“和而不同”。
经文教义中那些征服、迫害、杀戮“异教徒”的内容,都应该删除。
试想,一个教派的信徒,每天读诵的经文都是告诉他应该去征服、迫害、杀戮“异教徒”,经年累月之下,
这样的信徒如何能够与普通大众和睦相处?

