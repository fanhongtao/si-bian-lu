\begin{taged}{教育}
\section{2022-06-08 毒教材与禁学部}
\end{taged}

这一段时间,关于课本插画的讨论已经很多了。但教材的问题,又何止是插画?

相比于插画,更大的问题在于教材的内容。大家普遍认为现行的教材不适合学习。按照现有教材教学,老师难教、学生难学。
有个说法是:
\begin{itemize}[nosep, left=\parindent]
    \item 以前的教材,生怕大家学不懂;
    \item 现在的教材,生怕大家学懂了。
\end{itemize}
造成的直接后果,就是学生通过在学校学习,很难学懂弄通,被迫加入补习大军。


\href{http://product.m.dangdang.com/detail29334931-0-1.html}{《基础高等数学》}一书的介绍写到:
\begin{yinyong}
“由于中学数学已实行教学改革,教学内容发生较大变化,严重影响了大学高等数学的教学。为使中学数学与高等数学的教学内容有效衔接,
本书将高等数学需要而中学删去的数学内容统统找回来,主要内容有三角函数的积化和差与和差化积、反三角函数、参数方程与极坐标,
还有中学文科数学删除的排列与组合、二项式定理、数学归纳法、复数等。”
\end{yinyong}


教育,目的是为国家培养建设人才,国家需要什么样的人,就应该培养什么样的人。那么,国家需要什么样的人?
2022年2月28日,中央全面深化改革委员会第二十四次会议审核通过了\href{http://www.gov.cn/xinwen/2022-02/28/content_5676110.htm}{《关于加强基础学科人才培养的意见》},
这说明国家需要以“数、理、化”为代表的各种学科人才。


而我们的教育部呢,时常见他出新招,看起来动作不少,可结果是把自己搞得人憎狗嫌,人送外号“\textbf{禁学部}”。可谓出力不讨好,为什么?

首先,教育部这不让教、那不让学,视学科竞赛如洪水猛兽,拿出\textbf{扫黄打非}的劲头来整治培训班,
打着公平的名义,剥夺大家学习的机会。

其次,他又不对教材进行严格管理。一些省份自搞一套教材,这就给各种毒教材提供了天然的生长环境。
以江苏省为例,江苏使用的苏教版教材,还每年改版。有时我在想,就学校教的这点内容,有必要年年改版吗?
为什么不能出一个全国统一的教材,然后每隔几年再根据教学中发现的问题进行修正,就像
“\href{https://baike.baidu.com/item/%E4%BA%94%E5%B9%B4%E8%A7%84%E5%88%92/6544998}{五年计划}”一样。


% 在出台“奥数禁令”时,有几个理由:

% \begin{itemize}[nosep, left=\parindent]
%     \item 大约只有$5\%$的智力超常儿童适合学习奥数。 ——
%             这是一个无比正确的废话,不管学什么,都不可能人人学好。
%             中国有 \href{https://www.sohu.com/a/519935980_121124710}{3000万琴童},
%             可我除了知道郎朗、李云迪之外,没有怎么听说过其他的人。
%     \item 给家庭带了沉重的经济负担。 ——
%             这话说得,好像学语文、英语、艺术、体育都不用交钱似的。
%     \item 个别名校以奥数成绩作为招收学员的凭证,……加剧了教育资源的不均衡。 ——
%             这个倒是在理。不管是义务教育阶段的“学区+摇号”,还是中考的按成绩择优录取,
%             只要事前明确规则,事后公布名单和录取缘由,接受公众监督,解决起来也不难。
% \end{itemize}

% 窃以为,只要严格抓住学校录取的入口,“奥数禁令”完全没有必要。
% 不让学数理化、只让学艺(术)体(育)文(学)。
% 这个操作,倒是与美国近年来对中国留学生的政策相吻合,让人不得不产生怀疑。

%这种挖社会主义墙角的恶政一日不除,\textbf{禁学部}这个头衔就得一直戴下去。
