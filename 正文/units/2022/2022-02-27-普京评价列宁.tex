\begin{taged}{民族}
\section{2022-02-27 普京评价列宁}
\end{taged}

在微博上看到一段普京评价列宁的视频\footnote{\href{https://m.weibo.cn/status/4739964812135108}{微博原文},视频已收集在“素材”目录中。},感觉普京讲得很有道理。

“关于列宁和他在我国历史上的作用,以及我的看法,我认为他是一个革命家而不是政治家。

当我谈到我们国家1000年的历史时,我们都知道,它是中央集权和统一的。

但弗拉基米尔列宁提出了什么?

他比联邦更进一步,提出了一个可以说是邦联的制度。

这是他的决定,将民族群体与特定的领土捆绑在一起,这样他们就获得了从苏联分离的权利。

所发生的事情是,一个严格的中央集权国家变成了一个事实上的联邦。

享有分离权的民族群体被归入特定的领土。

但这些领土的划分方式并不总是与各民族传统上居住的地方相对应,而且至今仍不对应。

这就是在前苏联各共和国之间,甚至是在俄罗斯联邦内部的关系中出现的裂痕。

这些裂痕至今依然在。

这样的裂缝有两千条,稍微不注意哪怕一秒钟都会带来严重后果。
”

