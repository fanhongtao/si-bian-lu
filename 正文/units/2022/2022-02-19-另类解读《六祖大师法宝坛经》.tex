\begin{taged}{宗教}
\section{2022-02-19 另类解读《六祖大师法宝坛经》}
\end{taged}

一个青年,家境贫困,靠打柴为生,在机缘巧合之下,成为一位大师的弟子。
在他学有所成之即,遭人追杀,被迫亡命天涯…… 后来,他广收门徒,将毕生所学发扬光大。

只看上面的介绍,像不像某本武侠、修仙小说?但今天我们要讲的这位青年,不是小说中的主人公,
而是禅宗六祖惠能大师。

佛教中的文献由经、律、论三部分组成。
通常,“经是佛所说之经契”\footnote{https://baike.baidu.com/item/经律论},
也就是说,以“xxx经”命名的,都是由“佛”所说的。但有一个例外,那就是《六祖大师法宝坛经》\footnote{《六祖大师法宝坛经》已收录在“素材”目录中。}。

《六祖大师法宝坛经》简称《坛经》,是六祖惠能大师口述,由他的门人释法海集录的。

我不是法师,不能讲经说法,所以我的解读不是从佛法的角度来看,而是从世俗有角度来看。

\textbf{第一,《坛经》很切合宗教中国化。}

主人公惠能是中国人,《坛经》里所涉及的事情也都发生在中国境内。用现在流行的话来说,就是“讲好中国故事”。

\textbf{第二,对宗教人士做恶要保持足够的警惕。}

在一般人的印象中,佛教是最“与世无争”的宗教,所以现在有“佛系”一词。
但就在佛教内部,依然有着腥风血雨的一面。从“恐有恶人害汝”的预想,到“逐后数百人来,欲夺衣钵”的现实。
有数百人追杀惠能,要不是他跑得快,早就丧命了。

《坛经》虽是文言文写的,但内容浅显易懂,受过九年义务教育的人,都能不费力的读懂(字面意思)。
至于经文有没有其它的含义,咱也不知道。
《坛经》中有不少有趣的内容,值得一读。例如:“祖以杖击碓三下而去。惠能即会祖意,三鼓入室。”,
读这段内容,仿佛看到《西游记》中孙悟空被菩提祖师敲击三下,然后三更天跑去学习的情景。
由于《西游记》是明代吴承恩写的,可能是他读过《坛经》,然后将这一情节写入《西游记》。

