\toftagthis{宗教}
\section{2022-04-12 拒绝清真食品}

不知从什么时候开始,不少食品上都印有“清真”字样。

开始以为这是像 “绿色食品” 一样的某种食品安全认证,也没有太在意。后来才知道,这是\textbf{斯兰教}的认证。

或许有人要说,“清真”是民俗食品。那么什么是“民俗食品”?
顾名思义,源自某一个或几个民族特有的食品,可以叫民俗食品。
如:我们可以认为酥油茶是藏族的民俗食品。
但民俗食品不能有强制性、排他性。即:不能说只有藏族人、严格按照某种做法做的才能叫酥油茶,
而一个满族人按严格照做藏族的做法做出的东西就不能叫酥油茶。

“清真”食品是要符合斯兰教教义要求的食品,所以它只能是宗教食品。等什么时候推出的“清真猪肉”,再来讨论它的民俗食品属性。

而分析斯兰教的《古兰经》中的经文:

“终身不信道、临死还不信道的人,必受真主的弃绝,必受天神和人类全体的诅咒。他们将永居火狱,不蒙减刑,不获宽限”。\footnote{第二章 黄牛(巴格勒) \; 161 \; 162}

“你号召不信道者,就象叫唤只会听呼喊的牲畜一样。(他们)是聋的,是哑的,是瞎的,故他们不了解。 ”\footnote{第二章 黄牛(巴格勒) \; 171}

“不信道的人,他们的财产和子嗣,对真主的刑罚,绝不能裨益他们一丝毫;这等人是火狱的居民,将永居其中。 ”\footnote{第三章 仪姆兰的家属(阿黎仪姆兰) \;  116}

可以得出一个结论:它是一个仇视非斯兰教群众的宗教。而我本人,是一个无神论者,所以我拒绝“清真”食品。

不仅我拒绝“清真”食品,以下人群也应该拒绝“清真”食品:

\begin{itemize}[nosep, left=\parindent]
    \item 共产党员 —— 作为一个不能有宗教信仰的共产党员,成天吃宗教食品,你的党性原则在那里?
    \item 其它宗教 —— 以基督教信徒为例,拿你的主赐予你的财富,去买其他神灵的物品,这是打算叛主?
    \item 追求人人平等的的 —— 喊着追求平等,却在经济上支持不平等的宗教,这算是“嘴上喊不要,身体却很诚实”?
    \item ……
\end{itemize}

常见的“清真”食品:蒙牛、伊利这两家公司的所有产品、超市的“清真”牛肉……

