\begin{taged}{宗教}
\section{2022-07-16 割刑与七十二处女}
\end{taged}

女性\href{https://baike.baidu.com/item/%E5%89%B2%E7%A4%BC/2056506}{割礼},
“目的是割除一部分性器官以免除其性快感,并且确保女孩在结婚前仍是处女,即使结婚后也会对丈夫忠贞”。
这种野蛮残忍的做法,那有一丁点儿“礼”的成分在里面?所以我叫它为“\textbf{割刑}”。

割刑,表面上是宗教行为,本质上是对妇女的迫害。那么,为什么会有这样的需求?

说割刑有什么宗教意义,我是半点也不相信的。
按照宗教所宣称的,1)神是全能的;2)人是神创造的。
那么,为什么神要创造一个半成品的女人,还需要给她一刀才算合格?
究竟是神的无能,不能创造一个合格的人;
还是人的背叛,神给部件都敢割掉?
所谓“雷霆雨露,莫非天恩”,如果真相信人是神创造的,岂敢主动去损坏神给自己的身体?


我想,这或许可以从“72处女”\footnote{相关内容请自行百度。}这一愿景来考虑。

七十二这个数字,很有意思。小学的数学告诉我们:一天有 24 小时,而 $72 = 24 \times 3$。

假设有人真娶了妻妾 72 人,那么为了将所有妻妾都“宠幸”一遍,他需要以每小时3人(20分钟/人)的速度,
24 小时不停歇才能保证雨露均沾。就算是平常听男人吹牛,也只有人自夸“一夜七次郎”,
绝不会有人自诩“一夜七十二次郎”。

一个丈夫下班回家,如果家里的娇妻说“官人我要!”,那他可以回答说“来!”,说“官人我还要!”,也可以回答“再来!”。

可如果是家里有 72 个妻妾,人人都说“官人我要!”,这就不好玩了。有道是:
\begin{shici}
    原曾想,娶妻妾七二,后宫规模赛皇帝。\\
    怎料到,交公粮日夜,悲惨生活似男妓。
\end{shici}

这种场景,想想就很恐怖。为了掌握主动权,让妻妾们不会主动提出需求,
除了通过宗教的方式在思想上进行控制,告诉她们性是肮脏的之外,割刑就能派上用场。
在性发育之前,通过割掉女人的部分性器官,让她们没有性快感,这样她们就不会主动提出要求。
通过这种操作,主动权就又回到了男人的手里,世界又变得美好。



\contentsep

在不考虑人为干预的情况下,从统计学的角度看,生男、生女的概率大致相等,即男人和女人的数目大体相当。
那要怎么才能满足 1 男对 72 女(处不处的先不管)?
按可操作阶段的先后顺序排列,可能有以下几种方法:

\begin{itemize}[nosep, left=\parindent]
    \item 控 —— 在胎儿还在母体中,就预先判断出性别,打掉多余的男胎,在出生前就进行控制;
    \item 杀 —— 生下的男婴中,每 72 个只保留 1 个,其他都杀掉;(据说养殖场就是这种操作)
    \item 转 —— 保留少部分男婴,其他的通过手术转变性别;(LGBT?)
    \item 打 —— 每72个男人中,让其中的 71 个男人打光棍,剩下的 1 个就可以匹配 72 个女人了。
    \item 指 —— 正如美国《性别平等法案》那样的“指鹿为马”,连手术都省了。72 个彪形处“女”?
\end{itemize}

当然,或许还有我想不到的、更为“先进”的手段,但总脱离不了外力的干预。
问题是:施力者是谁?为何要施力?

