\toftagthis{宗教}
\section{2021-04-03 邪教论}

\subsection{引子}

假如你得到了:

\begin{itemize}[nosep, left=\parindent]
    \item 发财秘诀 —— 照此修为,必家财万贯、富可敌国;
    \item 武功秘籍 —— 照此修炼,能拔山盖世、万夫莫当;
    \item 长生秘法 —— 照此修行,可长生久视、寿与天齐;
\end{itemize}

首先肯定会自己修持,这点毋庸置疑。除此之外,你还会怎么做?

\begin{itemize}[nosep, left=\parindent]
    \item 藏着掖着,不让他人知晓 —— 敝帚自珍,这算是遵从人的本性;
    \item 小范围内分享,如,家人、师徒 —— 优待自己人,这也好理解;
    \item 有条件的公开 —— 法不轻传,设置条件也是应有之义;
    \item 无条件的公开 —— 人间大爱,可谓圣人境界。
\end{itemize}

会不会有人在得到秘诀、秘籍、秘法后,打着“我是为你好”的旗号,强制他人学习?什么,不学?neng死你!

\subsection{什么是邪教}

\href{http://www.chinafxj.cn/}{中国反邪教网} 的《\href{http://www.chinafxj.cn/c/2020-11-03/1286958.shtml}{如何识别和防范邪教—— “对邪教说不”签名活动宣讲提纲}》一文中,对“邪教”的定义如下:

\begin{screen}
    “邪教组织”,是指冒用宗教、气功或者以其他名义建立,神化、鼓吹首要分子,利用制造、散布迷信邪说等手段蛊惑、蒙骗他人,发展、控制成员,危害社会的非法组织。
\end{screen}

文中还给出了\textbf{如何辨识邪教}的方法:

\begin{itemize}[nosep, left=\parindent]
    \item 让你荒了田、抛了家、弃了学去信“神”的是邪教;
    \item 宣扬“世界末日”就要到了,只有加入他们的组织才能得救的是邪教;
    \item 有病不让就医,鼓吹入了“教”能治病、能消灾避难,搞“祷告治病”“驱鬼治病”的是邪教;
    \item 欺骗、威逼妇女受教主凌辱的是邪教;
    \item 说传统宗教过时了、要信新“神”的是邪教;
    \item 非法聚会时鬼鬼祟祟、乱喊乱叫、乱唱乱跳的是邪教;
    \item 让你用骗人的手段诱使他人加入的是邪教;
    \item 加入后不让退出的是邪教;
    \item 把社会、政府、普通老百姓当成“魔”的是邪教;
    \item 以“神”的名义煽动成员对抗政府的是邪教。
\end{itemize}

在我看来,除了上述的种种,还有一个更快捷的判断邪教的标准:

\begin{screen}
    宣扬使用使用武力的,都是邪教。
\end{screen}

怎么理解?邪教之所以“邪”,是因为它会对社会造成危害:

\begin{itemize}[nosep, left=\parindent]
    \item 对外:凡是不相信教义的,都是异端,要武力征服。
    \item 对内:凡是不遵从教义的,都不虔诚,要严加惩处;想要脱离退出的,更是十恶不赦,要追杀到底,生是我的人,死是我的鬼,还想退出?没门儿!
\end{itemize}

为什么邪教会伴随着武力?从根源上讲,是因为它\textbf{不(够)好}。

自身不好,还想让他人学习、遵从,除了诉诸武力之外,别无他法。

\begin{itemize}[nosep, left=\parindent]
    \item Jack Ma 如日中天时,创办湖畔大学,设定了严苛的报名条件,最后\href{https://zhuanlan.zhihu.com/p/265168680}{录取率仅4.07\%},堪称百里挑一。为什么这么多人报名,Jack Ma 是一位成功的商人,大家想要学习他的财富密码。
    \item UFC世界冠军\href{https://baike.baidu.com/item/张伟丽/19688961}{张伟丽},如果她现在要开班收徒,传授格斗技巧,会不会愁招不到学生?至少我是愿意去报个名,做个记名弟子。至于学不学得会,那是另外一回事。
    \item 假设有人宣讲长生,我肯定不会相信,但不妨碍我把他/她列为观察目标。等到某一天,我的子孙后代说:我爷爷的爷爷的爷爷在世时,就知道这位老神仙,我们带上礼物,去拜访他吧。真到那时,老神仙的武力,只会用在防止拜访的人太多,对其造成干扰上。反倒是我们需要准备好武力,谁敢阻挡,就是生死大敌。
\end{itemize}

\subsection{范氏定理}

讲到这里,我们可以归纳一下,形成以下定理:

\begin{itembox}[l]{范洪滔定理一}
    只要有榜样,就会有人主动学习。
\end{itembox}

\begin{itembox}[l]{范洪滔定理二}
    好东西,不需要武力强制推广;以武力强制推广的,都不是好东西。
\end{itembox}

以上定理,除了可以用来判断一个宗教是否是好宗教之外,还可用于辅助理解其它事物,如:美国推广的美式民主。


