\begin{taged}{民族}
\section{2023-01-16 中国的民族问题}
\end{taged}

我以为,中国目前存在较为严重的民族问题(隐患)。现象有:

\begin{itemize}[nosep, left=\parindent]
    \item 针对少数民族的学生的考试加分。
    \item 针对少数民族的补助。
    \item 公务员限定招某个(或某几个)少数民族。
    \item ……
\end{itemize}

现在有种说法叫:\zhongdian{一等洋人,二等官;三等少数,四等汉!}
给人的感觉就是中国现在依旧还是被蒙古族(元朝)或满族(清朝)统治,汉族人处于社会的最低层。
虽然宪法上说“中华人民共和国各民族一律平等”,可\zhongdian{现实告诉我们,这种平等,仅仅停留在纸面上}。

以考试加分为例,考试加分,
这很没有道理。如果说刚解放的时候,很多少数民族完全不会汉语,在全国统一考试的情况下,
适当加分作为一种补偿还有情可原。如今新中国已经成立70多年,还要加分到什么时候?
中考、高考乃至公务员的考试,每一分都可以压倒成百上千的竞争对手。
可就因为民族不同,有的人可以加上几十分。平等在那里?

我认为,\textbf{考试除了为烈士子女加分以外,其它的所有加分都应该取消。}
被评为烈士,说明当事人为了国家、人民的利益,献出了自己的生命。为其子女加分,谁也挑不出毛病来。
有意见?想为子女加分?自己也可以做先个烈士。
至于其他的人,其对社会所作的贡献,已经从经济、政治上获取了相应的回报,再加分就不适当。

公务员限招,就更是危害无穷。公务员,是选拔出来参与国家治理的人。
选拔不以能力而论出身的做法,连封建时代的科举制都不如。
在这样的制度下,只会强制民族隔阂。
如果一个民族的人,只能由该民族的人来领导,那么中国是不是至少应该分裂成56个国家?
因为按这个理论推导下去,不管谁来做国家主席,另外55个民族都不会同意,不服管理。
再考虑到中国有一些来自其它国家的移民,都照此办理,分成百国都不够。

“铸牢中华民族共同体意识,促进各民族像石榴籽一样紧紧抱在一起”这一提法很好。
但如果政府在日常工作中,对不同的民族进行区别处理,不断的强化各民族的特殊性,只会导民族之间致相行渐远。

时常听说,有人将自己的“民族”从“汉族”改为其它少数民族。
为什么会有这种操作?因为其中存在着巨大的套利空间!
加分、补助……,在各种民族政策的加持下,已然是人生赢家。
其后果,必然是民族之间的分裂与对抗。
在社会稳定、经济形式好的时候,大家忙着赚钱,其影响还不彰显。
可一旦大环境转差、蛋糕变小,“五胡乱华” 或 “驱逐胡虏,恢复中华” 这样的场景再现,也是意料之中的事情。
毕竟,那里有压迫,那里就有反抗。

好的民族政策应该是朝着促进不同民族之间交流、交融,最终实现民族的统一,都统一到“中华民族”。
据说,发生过种族大屠杀的卢旺达,
“为消除大屠杀带来的民族隔阂,增强国家凝聚力,政府倡导民族和解,不再区分民族,统一称卢旺达人”\footnote{https://baike.baidu.com/item/卢旺达/422809}。
这也许值得我们参考。

\contentsep

在我上小学的时候,有一个中途转学过来的同学好像是是藏族,听说是从西藏转来的。
那时候的我根本不关心这点,他和我们长得差不多,日常生活也没有明显的差异,
所以没有人会关心民族这回事,该怎么玩就怎么玩。

高中阶段,班上有两个汶川的同学,好像是羌族的。因为他俩成绩好,所以给人的感觉是,
汶川那个地方,人杰地灵。不过也就仅此而已。

我认为,让人不会去关注民族间差异的政策,才是好的民族政策。
而那些反复强调民族特殊性的政策,只会让不同民族之间产生隔阂,最终给国家的治理埋下隐患。

