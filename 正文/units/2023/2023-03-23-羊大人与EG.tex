\begin{taged}{国事}
\section{2023-03-23 羊大人与EG}
\end{taged}

历史上,外国人被视作未开化之人,被称之为\textbf{“蛮夷”}或\textbf{“鬼佬”}。

到了满清,因为科技落后,被外国入侵,给外国人一个高大上的称呼,叫\textbf{“洋人”}或\textbf{“洋大人”}。

如今,虽然我国的科技水平、综合国力已经提升上来了,但是国内仍然存在很多给外国人超国民待遇的事情。
为了表示对这一现况的不满,对外国人又有了新的称呼。
可能是外国人多喜欢抬着羊,所以他们被叫做\textbf{“抬羊倌”}或\textbf{“羊大人"}。

对外国人的超国民待遇,高校是一个重灾区,具体表现有:

\begin{itemize}[nosep, left=\parindent]
    \item 花费巨资招收留学生。按北师大教授胡必亮说法是:给每个留学生每年\textbf{才十万人民币}\footnote{\url{https://weibo.com/7000669644/4882197298024595}},还可以再多一点。现在据说已经涨到十五至二十万了。与这个数据对比的是:中国还有近六亿人,月收入不足2000元。
    \item 提供住宿条件明显好于国内学生的留学生公寓。留学生通常是每两人配置一套有空调、独立卫生间的套房,而国内的学生则多是四至六人一间房。
    \item 因为留学生不懂中文,学校又搞出针对留学生的英文授课、让国内学生给留学生写报告等操作。
    \item 更别提为(一些学校)为每个男性留学生配备的两个女性\textbf{“学伴”}。这等行径,就算学伴不是慰安妇,至少校方也是皮条客。不然为啥校领导不让自家女眷去作伴?
    \item ……
\end{itemize}

我不知道现在高校的思政课针对这样的现象是怎么讲的。
虽然国事咱不懂,但家事还是能理解的。
什么样的家庭才在自家孩子还饿着肚子时,先让别人家的孩子吃饱饭?
答曰:\zhongdian{包衣奴才侍奉落魄少主}。
为啥是“落魄少主”?少主不落魄,怎么会和奴才家的儿子抢饭吃!
为啥是“包衣奴才”?不是奴才,怎么会如此自甘下贱。

\contentsep

EG 是英文 Easy Girl 的缩写。Easy Girl 是指那些看见羊大人就往上扑女子,因其廉价,所以Easy。
在微博上经常可以看到这方面的帖子,这里就不针对现象展开说明。

Easy Girl 的产生,除了EG们自身的原因之外,我认为更多的在于社会上,特别是官方,还有一种“唯洋是崇”的做法。
用官方的话讲,这叫做缺少“民族自信”。
正所谓“上行下效”,有处处为羊大人提供超国民待遇的官方机构,自然会有携洋以自重的EG。
究其根源,在“官”而不在“民”。超国民待遇一日不除,EG现象就会重复上演。

