\begin{taged}{国事}
\section{2023-01-17 我的生育观}
\begin{center}
    写在中国人口减少之际
\end{center}
\end{taged}

根据国家统计局发布的《2022年国民经济顶住压力再上新台阶》\footnote{\url{http://www.stats.gov.cn/xxgk/sjfb/zxfb2020/202301/t20230117_1892123.html}}
“年末全国人口(包括31个省、自治区、直辖市和现役军人的人口,不包括居住在31个省、自治区、直辖市的港澳台居民和外籍人员)141175万人,比上年末减少85万人。
全年出生人口956万人,人口出生率为6.77‰;死亡人口1041万人,人口死亡率为7.37‰;人口自然增长率为-0.60‰。”和2021年相比,中国2022年的人口减少了85万人。

人口,是一个国家的根本,是最重要的资源。
设想一个人口日益减少的国家,最终只会是消亡于历史的长河之中,
其文明也必然断了传承。

由于前二三十年的计划生育政策,改变了人们的生育观念。
再加之高房价等现实因素,结合网络上各种鼓吹不婚不育的风潮,
导致中国的生育率下降的很厉害。
希望各级政府尽快推出促进生育的政策,同时严惩(严禁)不婚不育这种不正之风的传播。

\contentsep

我个人的生育观,可以用下面的打油诗来表示:

\begin{shici}
    一个两个太少,\\
    三个四个正好。\\
    五个六个偏多,\\
    七个、八个?\\
    (属猪呀,一生一窝)\\
    节育措施做好。\\
    (避孕套了解一下)
\end{shici}

