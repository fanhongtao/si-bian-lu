\toftagthis{科学, 宗教}
\section{2020-08-23 科学与玄学}

\subsection{什么是科学?}

按照\href{https://baike.baidu.com/item/%E7%A7%91%E5%AD%A6/10406}{百度百科}的定义,

\begin{screen}
    科学是一个建立在可检验的解释和对客观事物的形式、组织等进行预测的有序的知识系统,是已系统化和公式化了的知识。其对象是客观现象,内容是形式化的科学理论,形式是语言,包括自然语言与数学语言。
\end{screen}

什么,还是不明白。用人话来讲,就是:用可度量的方式来认识世界,就是科学。

“科学”有以下几个特点:

\begin{enumerate}[nosep, left=\parindent]
    \item 不可观察的东西是不存在的。
    \item (再近一步)只有能用数学描述的才有资格被称为科学。
\end{enumerate}

其核心要义就是:可观察、可度量。

\subsection{什么是玄学?}

“玄学”中的“玄”,起源于《老子》中的一句话“玄之又玄,众妙之门”。

“玄学”一词本是道教用语,这里我用它来统称在“科学”之外,对世界的各种解释,包括各种大大小小的宗教。

$$
\text{人类对世界的认识}
\begin{cases}
    \text{科学} \\
    \text{玄学}
    \begin{cases}
        \text{道教} \\
        \text{佛教} \\
        \text{基督教} \\
        \text{天主教} \\
        \text{伊斯兰教} \\
        \text{……}
    \end{cases}
\end{cases}
$$

与科学相对应,玄学的核心要义就是:不可观察、不可度量。

道教的“仙”,佛教的“佛”,其它教的“神”在那里?能观察吗?不能!能度量吗?不能!

\subsection{边界}

科学也好,玄学也罢,都只是人类试图了解世界的产物。

如果把世界的本质当作全集 $I$,科学作为集合 $S$,玄学作为集合 $M$,根据我的定义,很明显:

$$ I = S \cup M $$

那么两者的交集又是什么呢?

玄学与科学,是此(玄学)消彼(科学)长的关系。在人类历史初期,还没有科学,所以

$$ I =M $$

这也解释了为什么在远古时期会有满天神佛。连下个雨这么平常的事情,都有风师、雨伯、雷公、电母至少四位神仙,这还没有将龙王计算还内。

随着以“观察+度量”为主旨的科学的兴起,玄学的范围被极大地压缩,最明显的就是:神仙变少了。科学能解释的地方,神仙就没有了用武之地。可谓科学所到之处,鬼神辟易。但科学的发展并不是一蹴而就地,会有一个提出猜想,(试验)观察检验的过程。我们可以将那些科学还解释不了,但又正在试图解释的地方,当作科学与玄学的交集。以两军对垒进行比喻,这个交集,就是两军交战之处。

说了很多科学的好话,那么科学最终会全胜玄学吗?即:

$$ I = S $$

我以为,答案同样是肯定的:不能!原因同样很简单:人心难测!

\begin{enumerate}[nosep, left=\parindent]
    \item “我是谁?我从哪里来?我要到哪里去?”这一人类的终极问题,科学所给出的答案,并不能让人满意。就目前来看,科学仅能给出从受精卵开始到医学宣布死亡为止这段时间的解释。此前、此后,“我”在那里?没了吗?真的没了吗?想要“有”,该怎么办?
    \item 所谓“有人的地方,就有江湖”。江湖的恩怨情仇,这些事,科学大概是无法涉足的,这就给玄学留下了生存空间。
\end{enumerate}

所以,科学最终也只能是与玄学并存。
