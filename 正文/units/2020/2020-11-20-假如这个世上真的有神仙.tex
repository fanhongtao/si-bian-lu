\begin{taged}{宗教}
\section{2020-11-20 假如这个世上真的有神仙}
\end{taged}

假如这个世上真的有\textbf{神、仙},那么我希望能做到以下四个选项中的至少一个。

\subsection{我能成神仙}

既然“世人都说神仙好”,像我这样的上进青年,也是追求进步的,也想成为其中的一员。

\subsection{我能与神仙和平共处}

或许我这一世机缘未到、天资不足,不能光荣的成员神仙队伍中的一员,那我希望能与神仙们和平共处。

大家互通有无,共同发展,构建神(仙)界、凡间双循环相互促进的新发展格局,岂不快哉。

\subsection{我能远离神仙}

\textbf{和平共处,不是为奴为仆},我对成为神的“仆人”没有任何兴趣。至于“迷途的羔羊”?羔羊就应该乖乖地成为羊肉卷、羊肉串,羊肉汤也行。而对于那些乱认“神父”的人,我只能说,不要\textbf{\color{red}认贼作父},耗子尾汁!

如果不能与神仙们和平共处,那我希望有一块能远离神仙的乐土。仙凡永隔,也未尝不是一件好事,它其实是对弱小凡人的一种保护。

“逝将去女,适彼乐土。乐土乐土,爰得我所。” \footnote{《诗经》中的《国风·魏风·硕鼠》}

\subsection{我能屠神、诛仙}

如果即不能和平共处,又不能远离,那就只好一战。正所谓“那里有压迫,那里就有反抗”。

\textbf{神仙轮流做,今年到我家。} 能够参与到屠神、诛仙的事业中,此生也不算白过。嗯,\textbf{屠神英雄},这名号不错,很适合我,想想还有点小激动。

\subsection{结束语}

假如这个世上真的有神仙,我希望是:

\begin{itemize}[nosep, left=\parindent]
    \item 农神 \href{https://baike.baidu.com/item/后稷}{后稷}
    \item 兵仙 \href{https://baike.baidu.com/item/韩信}{韩信}
    \item 医神 \href{https://baike.baidu.com/item/华佗}{华佗}
    \item 诗仙 \href{https://baike.baidu.com/item/李白}{李白}
    \item 茶仙 \href{https://baike.baidu.com/item/陆羽}{陆羽}
    \item 战神 \href{https://baike.baidu.com/item/粟裕}{粟裕}
    \item 歌神 \href{https://baike.baidu.com/item/张学友}{张学友}
    \item ……
\end{itemize}

什么?你说还有\textbf{蚁力神}?那咱们先去考察一下,如果它确实有用,拿我的\textbf{封神榜} 给它封个神也不是什么大事。
