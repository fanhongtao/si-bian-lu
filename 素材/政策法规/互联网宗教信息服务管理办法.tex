\documentclass[UTF8, 11pt, oneside]{ctexart}

\usepackage{geometry}
\geometry{a4paper,left=2cm,right=2cm,top=2cm,bottom=1cm}

\usepackage{hyperref}
\hypersetup{colorlinks=true, linkcolor=red}

\newcommand{\juzhong}[1]{\begin{center}#1\end{center}}
\newcommand{\kongge}{\hspace{0.5em}}

\begin{document}

12月3日,国家宗教事务局令第17号公布了《互联网宗教信息服务管理办法》(以下简称《办法》),《办法》由国家宗教事务局、国家互联网信息办公室、工业和信息化部、公安部和国家安全部五部门联合制定,自2022年3月1日起施行。

《办法》根据《中华人民共和国网络安全法》《互联网信息服务管理办法》《宗教事务条例》等法律法规制定。《办法》坚持保障公民宗教信仰自由与维护国家意识形态安全相统一,坚持维护信教公民合法权益与践行社会主义核心价值观相统一,坚持规范互联网宗教信息服务与促进宗教健康传承相统一,坚持权利与义务相统一,体现保护合法、制止非法、遏制极端、抵御渗透、打击犯罪的原则。

《办法》共五章三十六条,明确从事互联网宗教信息服务,应当向所在地省级人民政府宗教事务部门提出申请,并对许可条件、申请材料、使用名称、受理时限等作了规定。明确网上讲经讲道应当由取得《互联网宗教信息服务许可证》的宗教团体、宗教院校和寺观教堂组织开展。明确除《办法》第十五条、第十六条规定的情形外,任何组织或者个人不得在互联网上传教,不得开展宗教教育培训、发布讲经讲道内容或者转发、链接相关内容,不得在互联网上组织开展宗教活动,不得直播或者录播宗教仪式。明确任何组织或者个人不得在互联网上以宗教名义开展募捐。

《互联网宗教信息服务管理办法》全文如下。\footnote{原文出自: \url{http://www.sara.gov.cn/qtldyw/364754.jhtml}}

\newpage

\juzhong{\Large\textbf{互联网宗教信息服务管理办法}}

\juzhong{第一章 \kongge 总则}

第一条 \kongge 为了规范互联网宗教信息服务,保障公民宗教信仰自由,根据《中华人民共和国网络安全法》《互联网信息服务管理办法》《宗教事务条例》等法律法规,制定本办法。

第二条 \kongge 在中华人民共和国境内从事互联网宗教信息服务,适用本办法。

本办法所称互联网宗教信息服务,包括互联网宗教信息发布服务、转载服务、传播平台服务以及其他与互联网宗教信息相关的服务。

第三条 \kongge 从事互联网宗教信息服务,应当遵守宪法、法律、法规和规章,践行社会主义核心价值观,坚持我国宗教独立自主自办原则,坚持我国宗教中国化方向,积极引导宗教与社会主义社会相适应,维护宗教和顺、社会和谐、民族和睦。

第四条 \kongge 互联网宗教信息服务管理坚持保护合法、制止非法、遏制极端、抵御渗透、打击犯罪的原则。

第五条 \kongge 宗教事务部门依法对互联网宗教信息服务进行监督管理,网信部门、电信主管部门、公安机关、国家安全机关等在各自职责范围内依法负责有关行政管理工作。

省级以上人民政府宗教事务部门应当会同网信部门、电信主管部门、公安机关、国家安全机关等建立互联网宗教信息服务管理协调机制。

\juzhong{第二章 \kongge 互联网宗教信息服务许可}

第六条 \kongge 通过互联网站、应用程序、论坛、博客、微博客、公众账号、即时通信工具、网络直播等形式,以文字、图片、音视频等方式向社会公众提供宗教教义教规、宗教知识、宗教文化、宗教活动等信息的服务,应当取得互联网宗教信息服务许可,并具备下列条件:

(一)申请人是在中华人民共和国境内依法设立的法人组织或者非法人组织,其法定代表人或者主要负责人是具有中国国籍的内地居民;

(二)有熟悉国家宗教政策法规和相关宗教知识的信息审核人员;

(三)有健全的互联网宗教信息服务管理制度;

(四)有健全的信息安全管理制度和安全可控的技术保障措施;

(五)有与服务相匹配的场所、设施和资金;

(六)申请人及其法定代表人或者主要负责人近3年内无犯罪记录、无违反国家宗教事务管理有关规定的行为。

境外组织或者个人及其在境内成立的组织不得在境内从事互联网宗教信息服务。

第七条 \kongge 从事互联网宗教信息服务,应当向所在地省、自治区、直辖市人民政府宗教事务部门提出申请,填报互联网宗教信息服务申请表,并提交下列材料:

(一)申请人依法设立或者登记备案的材料以及法定代表人或者主要负责人身份证件;

(二)宗教信息审核人员参加宗教政策法规和相关宗教知识的教育培训,以及具备审核能力的情况说明;

(三)互联网宗教信息服务管理制度、信息安全管理制度和技术保障措施材料;

(四)用于从事互联网宗教信息服务的场所、设施和资金情况说明;

(五)申请人及其法定代表人或者主要负责人近3年内无犯罪记录和无违反国家宗教事务管理有关规定情况承诺书;

(六)拟从事互联网宗教信息服务的栏目、功能设置和域名注册相关材料。

申请提供互联网宗教信息传播平台服务的,还应当提交平台注册用户管理规章制度、用户协议范本、投诉举报处理机制等。用户协议范本涉及互联网宗教信息服务的内容应当符合本办法有关规定。

互联网宗教信息服务申请表式样由国家宗教事务局制定。

全国性宗教团体及其举办的宗教院校从事互联网宗教信息服务,应当向国家宗教事务局提出申请。

第八条 \kongge 从事互联网宗教信息服务所使用的名称,除与申请人名称相同以外,不得使用宗教团体、宗教院校和宗教活动场所等名称,不得含有法律、行政法规禁止的内容。

第九条 \kongge 省级以上人民政府宗教事务部门自受理申请之日起20日内作出批准或者不予批准的决定。作出批准决定的,核发《互联网宗教信息服务许可证》;作出不予批准决定的,应当书面通知申请人并说明理由。

《互联网宗教信息服务许可证》由国家宗教事务局印制。

申请人取得《互联网宗教信息服务许可证》后,还应当按照国家互联网信息服务管理有关规定办理相关手续。

第十条 \kongge 从事互联网宗教信息服务,应当在显著位置明示《互联网宗教信息服务许可证》编号。

第十一条 \kongge 申请人取得《互联网宗教信息服务许可证》后,发生影响许可条件重大事项的,应当报原发证机关审核批准;其他事项变更,应当向原发证机关备案。

第十二条 \kongge 终止互联网宗教信息服务的,应当自终止之日起30日内,到原发证机关办理注销手续。

第十三条 \kongge 《互联网宗教信息服务许可证》有效期3年。有效期届满后拟继续从事互联网宗教信息服务的,应当在有效期届满30日前,向原发证机关重新提出申请。

\juzhong{第三章 \kongge 互联网宗教信息服务管理}

第十四条 \kongge 互联网宗教信息不得含有下列内容:

(一)利用宗教煽动颠覆国家政权、反对中国共产党的领导,破坏社会主义制度、国家统一、民族团结和社会稳定,宣扬极端主义、恐怖主义、民族分裂主义和宗教狂热的;

(二)利用宗教妨碍国家司法、教育、婚姻、社会管理等制度实施的;

(三)利用宗教宣扬邪教和封建迷信,或者利用宗教损害公民身体健康,欺骗、胁迫取得财物的;

(四)违背我国宗教独立自主自办原则的;

(五)破坏不同宗教之间、同一宗教内部以及信教公民与不信教公民之间和睦相处的;

(六)歧视、侮辱信教公民或者不信教公民,损害信教公民或者不信教公民合法权益的;

(七)从事违法宗教活动或者为违法宗教活动提供便利的;

(八)诱导未成年人信教,或者组织、强迫未成年人参加宗教活动的;

(九)以宗教名义进行商业宣传,经销、发送宗教用品、宗教内部资料性出版物和非法出版物的;

(十)假冒宗教教职人员开展活动的;

(十一)有关法律、行政法规和国家规定禁止的其他内容的。

第十五条 \kongge 取得《互联网宗教信息服务许可证》的宗教团体、宗教院校和寺观教堂,可以且仅限于通过其依法自建的互联网站、应用程序、论坛等由宗教教职人员、宗教院校教师讲经讲道,阐释教义教规中有利于社会和谐、时代进步、健康文明的内容,引导信教公民爱国守法。参与讲经讲道的人员实行实名管理。

第十六条 \kongge 取得《互联网宗教信息服务许可证》的宗教院校,可以且仅限于通过其依法自建的专用互联网站、应用程序、论坛等开展面向宗教院校学生、宗教教职人员的宗教教育培训。专用互联网站、应用程序、论坛等对外须使用虚拟专用网络连接,并对参加教育培训的人员进行身份验证。

第十七条 \kongge 除本办法第十五条、第十六条规定的情形外,任何组织或者个人不得在互联网上传教,不得开展宗教教育培训、发布讲经讲道内容或者转发、链接相关内容,不得在互联网上组织开展宗教活动,不得以文字、图片、音视频等方式直播或者录播拜佛、烧香、受戒、诵经、礼拜、弥撒、受洗等宗教仪式。

第十八条 \kongge 任何组织或者个人不得在互联网上成立宗教组织、设立宗教院校和宗教活动场所、发展教徒。

第十九条 \kongge 任何组织或者个人不得在互联网上以宗教名义开展募捐。

宗教团体、宗教院校和宗教活动场所发起设立的慈善组织在互联网上开展慈善募捐,应当符合《中华人民共和国慈善法》相关规定。

第二十条 \kongge 提供互联网宗教信息传播平台服务的,应当与平台注册用户签订协议,核验注册用户真实身份信息。

第二十一条 \kongge 未取得《互联网宗教信息服务许可证》的互联网信息传播平台,应当加强平台注册用户管理,不得为用户提供互联网宗教信息发布服务。

第二十二条 \kongge 从事互联网宗教信息服务,发现违反本办法规定的信息的,应当立即停止传输该信息,采取消除等处置措施,防止信息扩散,保存有关记录,并向有关主管部门报告。

第二十三条 \kongge 宗教事务部门应当加强对互联网宗教信息服务的日常指导、监督、检查,建立互联网宗教信息服务违规档案、失信联合惩戒对象名单和约谈制度,加强对互联网宗教信息服务相关从业人员的专业培训,接受对违法从事互联网宗教信息服务的举报,研判互联网宗教信息,会同网信部门、电信主管部门、公安机关、国家安全机关依法处置违法行为。

第二十四条 \kongge 网信部门应当加强互联网信息内容管理,依法处置违法互联网宗教信息。

第二十五条 \kongge 电信主管部门应当加强互联网行业监管,依法配合处置违法从事互联网宗教信息服务的行为。

第二十六条 \kongge 公安机关应当依法加强互联网信息服务安全监督管理,防范和处置互联网宗教信息服务中的违法犯罪活动。

第二十七条 \kongge 国家安全机关应当依法防范和处置境外机构、组织、个人,以及境内机构、组织、个人与境外机构、组织、个人相勾结在互联网上利用宗教进行的危害国家安全活动。

\juzhong{第四章 \kongge 法律责任}

第二十八条 \kongge 申请人隐瞒有关情况或者提供虚假材料申请互联网宗教信息服务许可的,宗教事务部门不予受理或者不予许可,已经许可的应当依法撤销许可,并给予警告。

擅自从事互联网宗教信息服务的,由宗教事务部门会同电信主管部门依据职责责令停止相关服务活动。

第二十九条 \kongge 违反本办法第十条、第十一条、第十四条、第十五条、第十六条、第十七条、第十八条、第十九条规定的,由宗教事务部门责令限期改正;拒不改正的,会同网信部门、电信主管部门、公安机关、国家安全机关等依照有关法律、行政法规的规定给予处罚。

第三十条 \kongge 互联网宗教信息传播平台注册用户违反本办法规定的,由宗教事务部门会同网信部门、公安机关责令互联网宗教信息传播平台提供者依法依约采取警示整改、限制功能直至关闭账号等处置措施。

第三十一条 \kongge 违反本办法规定,同时还违反《互联网信息服务管理办法》及国家对互联网新闻信息服务、互联网视听节目服务、网络出版服务等相关管理规定的,由宗教事务部门、网信部门、电信主管部门、公安机关、广播电视主管部门、电影主管部门、出版主管部门等依法处置。

第三十二条 \kongge 国家工作人员在互联网宗教信息服务管理工作中滥用职权、玩忽职守、徇私舞弊,依法给予处分。

第三十三条 \kongge 违反本办法规定,构成违反治安管理行为的,依法给予治安管理处罚;构成犯罪的,依法追究刑事责任。

\newpage

\juzhong{第五章 \kongge 附则}

第三十四条 \kongge 本办法施行前已经从事互联网宗教信息服务的,应当自本办法施行之日起6个月内依照本办法有关规定办理相关手续。

第三十五条 \kongge 本办法由国家宗教事务局、国家互联网信息办公室、工业和信息化部、公安部和国家安全部负责解释。

第三十六条 \kongge 本办法自2022年3月1日起施行。

\end{document}

