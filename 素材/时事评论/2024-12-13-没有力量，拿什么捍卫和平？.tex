\documentclass[UTF8,11pt,oneside]{ctexart}

\def\articletitle{没有力量,拿什么捍卫和平?}
\usepackage{CJKfntef}
\usepackage{float}

\usepackage{geometry}
\geometry{a4paper,left=2cm,right=2cm,top=2cm,bottom=1cm}

\usepackage{graphicx}

\usepackage{hyperref}
\hypersetup{colorlinks=true, linkcolor=red}

\linespread{1.6}

\usepackage{fancyhdr}
\usepackage{ifthen}
\pagestyle{fancy}
\fancyhf{}
\setlength{\headheight}{14pt}
\fancyhead[R]{\ifthenelse{\value{page}>1}{\thepage}{}}
\fancyhead[C]{\ifthenelse{\value{page}>1}{\articletitle}{}}
\renewcommand\headrulewidth{0pt}

\usepackage{tcolorbox}
\tcbuselibrary{skins}

\newcommand{\zd}[1]{\textbf{\textcolor[RGB]{123,12,0}{#1}}} % 重点

\newcommand{\yinyong}[1]{% 引用
    \begin{tcolorbox}[enhanced,
        frame hidden, interior hidden,
        before skip = 5mm, left skip=10mm,
        borderline west={5pt}{0pt}{gray!50}]
        #1
    \end{tcolorbox}
}

\newcommand{\dianping}[1]{% 点评
    \begin{tcolorbox}[enhanced,
        colframe=red!50!black,
        title=点评]
        #1
    \end{tcolorbox}
}

\newcommand{\xhx}[1]{%下划线(模拟微信中的划线功能,用于标注我个人认为的文章中精彩的地方)
    \CJKunderline*[thickness=1.5pt, format=\color[RGB]{84,216,140}]{#1}
}

\newcommand{\biaoti}[1]{% 标题
    \section*{#1}
}

\newcommand{\SetSectionType} {
    \ctexset{
        section={
            number = \chinese{section},
            aftername={、},
            format=\Large\bfseries,
        }
    }
}



\begin{document}

\begin{center}
    \LARGE{\articletitle\footnotemark}
\end{center}
\footnotetext{
    原文出自公众号“平原公子”的文章 《\href{https://mp.weixin.qq.com/s/IWMHs7ArP2KjeoSVq2r7Pg}{\articletitle}》
}

和平不是呼吁出来的,不是想象出来的,不是祈祷出来的,不是妥协退让出来的……历史上,呼吁和平呼吁得越多,和平就越少。

我们当然珍惜和平,当然愿意捍卫和平。

但用什么来珍惜和捍卫呢?

巴勒斯坦人珍不珍惜和平?叙利亚人愿不愿意捍卫和平?但他们能做到吗?所以单方面“呼吁和平”没有意义。

没有人喜欢战争,但你越是害怕战争,逃避战争,战争就越是来找你,不信你问问叙利亚人民,那从天而降的导弹、炸弹是怎么来的?那入侵国土的以色列军队是如何一夜之内出现的?

我们当年遭受了很多苦难,那不是简简单单几句话就能说的清的,反法西斯战争我们付出了3000多万条生命,华夏每一寸土地上都染着血……在那个最绝望的时候,“呼吁和平”管用吗?国际上的观瞻有用吗?就算有几个国际友人同情……救得了南京的30万无辜死难者吗?救得了举国3000万英魂吗?救得了在血海中受苦受难的四万万人吗?

所谓的“国际主流”,甚至就没有把我们当做“主流”,他们书写的世界史,二战的起点是1939年德国闪击波兰,但那个时候,南京大屠杀都已经发生两年了,七七事变发生在1937年,九一八事变发生在1931年,一二八事变发生在1932年,甲午中日战争发生在1894年……我们早就被军国主义法西斯侵略了很多年了,但“国际主流”直接忽略了我们。

所以我们的痛苦,他们能够感受到吗?我们的呼号,他们能够听到吗?人类的悲欢并不相通。

中国的抗战,并不是1937年开始的,也不是1931年开始的,而是1894年开始的。

1937年底南京城的日军是从哪里来的?是1937年8月上海淞沪会战的日军打过来的,那1937年8月上海的日军又是从哪里来的?那是因为1926年日军在中国上海成立了上海警备本部,第二年就直接在上海驻扎“海军陆战队”。

为什么日本能够在中国的上海驻军?

那是因为当年甲午中日战争、庚子事变八国联军侵华,日本获得了在中国东北、华北、上海的驻兵权。1901年《辛丑条约》中公然允许外国驻军导致的,日本“驻屯军”司令部设于天津海光寺,兵营分别设于海光寺和北京东交民巷,兵力部署于北京、天津、塘沽、秦皇岛、山海关、上海等地,也就是整个中国北方的核心地区,都驻扎着日军。

淞沪会战的日军主力,其实就是关东军,他们早就驻扎在中国东北了。

所以,日本敢于在上海发动“一二八”事变。在一二八事变后,蒋介石政府妥协,双方在签订《上海停战协定》,协定规定上海为“非武装区”,十九路军退出上海,中国不得在上海至安亭、昆山、苏州一带地区驻军,而日本可进驻“若干”军队。

再往前看,九一八事变,进攻沈阳北大营的日本关东军,是从哪里来的?

那是日俄战争后,日本获得了俄国在中国东北所有的权益,包括租借辽宁关东州,以及在中东铁路支线南满铁路附近驻军的权利。日本长期驻扎东北,对东北蚕食鲸吞,并且勾结奉系军阀,渗透东北军政,土肥原贤二和本庄繁都曾是张作霖的军事顾问,日军早就实际上控制了东北。

中国的抗战为何那么艰难?因为之前的清政府、北洋军阀、蒋介石政府早已用一次次的妥协,把中国的战略重地、军事要冲、厂矿工业、经济命脉都交出去了。卧榻之侧,都是端着刺刀的日本人,所以他们可以随时对我们发动进攻、为所欲为,日本军队甚至可以用我们的煤炭、钢铁资源制造武器,来屠杀我们。

正因为一退再退,才会退无所退,任人宰割。

与人为善就能获得对等的善意吗?

放下武器就能得到“宽恕”和“包容”吗?

想想叙利亚吧,他们不想要和平吗?但是美国、以色列、土耳其、各路宗教极端势力放过他们了吗?

在巴以冲突中,为什么我们看到巴勒斯坦人被集中起来押到万人坑前的画面……就会有忍不住的愤怒?因为我们也曾遭遇过这样的对待。

今天,我们确实和平了几十年了,而且是世界上最和平安定的国家,但我们的和平,是“呼吁”、“祈祷”出来的吗?不是,是因为新中国立国之战打出了威风和尊严,是因为我们自强不息发展工业和科技强大了,是因为我们手上握着“剑”……强盗不得不低头讲道理!

和平绝不是天上掉下来的、自然而然的,和平是需要用鲜血和生命去换的,和平是需要强大的力量去捍卫的。

和平很宝贵,但力量更宝贵。

正因为要捍卫和平,所以要拥有力量。

我希望,百年之后,千年之后,我们的孩子,看到的不再是我们受苦受难的雕像,而是我们这代人昂扬奋发、建功立业、一往无前的雕像,我希望我们的雕像树在那里,眼里不再有痛苦、迷茫,只有坚毅、果决、胜利的喜悦、解放全人类的勇敢和善良。

我希望我们这代人的灵魂,一半化作白鸽,一半化作钢铁,捍卫世世代代的和平。

\end{document}

