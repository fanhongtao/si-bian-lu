\documentclass[UTF8, 11pt, oneside]{ctexart}

\def\articletitle{死在出租屋的女孩,更多调查出来了}  % 作者修改了标题,我这里同步修改,但不修改文件名。
\usepackage{CJKfntef}
\usepackage{float}

\usepackage{geometry}
\geometry{a4paper,left=2cm,right=2cm,top=2cm,bottom=1cm}

\usepackage{graphicx}

\usepackage{hyperref}
\hypersetup{colorlinks=true, linkcolor=red}

\linespread{1.6}

\usepackage{fancyhdr}
\usepackage{ifthen}
\pagestyle{fancy}
\fancyhf{}
\setlength{\headheight}{14pt}
\fancyhead[R]{\ifthenelse{\value{page}>1}{\thepage}{}}
\fancyhead[C]{\ifthenelse{\value{page}>1}{\articletitle}{}}
\renewcommand\headrulewidth{0pt}

\usepackage{tcolorbox}
\tcbuselibrary{skins}

\newcommand{\zd}[1]{\textbf{\textcolor[RGB]{123,12,0}{#1}}} % 重点

\newcommand{\yinyong}[1]{% 引用
    \begin{tcolorbox}[enhanced,
        frame hidden, interior hidden,
        before skip = 5mm, left skip=10mm,
        borderline west={5pt}{0pt}{gray!50}]
        #1
    \end{tcolorbox}
}

\newcommand{\dianping}[1]{% 点评
    \begin{tcolorbox}[enhanced,
        colframe=red!50!black,
        title=点评]
        #1
    \end{tcolorbox}
}

\newcommand{\xhx}[1]{%下划线(模拟微信中的划线功能,用于标注我个人认为的文章中精彩的地方)
    \CJKunderline*[thickness=1.5pt, format=\color[RGB]{84,216,140}]{#1}
}

\newcommand{\biaoti}[1]{% 标题
    \section*{#1}
}

\newcommand{\SetSectionType} {
    \ctexset{
        section={
            number = \chinese{section},
            aftername={、},
            format=\Large\bfseries,
        }
    }
}



\begin{document}

\begin{center}
    \LARGE{\articletitle\footnotemark}
\end{center}
\footnotetext{
    原文出自公众号“燕梳楼”的文章 《\href{https://mp.weixin.qq.com/s/0za9XAa3BZV_k66H5e9Ohg}{\articletitle}》
}

昨天取关我的人,今天估计都后悔了。

因为我说的是基于事实,看似没有人情味,恰恰可以一掌拍醒很多人。

我以前就说过,我写文立言不是为了贩卖鸡汤或趋炎附势,而是希望能穿透眼花缭乱的社会热点,给大家有用的启发和实践。

关于《一个女大学生死在我出租屋》的网文,我昨天的评论确实是酒后慷慨了些,但真相确实就如我所说的那么扎心,不管你愿不愿意,事实都在那里。

不过我还是要为一些不恰当的表述向女同胞们道歉。文中使用了大龄剩女、中年妇女等描述性词语,导致一些女读者由此产生误解甚至愤怒,我觉得这是我的问题。

还有对于抑郁症群体的另一面关注不够,也让昨天的评论显得不够温情甚至冷漠。对此我接受热心读者的批评。我希望能在一些公共事件中更好和大家对话,并实现共情。

作为一个有着20多年从业经历的老新闻人,对于“一个女大学生之死”这样的网文一眼就能看出问题在哪儿。果然,今天随着更多真相流出,这篇网文除了女子死在出租屋是真的之外,大部分细节全是脑补的。

比如女子并不是毕业于211名校,只是一个专升本的普通院校,学校叫北方工业大学;1万元的公寓房租也不是家里借来的,家里经济条件并不没有那么差;骨灰也不是随手扔了,而是按风俗撒到了河里。

这其中最撩拨公众神经的,其实就是小作文中关于女孩“几次考公笔试第一,都在面试环节被卡掉了”。这里面的映射大家都懂,无非是想说考公考编都是有黑幕的,反正只要没录取就是有黑幕。

没有黑幕,为什么笔试第一却过不了面试?而且还多次笔试第一?这不就是欺负这女子没背景没关系吗?每年考公大军那么多,最后成功上岸的确实寥寥无几,所以女子遭遇很容易让人代入。

但事实如何呢?根据当地官方公布,离世女子分别在2018、2019和2022年三次参加公考,笔试成绩分别排在第133名、第65名和第25名。也就是说,女子最好的成绩是25名,并不存在笔试第一之说。

至于多次笔试第一就更是子虚乌有的事情了,所以离世女子几次考公都未能进入面试环节,也就不存在面试被淘汰的问题。体制内的都懂,很多岗位也就一两个名额,即使最好成绩是25名,也不可能有机会进入面试。

其实看到这里,大家对这篇咪蒙体小作文基本上也差不多看明白了,除了确实有女子死在出租屋外,其他细节都是情节和流量需要,211大学生,多次考公笔试第一被刷,最后饿死在西安公寓,这些关键词叠加在一起很难不引发关注。

我之所以第一反应就是咪蒙《一个寒门状元之死》的套路,就是因为这太不符合常识,很多人都当过房东,除了催租等必要沟通外,平时不会有更多联系,那么多的细节从何而来?那么就只能靠编。

那么作者为什么要编这个故事呢?很简单,为了流量。套用一句很操蛋的网络流行语来说,“即使这一切都不是事实,你们也没损失什么”。但真的没有损失什么吗?

这背后对公信力的冲击,对就业环境的悲观,对考公考编体制的质疑,对学历贬值读书无用论的默认,对农村阶层流动的嘲讽,对社会舆论的撕裂,都能在小作文中找到隐喻。

当然这可能只是一种偶然作用的结果。作者可能觉得这是一个很凄凉的故事,而贞观公众号也觉得这故事能带来流量,所以就有了这篇女大学生之死,他们最大的问题就是没有标注:小说体。

所以谁在带节奏又是谁在吃人血馒头,已经很清楚了。无论这女子自杀也罢饿死也罢,是巨婴也罢绝望也罢,她都人畜无害,只想一个人悄悄的离开这个世界。

或许是她攒够了失望决意离开,也或者是抑郁成疾猝死在家中,她都不该被拿出来被议论和消费,甚至还要被口水再社死一次,这一定不是她想要的。

而对于我们来说,就是要客观看待这样一起偶发孤立事件,这是个案不是普遍,社会不是这样子的,读大学是有用的,不是所有考公都有黑幕的。

当然,我们也要看到整个社会面的就业压力,尤其是大学生群体的生存现状,即使这是她的个体选择,社会也有义务去关注和纾解这种悲剧。

在北京读大学,家人供养7年考公,有品质的公寓生活,能做到这些已经不算很差了,更不可能到了饿死的地步,除非她自己去意已决。

\textbf{我身边有很多凭实力考公考编成功上岸的例子,虽然不排除有暗箱操作的现象,但我相信这不是主流,更不是攻讧的借口。}

\textbf{只有相信相信的力量,相信努力的意义,相信这世界有光,才能去追逐梦想,而不是被黑暗埋葬。}

\textbf{所以我也希望我们燕梳楼的读者们,可以用世界的客观规律交流而不是互换生活的不如意。}

\textbf{\xhx{一个人走向成熟,不是知其奈何而戕身伐命,而是知其不可奈何而安之若命。}}

\zd{\xhx{活着,才是对生命最大的尊重,才是对命运最好的抗争!}}

\end{document}

