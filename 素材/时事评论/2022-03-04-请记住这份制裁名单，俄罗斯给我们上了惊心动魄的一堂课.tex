\documentclass[UTF8, 11pt, oneside]{ctexart}

\usepackage{geometry}
\geometry{a4paper,left=2cm,right=2cm,top=2cm,bottom=1cm}

\usepackage{hyperref}
\hypersetup{colorlinks=true, linkcolor=red}

\usepackage[shortlabels]{enumitem}

\linespread{1.6}

\setlist[itemize]{nosep, left=\parindent}

\begin{document}

\begin{center}
    \LARGE{请记住这份制裁名单,\\ 俄罗斯给我们上了惊心动魄的一堂课}
\end{center}

今年开始,俄乌战争爆发,中国又可以摸着俄罗斯过河:

金融圈、科技战、舆论战、信息战、网络战、电子战、经济战、贸易战、资源控制战、武器
热战等……

未来某一天,中国可能也要面对这一切。

大家可以看看哪些是可以承受的、哪些是不可承受的、哪些是可以取而代之的。

\section{金融战:}

\begin{itemize}
    \item SWIFT —— 部分俄罗斯银行被踢出
    \item 苹果支付 —— 限制支付功能
    \item 谷歌支付 —— 无限期暂停使用
    \item PayPa —— 停止俄罗斯新用户注册
    \item DMarket —— 冻结俄罗斯和白俄罗斯用户资产
    \item 伦交所、纽约交易所 —— 俄罗斯公司股票除名
    \item Cexlo —— 加密货币网站禁止俄罗斯用户使用
    \item 瑞士 —— 一个永久中立国,脸都不要了,打算冻结了俄罗斯存在瑞士银行的资产。
\end{itemize}

从 SWIFT 到支付系统,再到俄罗斯上市公司在海外的业务和和融资,俄罗斯要么被剔出群,要么被勒令退出市场。

相较于俄乌正面战场,欧美发动的金融战,更加血雨腥风,甚至直接清零。



\section{科技战:}

\begin{itemize}
    \item AMD —— 对俄罗斯断供芯片
    \item Intel —— 停止向俄罗斯供应芯片
    \item 台积电 —— 停止向俄罗斯供应芯片
    \item 格芯 —— 断供俄罗斯芯片
    \item 苹果电脑 —— 暂停在俄罗斯的所有产品销售
    \item 戴尔 —— 暂停向俄罗斯发货
    \item 半导体设备 —— 韩国日本向俄出囗被限制
    \item 爱立信 —— 暂停向俄罗斯发货
\end{itemize}

自从中兴华为被锤后, 这几年中国默默把卡脖子赛道啃下。

大基金负责找钱, 中国烟草负责掏钱, 证监会负责上市。

华为海思负责设计, 沪硅负责硅片, 中微负责蚀刻, 上微负责光刻, 鼎龙负责抛光垫, 中芯
国际负责制造, 兆易汇顶负责封测。

华为说, 只要我一天不死, 就和大家努力自建 IDM 的一天。


\section{网络舆论战:}

\begin{itemize}
    \item 推特油管脸书 —— 封锁俄罗斯账号, 谁帮俄罗斯说话就封谁
    \item 英国广播公司 —— 吊销广播执照
    \item 欧洲电视网 —— 取消俄罗斯资格
    \item 世界移动通信大会 —— 拒绝对俄罗斯代表团的认证
    \item 星链网络 —— 向乌克兰提供网络
    \item 环球影业 —— 阻断发行
    \item 华纳兄弟 —— 取消所有电影发行
\end{itemize}

蛇岛守军阵亡, 结果发现投降;

最大运输机安-225 被摧毁, 又出来辟谣

第一大坝说被俄罗斯炸, 结果中国卫星过去一看, 完全没这回事。

可想而知, 中国互联网防火墙、中国北斗卫星和本土数字经济是多么高目詹远瞩。


\section{信息战}

\begin{itemize}
    \item 希拉里 —— 呼吁黑客对俄罗斯进行攻击
\end{itemize}

世界最大的黑客组织宣布参战, 一上来就把俄罗斯的大外宣网站给干掉了。

\begin{itemize}
    \item 美国总统国务卿 —— 对俄打信息战, 还加大力度在俄罗斯政界招募间谍。
    \item 谷歌地图 —— 暂时关闭在乌克兰的谷歌地图实时交通信息
\end{itemize}

现在才知道北斗卫星的重要性吧, 没有导航分分钟被包抄, 只有自立自强才能不被掣肘。


\section{贸易战:}


\begin{itemize}
    \item 奥迪 —— 停止向俄罗斯发货
    \item 宝马 —— 停止向俄罗斯发货
    \item 戴姆勒 —— 暂停业务
    \item 沃尔沃 —— 暂停业务
    \item 福特 —— 暂停业务
    \item 现代 —— 暂停业务
    \item 通用 —— 暂停业务
    \item 大众汽车 —— 暂停业务
    \item 哈雷摩托 —— 停止发货
    \item 马士基 —— 停止运往/ 来自俄罗斯的货物;
    \item 三菱 —— 暂停在俄罗斯生产和销售汽车
    \item 耐克 —— 暂停在俄线上销售业务
    \item 波音 —— 暂停向俄罗斯出售波音飞机的零配件, 同时停止提供对波音飞机的维护保养和技术服务。
\end{itemize}

波音这个可真是要命的, 一架飞机两三百人呢。假如真因为没有例行维护而坠机的话,波音、空客是不是属于故意杀人。

未来有一天, 波音也会给我们玩这招, 自主发展自己大飞机是多么重要。


\section{资源控制战:}

\begin{itemize}
    \item 英国石油公司 —— 退出俄罗斯石油公司20 % 的股份
    \item 壳牌(Shell) —— 退出在俄罗斯的合资事业
    \item 埃克森美孚(Exxon Mobil) —— 退出在俄罗斯的合资事业
    \item 挪威国家石油 —— 退出相关合作
    \item 西班牙雷普索尔 —— 退出相关合作
    \item Eni 石油公司 —— 退出相关合作
    \item 道达尔能源 —— 不再投资新项目
\end{itemize}


\section{武器热战:}

\begin{itemize}
    \item 德国 —— 1000 件反坦克武器和500 枚毒刺地对空导弹
    \item 加拿大: 700 万加元武器。
    \item 澳大利亚: 700 万澳元武器。
    \item 波兰: 武器若干。
    \item 荷兰: 武器若干。
    \item 美国: 武器若干
\end{itemize}

德国要增加军费,日本要发展核武器, 从此打开潘多拉魔盒


\section{运输战:}

\begin{itemize}
    \item 联邦快递 —— 暂时停止向俄罗斯发货
    \item UPS  —— 暂时停止向俄罗斯发货
    \item 美国 —— 禁止俄罗斯飞机进入领空
    \item 欧盟 —— 对俄罗斯实施整体关闭领空
    \item 加拿大 —— 不止禁飞, 还禁止俄罗斯船只进入
\end{itemize}

\textbf{制裁范围越演越大, 体育界、美术界、音乐界、文学界、教育界、电影界、动物界都开始驱赶俄罗斯人。}



\section{体育无国界?}

\begin{itemize}
    \item 阿迪达斯 —— 暂停与俄罗斯足协的合作关系
    \item 世界跆拳道联盟 —— 撤销普京荣誉黑带九段称
    \item 国际柔联 —— 撤销普京荣誉称号
    \item 奥组委 —— 禁止俄罗斯参赛
    \item 国际排联 —— 禁止俄罗斯参赛
    \item 国际滑联 —— 禁止俄罗斯参赛
    \item 国际足球 —— 禁止俄罗斯参赛
    \item 国际乒联 —— 禁止俄罗斯参赛
    \item 国际篮联 —— 禁止俄罗斯参赛
    \item 国际泳联 —— 禁止俄罗斯参赛
\end{itemize}


\section{艺术无国界?}

\begin{itemize}
    \item 慕尼黑爱乐和鹿特丹爱乐 —— 辞退俄罗斯音乐指挥家
    \item 苏黎世歌剧院 —— 解除俄国女高音演唱合同
    \item 迪士尼 —— 暂停在俄罗斯上映电影
    \item 希腊 —— 取消天鹅湖演出
    \item 欧洲电视网 —— 宣布剥夺俄罗斯人的参赛资格
\end{itemize}



\section{学术无国界?}

\begin{itemize}
    \item 欧洲大学 —— 开除俄籍学生
\end{itemize}

俄罗斯德国合作的太空望远镜上的德国设备,已被德国远程关闭。

\begin{itemize}
    \item NASA  —— 除了国际空间站, 冻结与俄方一切太空合作。
    \item 意大利米兰 —— 禁俄罗斯作家的书籍
\end{itemize}



\section{游无国界?}

\begin{itemize}
    \item CDPR  —— 宣布停止向俄罗斯和白俄罗斯贩售游戏
\end{itemize}



\section{动物无国界?}

\begin{itemize}
    \item 国际爱猫联盟 —— 在俄人士的猫不得参加其展
\end{itemize}


\section{...}

\textbf{提醒一下中国土豪}

在欧美有巨量资产的中国土豪富婆们提个醒,你的钱不一定总是你的。

\begin{itemize}
    \item 德国 —— 没收俄富豪价值近6 亿美元超级游艇
    \item 美国 —— 俄富豪在美别墅游艇等财产没收
    \item 瑞士 —— 考虑冻结俄罗斯在瑞士资产
\end{itemize}

\vspace{1em}

\textbf{还有有三家公司行为非常别致、非常出众:}

\begin{itemize}
    \item tiktok  —— 在欧盟屏蔽俄罗斯两大官媒播报文
    \item 联想 —— 停止向俄罗斯供货
    \item 好丽友 —— 仅在俄罗斯与中国进行涨价, 其他国家不涨价
\end{itemize}

\vspace{1em}

以后, 谁要是再说下面这几词, 就用这些例子扇他一个大耳光:

\begin{itemize}
    \item 《体育无关政治》
    \item 《科学没有国界》
    \item 《艺术不分国家》
    \item 《诺贝尔和平奖》
    \item 《永久中立国家》
    \item 《誓死捍卫你说话的权利》
    \item 《私有财产不受侵犯》
\end{itemize}


\section{最后:}

前世苏联还是今生俄罗斯, 都走在他南方邻居的前面把雷全部踩了一遍, 真是世界级好老师。

俄罗斯还提醒我们: 打仗不仅仅是枪炮, 还有金融战、舆论战、科技战, 现代战争讲的是综
合抵抗力, 任意一个短板都会导致战争输得一败涂地。

这一次大型实战化观摩课, 从俄罗斯闪电战战术, 到欧美每一项制裁, 再到不同层面的战
争, 中国都要认真评估分析, 提前做好准备。
\footnote{
    录注:我看到的原文出自\textbf{微信}公众号 “贩财局”,3月4号的文章,本想第二天收录,结果发现文章被禁。
    所幸今天(3月6号)在微博用户 “\href{https://m.weibo.cn/profile/5559135126}{袍哥说事}” 上找到 \href{https://m.weibo.cn/status/4744003536814848}{内容的图片},这才有机会收录。 }
\footnote{和原文相比,文字没有改动,但排版上有以下细节不同:1. 多了小节的编号(方便在PDF中按小节跳转)。2. 所举的例子的前有小圆点,更容易与描述的文字区分。}

\end{document}

