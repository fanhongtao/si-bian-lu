\documentclass[UTF8,11pt,oneside]{ctexart}

\def\articletitle{如果无法阻止和化解这场危机,中国将不战而败!}
\usepackage{CJKfntef}
\usepackage{float}

\usepackage{geometry}
\geometry{a4paper,left=2cm,right=2cm,top=2cm,bottom=1cm}

\usepackage{graphicx}

\usepackage{hyperref}
\hypersetup{colorlinks=true, linkcolor=red}

\linespread{1.6}

\usepackage{fancyhdr}
\usepackage{ifthen}
\pagestyle{fancy}
\fancyhf{}
\setlength{\headheight}{14pt}
\fancyhead[R]{\ifthenelse{\value{page}>1}{\thepage}{}}
\fancyhead[C]{\ifthenelse{\value{page}>1}{\articletitle}{}}
\renewcommand\headrulewidth{0pt}

\usepackage{tcolorbox}
\tcbuselibrary{skins}

\newcommand{\zd}[1]{\textbf{\textcolor[RGB]{123,12,0}{#1}}} % 重点

\newcommand{\yinyong}[1]{% 引用
    \begin{tcolorbox}[enhanced,
        frame hidden, interior hidden,
        before skip = 5mm, left skip=10mm,
        borderline west={5pt}{0pt}{gray!50}]
        #1
    \end{tcolorbox}
}

\newcommand{\xhx}[1]{%下划线(模拟微信中的划线功能,用于标注我个人认为的文章中精彩的地方)
    \CJKunderline*[thickness=1.5pt, format=\color[RGB]{84,216,140}]{#1}
}

\newcommand{\biaoti}[1]{% 标题
    \section*{#1}
}

\newcommand{\SetSectionType} {
    \ctexset{
        section={
            number = \chinese{section},
            aftername={、},
            format=\Large\bfseries,
        }
    }
}



\begin{document}

\begin{center}
    \LARGE{\articletitle\footnotemark}
\end{center}
\footnotetext{
    原文出自公众号“李光满说”的文章 《\href{https://mp.weixin.qq.com/s/mnCOe2aublqolL4377j5Kg}{\articletitle}》
}

当前,中国正面临一场重大危机,如果不能阻止和化解这场危机,中国将在全球竞争中不战而败。如此说并非耸人听闻,而是真实存在的,我们必须立即采取措施,以阻止和化解这一危机的发生及影响。

这场重大危机就是人口负增长危机。

我们且看下面这组数据:近几年中国人口变动情况:2018年净增人口530万人,2019年净增人口467万人,2020年净增人口204万人,2021年净增人口48万人,2022年净减少人口85万人,2023年净减少人口208万人,当年人口自然增长率为-1.48‰。

这一组数据清晰地告诉我们,我国人口正在形成净减少的趋势,我们的人口自然增长率已经出现负数。

再看2021年以来这组数据:2021年新生儿数量1062万,出生率7.52‰,2022年新生儿数量956万人,出生率为6.77‰,2023年的新生儿数量为902万人,创下1949年以来最低纪录,出生率为6.39‰。2024年上半年新生儿数量只有433万人,全年可能低于900万人。2023年中国新生人口数量跌破1000万大关,与1963年的近3000万峰值相比,减少了整整2000万。

最不可思议的是,中国新生儿出生率出现断崖式下跌的近五年,正是中国放开三胎生育政策后出现的状态。为什么放开生育政策反而出现新生儿断崖式下跌?

我们再看一组数据:2023年全球总和生育率为2.3,印度为1.98,法国为1.68,美国为1.62,塞尔维亚为1.5,英国为1.44,德国为1.35,芬兰为1.26,日本为1.22,意大利为1.20。中国总和生育率1970年代之前为6左右,1990年为2左右,2010年为1.5左右,2022年1.05,2023年在1.0左右,在全球主要经济体中倒数第二(仅比韩国略高)。

总和生育率是指平均每对夫妇生育的子女数,国际上通常以2.1作为人口世代更替水平,即平均每对夫妇大约需要生育2.1个孩子才能让出生和死亡逐渐趋于均衡,低于1.5的生育率被称为“很低生育率”,一旦总和生育率降至1.5以下并持续一段时间,就有可能跌入“低生育率陷阱”。

2010年我国总和生育率为1.5,已进入很低生育率,到2023年,直接断崖式下跌到1,已经进入严重“低生育率陷阱”。

这里我要提醒大家注意三个要点。

一是我国人口出生率断崖式下跌出现了2016年我国放开二孩政策之后,特别是2021年放开三孩政策之后直接下跌到了1,这一点很能说明问题,也很值得我们思考。

二是我国的总和生育率在全球主要经济体中排在除韩国外的倒数第二,比欧洲、美国、日本、印度都要低,我们总说日本出生率怎么怎么低,现在看中国,你可能会惊掉下巴。

三是2023年印度的总和生育率为1.98,出生率为16.3‰,出生人口数约为2306万至2322万人。印度已经取代中国成为全球第一人口大国,中国已经不再是世界第一人口大国。从趋势看,以后每年印度都会比中国多增加1300万以上新生人口,这一点将会对未来中印竞争产生关键性影响。

与此同时,我国逐渐进入中度或重度老龄化社会。截至2023年底,全国60周岁及以上老年人口已达2.97亿人,占总人口的21.1\%,65周岁及以上老年人口达到2.17亿人,占总人口的15.4\%。根据测算,未来10年内,60岁及以上的老年人口每年净增超过1000万人,预计2035年前后将突破4亿,到2050年左右将突破5亿。

10月11日,北京发布的报告显示,2023年北京市60岁及以上常住人口为494.8万人,占总人口的22.6\%,60岁及以上户籍人口为431.6万人,占户籍人口的30.2\%,首次突破30\%。

10月28日,江苏发布的报告显示,截至2023年末,江苏60岁及以上常住老年人口2089万人,占常住人口的24.5\%,65岁及以上老年人口1573万人,占比18.4\%,江苏省13市全部进入中度老年化社会,其中南通、泰州、盐城和扬州已进入重度老龄化社会。

所有数据都表明,我国在人口方面面临的严峻形势,一方面是严重的低生育率,即陷入严重的“低生育陷阱”,年轻人变少了。另一方面是陷入严重的中重度老龄化社会,老年人变多了。这一并非什么好事的“双向奔赴”,必定会让我们意识到我国人口问题的极端严重性。

10月28日,国务院办公厅发布了支持生育的若干措施,其核心内容是完善生育支持政策和激励机制、健全人口服务体系、降低生育养育教育成本,营造尊重生育、支持生育的良好氛围。我感觉,这一文件尽管对解决我国生育问题提出了一定的支持措施和鼓励政策,但相较于计划生育时期将计划生育列为三大国策之一,这一若干措施与当前我国所面临的严重的人口问题相比,仍然显得层次不够且份量不足。

到底该如何认识和应对中国已经出现的人口危机?我从以下二十个方面进行分析和判断,给大家提供一些解决这一问题的思路。

第一,有一个现象值得我们首先关注,那就是在中国这样一个最有家庭观念、最讲“不孝有三,无后为大”、最讲养儿防老、最讲传种接代的社会怎么突然就变成不想生孩子、不愿生孩子、许多人也确实不生孩子了呢?一种深入骨髓的传统文化本来是极难改变的,现在仅仅过了几十年,我们整个社会的生育观、家庭观、价值观、文化观就发生了如此巨大的变化甚至逆转,其中到底发生了什么?到底是一股什么力量在推动这一巨大且几乎是不可能完成的转变?

第二,我们都知道有一个小品叫“超生游击队”,在那样严厉打击超生的高压环境中,仍然有人宁可受严厉处罚也要多生孩子。还有一件事是张艺谋宁愿被罚数百万元,也要多生孩子,这两件事告诉我们什么?在那样一个视多生孩子为洪水猛兽的高压环境和生活并不并富裕的年代,仍有大量年轻人想要生二胎甚至生三胎,为什么现在国家强大了、家庭富裕了,年轻人却失去了结婚和生育意愿?这其中到底发生了什么?根源到底在哪里?是真的生不起养不起还是有其它更可怕的因素干扰?

第三,中国现在是全球主要经济体中除韩国外生育率最低的国家,已经进入负增长状态,已经形成越来越严重的新生人口断崖式下跌趋势。趋势一旦形成,就很难改变和扭转,这将会形成一种长期的状态。我们曾经以为在放开二胎、三胎政策后会出现生育率的回升甚至会出现一个新的生育高峰,然而这种情况并没有出现,而且我们国家已经开始鼓励生育了,生育率却仍然出现了持续下跌,这其中的原因到底是什么?

第四,纵观历史、横看世界,只有中国实行了一个家庭只生一个孩子的最彻底的计划生育政策。所有人都知道,一个家庭只生一个孩子是无法使现有人口、家庭、家族更替平衡的,是会对中国社会及家庭结构带来毁灭性冲击的。中国在八十年代初就将计划生育作为三大国策之一写入宪法,此后又制定了人口与计划生育法,这在世界上都是独一无二的存在。我们要问的是,到底是一股什么力量推动了这一进程?为什么我们在实行计划生育的时候不能是一个家庭可以生两个孩子而不是只能生一个孩子?如果当初实行一个家庭可以生两个孩子,中国的家庭、家族、文化传统都不会受到如此大的毁灭性冲击,可我们为什么如此决绝地要推行一个家庭只能生一个孩子的政策呢?

第五,长期以来,一些西方学者著书立说,鼓吹人口爆炸论、资源有限论、粮食危机论和能源危机论,认定地球人口最多只能够承载35亿等等,我国也有一些所谓的专家学者对此进行呼应,极大的影响了中国的决策。现在全球人口不是35亿,而是已经突破70亿,西方发达国家都在想方设法鼓励生育,唯有我国仍长期实施严厉的计划生育政策。对此我们是否需要调整思路,摆脱西方人口理论的洗脑和束缚,着眼于中华民族未来50年、100年甚至500年和1000年的繁衍、发展以及在全球大国竞争中的地位,从国家战略层面进行思考,做出正确的决策?

第六,我国人口问题一直存在,人口总量减少是当前我国面临的一个重大问题,除了一二线大城市、部分省会城市、东部经济发达城市的人口是净流入,全国中西部大部分地区、中小城市特别是农村地区大都是人口净流出和净减少,我国的人口危机和大城市病正在逐渐显现。由于未能及时对计划生育政策进行调整或调整力度不够,我国人口问题日益严重,如果解决不好,可能会影响我国未来100年甚至千年的国运和兴衰。

第七,从1982年计划生育政策写入宪法、强制实施独生子女政策以来,人们从强烈要生到不愿意生的生育观念已经发生了很大变化。由于受西方家庭观、生育观、教育观、文化价值观入侵及影响,四十年计划生育对中国社会形态和人们生活观念已产生深刻影响,中国传统文化中的传宗接代、养儿防老、儿孙满堂等观念已经受到极大摧残,一是家族被摧毁,无论是城市还是乡村,家族、宗族已经开始退出人们的意识。二是家庭少子化、小型化甚至无子化成为趋势和习惯。三是异地工作、乡村衰败、留守儿童等状态成为常态并无法扭转。四是晚婚、晚育、不婚、不育的错误观念以及其它的西方丑恶观念越来越深地影响中国年轻人的意识和行为。

第八,现在的互联网上普遍存在一种生不起孩子、养不起孩子的舆论,为什么会出现这种舆论?我们仔细观察周围不愿生孩子的家庭和年轻人,真的是生不起孩子、养不起孩子吗?\xhx{再怎么穷,不说生两个三个,难道连一个孩子都生不起养不起吗?现在能比几十年前更穷吗?}仔细观察会发现,一些本人和父母的生活和工作条件都很好的年轻人也都开始不再生孩子,而这些年轻人的父母大都希望他们的孩子生孩子,他们有条件养育孩子,而他们的孩子却不愿意生,甚至不愿意结婚,这其中显然有另外的原因,在大家都说生不起孩子、养不起孩子的时候,真的是因为穷吗?为什么我们互联网上一直都在宣传生不起孩子、养不起孩子的舆论呢?我以为这背后恐怕有更复杂、更深层的原因,这一点一定要引起我们的高度警觉。

第九,最近网上都在讨论脱口秀演员杨某称男人为垃圾、制造男女对立话题的事件,还有变性舞蹈演员金某在山西演出时展示彩虹旗的事件。这是孤立的偶然发生的事件吗?其实从人口论进入我国到娘炮文化在我们成为社会灾难,从金钱至上、娱乐至死对我们社会的普遍毒害到躺平文化的盛行,我们应该能够意识到,以上这些从理论进入到一些有毒观念对我们社会的渗透毒害以及对传统公序良俗的颠覆性改变,都不是孤立的,而是有联系的,中国人口问题恶化到今天这个地步,绝不是某一个事件、某一个人、某一种思想所能做到的,而是一种成体系的东西在背后作祟。

第十,现在的问题一方面是生育意愿的丧失,另一方面是生育能力的降低,有一部分育龄妇女生育能力降低,想生而不能生。出现这一问题的背后或许有着更复杂甚至是更可怕的因素,我们必须去探寻真相。为什么会有越来越多的年轻人特别是育龄妇女生育能力降低?不解决这个问题,我们真的可能会面临亡国灭种的危险。这里我们必须高度重视某些转基因食品、某些疫苗、某种药物、某些食品添加剂以及某些不良生活习惯等方面存在的隐患和可怕后果。

第十一,如果不解决以上问题,而是从根子上抛弃了家庭观念、传种接代、社会结构的中国传统,如果我们的年轻人逐渐失去了生育意愿并且生育能力降低,那么我们给予再好的生育保障和社会服务都可能是无效的。因为从根子上没有了生育意愿和生育能力,相当于在我们年轻人的思想和身体上打上了某种钢印,不去除这些钢印,是无法解决问题的。因此我们必须意识到这些钢印的存在和这种钢印对年轻人的毒害,然后想办法去除这些钢印。

第十二,解决人口问题的先决条件是解决思想问题,我以为最大的思想问题是某些丑陋的西方价值观,这些丑陋的西方价值观已经越来越表现出被某种强大势力操控的状态和趋势,这种强大势力既有宗教力量也有资本力量,这种势力对人类怀有普遍仇恨、对中国带有极端偏见和仇恨,企图灭绝我们民族、人种和文化。当前我们最需要做的是弄清真相,要意识到问题的严重性,要做出正确的决策,要清醒地知道我们该做什么以及我们该如何做。

第十三,人口问题是一个国家的根本问题,关系到种族繁衍、文明存续和国家的生死存亡。如果我们不能解决人口问题,不化解人口危机,我们将会在当今世界的竞争中丧失竞争力。人口问题不能只看眼前和短期,不能只看自己一个国家,而应该放在历史的、文化的、民族的、国家的高度去认识,人口问题的时间线要从百年千年的纵深去看。我认为\xhx{大国竞争最重要的有三条,一是人口,二是土地,三是文化,有这三条就永远不会消亡,哪怕有时会陷入低谷,但只要有庞大的人口优势就仍有重新崛起的机会,如果失去了这三条中的任意一条,就会万劫不复。中华文明能够成为世界文明史上唯一没有中断的文明,中华民族能够始终保持强大的生存和繁衍能力,能够一次又一次重新崛起并屹立于世界文明的中心,就是因为我们始终有众多的人口、广袤的土地和厚重的文化。因此一旦人口出现危机,则预示着我们民族也将面临重大危机。}

第十四,人口负增长将给我们带来哪些严重冲击和伤害?总和生育率必须在2.1以上才能维持人口和家庭的更替平衡,当总和生育率低至1或低于1时,不仅无法维持国家人口现状,更大的冲击和伤害是家庭解体,家族消失,中国人家国情怀的传统文化将遭到毁灭性打击。与此同时整个国家进入老龄化社会,从全国平均五个人养一个老人到平均一个人养一个老人,整个国家严重缺乏青壮年劳动力和科技创新动能,严重缺乏创造力和竞争力,整个社会毫无朝气,暮气沉沉,当发生战争时,严重缺乏青壮年军人。如此整个国家和民族将无法赢得残酷的大国竞争,将整体性失去生气与活力,将陷入全面衰落,最终不战而败。

第十五,现在在互联网上,总有一些阴阳怪气的声音,认为中国人口多是负担,中国大量的人口是低素质的垃圾人口,人口必须少而精,只有少生才能做到高素质。现在我们在一些国际资本大鳄的言论中会发现,他们把中华民族和中国人视为垃圾人口,要把中国人从世界上清除掉。我认为当今中国人口出现断崖式下跌就与某些国际邪恶势力数十年的渗透和毒害有关。在世界历史上,中国不仅一直是一个文明国家,而且一直都是一个人口大国,中华民族始终是世界上最优秀的种族,中国人勤劳、聪明、坚韧、勇敢,中华文明是人类历史上最伟大的文明。而那些海盗民族一直都视中国人、视中华民族为他们最大的敌人,极欲灭之而后快。由于他们无法打败中国,所以他们对中国发动思想战、价值观战、文化战、生物战、基因战,从思想和价值观上进行腐蚀和渗透,以达到灭亡中华民族的目的。经过数十年从人口理论到价值观念的全面入侵,对我们的思想价值体系、家庭文化观念、家庭生活方式产生严重冲击和侵害,而这正是当下我们一些人在生育观方面出现严重问题的背后因素,对此我们必须警醒并进行反思。

第十六,人口竞争是一场战争,这绝不是危言耸听。近一年来,我们看到以色列对加沙的巴勒斯坦人进行灭绝式大屠杀,这种屠杀是公开的、毫无顾忌的。几百年前,当欧洲殖民者到达美洲大陆的时候,他们所进行的就是对当地数千万土著印第安人实施最野蛮最残暴的屠杀,现在在北美大陆上仅存几十万印第安人被圈养在一些指定的贫瘠地区,清除了他们眼中的所谓垃圾人口之后,他们就可以心安理得地在占领土地上享受生活。当中国人口出生率出现断崖式下跌、已经出现严重负增长的时候,如果我们还不能从战争的高度对待,还不能从国家和民族生死存亡的高度对待,我们很可能会成为中华民族历史上的罪人。奥巴马说过,如果中国人都过上美国人那样的生活,对美国来说,那将是一场灾难。其实众多的中国人口从来都不是负资产,而是我们抗御侵略、建设国家、存续文明的根本来源和基本力量。毛主席曾说过,人民,只有人民,才是创造世界历史的动力。中华民族要想始终屹立于人类文明的巅峰,就必须始终保持庞大的人口规模,保持广大人民的英雄气和创造力。

第十七,要想扭转当前新生人口断崖式下跌的状况,必须立即进行两个方面的工作,一是弄清现在部分育龄妇女生育能力降低的真相,让所有育龄妇女都能生育。二是坚决清理我们在人口理论和行动中曾经奉为圭臬的思想,坚决清理和打击现在网络上传播的那一套关于LEBT、关于男女对立、关于躺平、关于丁克、关于晚婚晚育、关于不婚不育的思想,大力宣传中国传统生育观、家庭观、文化价值观,要在我们主流媒体、平台上全面抨击不婚不育的有害有毒思想以及这种思想将给他们个人、家庭及社会带来的灾难性影响,旗帜鲜明地进行正常年龄结婚生育幸福、多子女家庭幸福的宣传。形成全社会支持生育、支持多生孩子的良好氛围。

第十八,由于我国国情已经发生重大变化,出生人口形势面临重大危机,应该立即修订《中华人民共和国人口与计划生育法》,取消该法中确定的“实行计划生育是国家的基本国策”条文,将《人口与计划生育法》修订为《人口与促进生育法》,将促进生育作为人口立法的主题,实行全面支持、鼓励、帮助生育养育的政策。我认为这一点是战略性、纲领性、指导性的,如果我们不能从国家法律层面、从根本思想层面、从国家政策层面适应形势与国情变化,及时调整我们的国策和政策,必会给我们国家和民族的未来带来严重的负面影响。

第十九,最近国办出台的《关于加快完善生育支持政策体系推动建设生育友好型社会的若干措施》非常及时,现在一定要从经济、服务、时间和文化方面进行全方位支持生育,扫除促进生育工作中的各种障碍,真正解决年轻人在生育、养育孩子中存在的各种困难,减轻生养孩子的压力,在全社会形成一种大家愿意生育孩子的氛围。

第二十,自古以来,我国就是一个人口大国,人口众多是中华民族生生不息、中华文明永续传承、在世界上保持大国地位的重要基础,是我国实现全产业链、保持强大制造业、保持庞大市场规模的重要基础,一旦失去人口大国地位,一旦失去庞大的人力资源优势,一旦失去庞大的青壮年劳动力优势,我国的大国地位将不保,我国的竞争优势将不保,特别是在进入重度老龄化社会后,我国将未富先老、失去竞争力和创新活力、陷入严重衰退,我国的国运将出现重大危机。

人口问题是民之大事,族之大事,国之大事。我们一定要从战略的高度,从国家和民族生死存亡的高度,从文明存续、家庭传承的高度来认识,对我们的国策进行战略性调整,全力阻止人口负增长趋势,化解人口危机,全力阻止新生人口断崖式下跌,为民族繁衍计、为文明永续计,为人民幸福计,坚决打赢这场人口战争。


\end{document}

