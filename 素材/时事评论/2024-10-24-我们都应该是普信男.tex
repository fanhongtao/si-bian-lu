\documentclass[UTF8,11pt,oneside]{ctexart}

\def\articletitle{我们都应该是普信男}
\usepackage{CJKfntef}
\usepackage{float}

\usepackage{geometry}
\geometry{a4paper,left=2cm,right=2cm,top=2cm,bottom=1cm}

\usepackage{graphicx}

\usepackage{hyperref}
\hypersetup{colorlinks=true, linkcolor=red}

\linespread{1.6}

\usepackage{fancyhdr}
\usepackage{ifthen}
\pagestyle{fancy}
\fancyhf{}
\setlength{\headheight}{14pt}
\fancyhead[R]{\ifthenelse{\value{page}>1}{\thepage}{}}
\fancyhead[C]{\ifthenelse{\value{page}>1}{\articletitle}{}}
\renewcommand\headrulewidth{0pt}

\usepackage{tcolorbox}
\tcbuselibrary{skins}

\newcommand{\zd}[1]{\textbf{\textcolor[RGB]{123,12,0}{#1}}} % 重点

\newcommand{\yinyong}[1]{% 引用
    \begin{tcolorbox}[enhanced,
        frame hidden, interior hidden,
        before skip = 5mm, left skip=10mm,
        borderline west={5pt}{0pt}{gray!50}]
        #1
    \end{tcolorbox}
}

\newcommand{\xhx}[1]{%下划线(模拟微信中的划线功能,用于标注我个人认为的文章中精彩的地方)
    \CJKunderline*[thickness=1.5pt, format=\color[RGB]{84,216,140}]{#1}
}

\newcommand{\biaoti}[1]{% 标题
    \section*{#1}
}

\newcommand{\SetSectionType} {
    \ctexset{
        section={
            number = \chinese{section},
            aftername={、},
            format=\Large\bfseries,
        }
    }
}



\begin{document}

\begin{center}
    \LARGE{\articletitle\footnotemark}
\end{center}
\footnotetext{
    原文出自公众号“立刚科技观察”的文章 《\href{https://mp.weixin.qq.com/s/ItLJTXSvb2-m5fm9nZgdVQ}{\articletitle}》
}

我自己主要是关心信息产业发展,对于社会上很多事情并不特别敏感,最近看到炒男女对立,有一个词据说是女人发明的,叫普信男,我觉得非常有意思。

对于男女对立这个事情,我并不特别担心。虽然说网络上有很多炒作,但是我看我自己的周围,看我的生活中,狗屁的男女对立,女孩都想找男朋友,男人自然想找老婆,我们年轻的同事,结了婚很多都生了两个孩子。\xhx{那些对立者以后是要断子绝孙的,这个世界总归是正常人的。}

至于说到普信男,我想来想去这是个好词啊,普通而自信的男人。我自己首先就是一个普信男。

中国14亿人中,绝大部分都是普通人了,生活在小地方,出生在普通的家庭,过着一般的生活,这样的男人做老公是很正常的。天天想着找王思聪这样的,第一很难找到,毕竟这种富二代不多,其次人家也不会好好待你,你的感情和生活还不是支离破碎,有什么好的?

我们这种普通的男人,不仅是社会的大部分,大部分的普通男人,才会真心的待自己爱人。我这种身高矮,学历低的小地方社会下层,知道自己找个老婆不容易,才会好好的真心待自己的老婆。普通的男人怎么了,普通的男人是世界的大多数。

至于说到自信,一个男人连自信都没有,这不是没骨头吗,普通而自信的男人,才会渐渐的找到机会,才会有所发展。看生活中,那些已经取得了成功,做出大事情的人,自信是嵌在骨子里的,一个人不相信自己,不相信自己的民族,哪可能有美好的人生,怎么可能有奋斗的动力。

天生我才必有用。有这样的自信,有强大的内驱力,这样的人才会有所作为。

我记得我年轻的时候,读了个中专,在企业做会计。我觉得我这个水平,比一般的大学生那是强了很多了,就要去考个研究生,同事们都去劝我,你真要考就考一个本地大学,我说那不行,这多没意思,我这个专业第一是北师大,第二是北京大学,我觉得中国人民大学也行,就考了个人民大学。没点自信,我还在安徽做会计。

后来工作做图书编辑,我看信息产业有很大的发展机会,上学我自己肯定没学过,我就自己买台电脑,开始学习和研究,玩了一两年,办了本杂志叫《学电脑》。再看通信有机会又办了本杂志叫《通信世界》,同事都认为我是学文学的,不可能把这本杂志办起来,其实不出三年,我们杂志就是行业第一。

没点自信,这些事情都不可能去做,也就不会有小小的人生成就。有了自信,敢于一试,这才可能有机会。

至于说追姑娘,那我肯定是绝顶高手,技术水平是一流的,只是找到了老婆,没有机会发挥了,面对一个女孩连自信都没有,那你能追到谁?

我和世界上很多优秀男人都会有交流,我看男人一个最基本的特质,自信是最重要的基础,小地方的社会下层,连自信都没有,怎么会有奋斗,怎么会有机会?

普通而自信的男人,我想这是一个优质的男人。这样的男人才会努力进取,才会有远大的理想,一个男人悲悲切切,一个男人唯唯诺诺,一个男人整天想自己不行,那他就是不行啊。

\xhx{生在小地方,出生于普通家庭,这并不是你的错,这样的男人唯一支撑的内驱力,就是相信自己,奋斗不止。}

女孩们去找一个普信男,我相信绝大部分人会是幸福的一生。

对那些鼓吹男女对立的,我还是这个态度,并不觉得有多大问题,反正他们会断子绝孙的,这个世界最后还不是属于正常人的?

\end{document}

