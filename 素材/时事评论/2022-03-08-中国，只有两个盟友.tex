\documentclass[UTF8, 11pt, oneside]{ctexart}

\usepackage{color}

\usepackage{geometry}
\geometry{a4paper,left=2cm,right=2cm,top=2cm,bottom=1cm}

\usepackage{hyperref}
\hypersetup{colorlinks=true, linkcolor=red}

\linespread{1.6}

\ctexset{
    section/name = {Part-},
}

\newcommand{\zd}[1]{\textbf{\textcolor[RGB]{123,12,0}{#1}}} % 重点

\begin{document}

\begin{center}
    \LARGE{中国,只有两个盟友}
\end{center}

这是一篇两年前写的文章,今天读也毫不过时,当时没有发在今日头条上,原文未修改如下:
\footnote{
    我看到的原文是“微信”公众号 “一个坏土豆 ( iamhtd )”在 “今日头条”上发布的文章
    “\href{https://www.toutiao.com/i7072688443138015744}{中国,只有两个盟友}” }
\footnote{原文有图片,我没有收录。}

10年前,中国电影出了一部神作「\zd{让子弹飞}」,字字珠玑,句句经典

「汤师爷,你给翻译翻译,什么叫惊喜?」

「其实你和钱对于我都不重要,重要的是\zd{没有你}对我很重要」

「我就是想站着,还把钱挣了」

「他要是体面,你就让他体面,他要是不体面,你就帮他体面」

「这算是什么狗屁世道?就因为我是好人,才被人拿枪指着?」

.......

但是,最经典的,莫过于一句:

\zd{老百姓,永远是跟着胜利者走的.......}

\section{}

美国的盟友是谁?

如果让世界上200多个国家和地区,搞一场国家层面的国意调查「帝哥要毁灭一个国家,你选择谁」

不用想,美国必然是得票率最高的国家

最恨美国的,尤其是美国的亲密盟友

其实,美国真的有盟友吗?

所谓盟友,最基本的,是双方平等的关系

美国可以在日韩驻军,为啥日韩不能在美国驻军

美国可以在欧洲遍地军事基地,为啥欧洲不能在美国修军事基地

大家心知肚明,美国叫你一声「\zd{盟友}」,给你面子而已

日本不是傻子,自己的领空都不能自主,被勒索军费,领导人被羞辱,奈何看看美国在日本设置的纪律监察委员会、被日本控制的媒体、在国土上的驻军,只能继续谄媚的笑「日本东京地检特搜部是美国控制日本的一个类反腐部门」

欧洲更不傻,为啥每次想发展的时候,欧元想要飞跃的时候,美国就会正义感爆棚,要为自由和民主而战,要检查欧洲附近的大规模杀伤性武器,要制裁欧洲附近的独裁政权,然后连蒙带吓的让大家参战,奈何法国看看德国,德国看看法国,大家再看看欧洲的美军基地,然后都不吭声了

疫情爆发,美国现在最忙的就是印钞厂,天天开足马力24小时印钱,摆明了就是抢全世界的钱,现在是美国创造财富的巅峰时刻,星期一印出了德国的全年GDP,星期二印出了日本全年的GDP

英国想:FUCK,美国在抢你钱呢,中国你怎么还不反抗

日本想:八格,美国让你在承受膨胀呢,中国你怎么不说一声

印度想:卧C,美国怎么印钱印个没完了,中国你怎么还不吭声

中国:MMP,我说了你们继续帮美国来黑我吗

大家都想让美国去死,但是没人会动手,都看着中国,因为现在美国要搞的是中国,反正\zd{天塌下来个子高的顶上去},看看中国怎么应对,你不动,我也不动,\zd{你动了,我还是不动}

局势未明之前,没有哪个国家愿意将赌注压在中国身上,毕竟,世界上挑战霸权的国家屈指可数,而且多数都失败了,我现在站队,中国万一输了,怎么办。

疫情中中国援助了那么多国家,当他们真的心里没数吗,连个谢字都不知道说?大家可看看美国的脸色,又把话咽了下去....

\zd{中国的挑战者这一角色,是美国设定的},美国不会给中国其它的路走。在挑战成功前,绝无国家敢主动和中国靠近,除非是被美国逼上去。

一旦中国挑战成功,当天晚上,所有的国家都会向中国笑脸相迎,中国一下子多了一百多个盟友。

但也没有哪个国家敢和中国结仇,帮美国指责几句可以,撕破脸是绝对没这个胆子的,否则一旦中国挑战成功,中国的一百多个盟友会马上让它万劫不复。

挑战之路上,中国必定是孤独的。

因为对抗霸权这个事上,结盟,从来都不靠谱。



\section{}

\zd{挑战霸权,所有的结盟本身就不靠谱}

历史上成功的战争千千万,但以弱胜强的最基本一条必是团结

历史上失败的战争万万千,但以强败于弱多数都是因为心不齐

别说美国现在只是抢点钱,勒索点军费,就算是灭国之战,结盟也估计搞不起来,千百年的历史已经告诉过我们了。

张仪的纵横之术,天生就要比苏秦的合纵之术有优势

张仪只需要搞定一个人,苏秦需要搞定六个人

张仪只需要煽风点火,苏秦是需要与人性做斗争,猜忌、犹豫、懦弱、动摇这些都是人天生的弱点,苏秦想让人转向真的是太难了

所以往往是苏秦累死达成一项协议,张仪随便来两个谣言就拆散了。

秦国要灭六国,是公开的策略,结果六国你看我,我看你,先后在前318年至前241年,5次联军达到了秦国家门口,无一不是矛盾重重,或无功而返,或率先内讧,或打到一半撤军。

每次看张仪和苏秦的斗法,感觉苏秦真的是困难重重,引经据典,晓之以情,动之以理,作无数的思想工作,想把六国团结在一起。

张仪就像哄小孩一样,两下就让苏秦白干了,把所有的盟国拆的七零八落。

当时楚国与齐国结盟,是山东六国中最为强大的两个国家,两国结盟是什么目的,谁都看得出来,不久张仪到访楚国,对楚王说:「如果大王能够和齐国断交,我们马上送你们六百里地」

楚怀王一听大喜过望,什么秦国的威胁啥的早忘到一边了,决定派遣使者与齐国断交,大臣陈轸苦苦劝到:「大王你要这样干了,我们就和齐国结仇了,到时候齐国如果和秦国联起手揍我们,咋办?」

楚怀王说道:「六百里地啊,我就不要了吗?张仪是秦国的CEO,能言而无信吗?」

陈轸又说道:「你非要这样,那么还有个办法,我们一边与齐国断交,同时密函把事情原委告诉齐王,这样一来我们既得到了商於六百里土地,又不损失什么,齐国也能谅解我们的苦衷」

按理说这样的做法可谓一举两得,什么都不损失,平白得地六百里,就算没有得到也没有什么损失!可是楚怀王猪油蒙了心,一口回绝了陈轸的提议,又把楚国相印给了张仪并派人护送张仪秦国。

可是一到秦国张仪就得了病,在家一躺三个月,闭门不见客,秦王自然也不会认这个帐,楚怀王觉得可能是秦国觉得自己诚意还不够,又接着干蠢事,于是派遣宋遗持宋国节仗到了齐国的边境大骂齐王,气的齐愍王一怒之下决定与秦国结盟,对付楚国。

直到此事发生之后,张仪才宣布养好了伤上朝,楚国使者讨要商於之地却被张仪一句话噎了回去,「什么六百里地,我是在商於有一块名为\zd{六百}的六里土地」。

楚王肠子都悔青了,盛怒之下,派遣十万大军讨伐秦国誓要活捉张仪。结果两军在丹阳遭遇,楚军大败,被斩首八万人,随即韩国、魏国也趁机发兵偷袭楚国,楚国仓促退兵......

两千年前,就已经写下了「孙子兵法」的中国,比谁都明白这些道理,一早就想得很清楚「什么结盟啥的都不靠谱,一切都要靠自己」

中国为什么推进不结盟政策,这个时候让别人站队,没有任何意义



\section{}

\zd{全球中美之外,分为六类国家}

\zd{一类是如日本、韩国、德国等被美国胁迫的国家},德国还好点,日本是深受美国欺凌,敢怒不敢言,但是也就这德行了,被美国扔了两颗核弹,亲热的叫爸爸,中国从古至今,对日本照顾有加,日本有机会就想放中国黑枪。

日本现在被美国重重控制,有没有机会成为中国的盟友呢?有,但是要等到中国对美国有压倒性优势的时候,日本一定反复重申,中日友谊地久天长。然后领导人到南京大屠杀纪念馆,真诚道歉,磕头把地板都磕出坑来。

可惜,哪个时候,中国还需要日本吗?

所以日韩德这样的国家,想让他们坚定的支持中国,几乎没有这个可能性,尤其日本,不躲后面放黑枪就不错了。

德国无数次的想和中国走进,2019年默克尔启动最大规模代表团访华,可惜,心有余而力不足

\zd{二类是如丹麦、瑞典、荷兰这样的小国}

这类国家,更加无所谓,反正美国抢劫,又不是只抢我一个人的,抢我一块钱,抢你都要抢10块钱了

抢点钱就抢点钱,反正又不要我的命,交个税而已

美国是要打老二,我何德何能,就这点体量,反正也做不了老二,美国也打不到我头上来,看看热闹得了。

其实还有些国家,如波兰、冰岛这样的,主动要求美国驻军保障自己的安全,就给点钱而已嘛,多大的事。

花钱买平安,很多人是愿意的。

这些国家处于游离状态,某种程度上还希望美国维持霸权,因为多数人都是喜欢过安逸日子的,变来变去的太麻烦了

\zd{三类是英国、澳大利亚、加拿大这样的国家}

这些国家,根源上和美国是亲戚,相同的文化走得更近,虽然说国际关系中只有利益没有朋友,但好歹一个语种,一个族系,总能减少些摩擦。

对他们来说,深刻认同美国的战略,某些方面,希望美国能干倒中国,自己还能上去分一杯羹。

但是,又不敢和中国撕破脸,万一中国真崛起了,自己不是很麻烦。

这些国家给美国摇旗呐喊,要成为美国的帮凶,只在一念之间。或者直白点说盎格鲁撒克逊人这一系的,就不会对中国安半点好心。

英联邦成员印度当然也要算上,何况一直对边境问题念念不忘

\zd{四类是法国这样的国家,有自由和浪漫的传统}

发生历史上最多流血事件的革命,法国确有自由之名,算是反对美国的西方大国了。也是最早和中国建立正式外交的西方大国。

法国早在二战后初期,就开始无法忍受美国对欧洲和对法国的控制,尤其是当时法国的总统戴高乐。一心要恢复法国的荣光,不希望法国成为美国的附属,所以从50年代就开始反对美国。当时在60年代的时候,就挤兑美金,让美国差点下不来台。

从50年代开始,不光直接驱赶美国在法国的驻军,清理掉美国在法国的军事基地。而且还直接退出美国主导的北约,自己单干。「后2009年重返北约」

现在面对美国的挑拨离间,第一个喊出来:为什么我们不能和俄罗斯谈谈,为什么要一直被美国骗。

法国是可以争取的,也是西方国家中唯一敢和美国唱对台戏,还保留自由精神的国家,但是除非中美实力再接近,法国只会骑墙

\zd{五类国家是如非洲这样的国家,可以吃瓜}

\zd{六类国家是 美国给中国安排的朋友}

中国一直倡导不结盟运动,是邓公的高明之处,但是不少国家被美国逼上梁山,结果和中国不自觉的走近了


\section{}

\zd{中巴关系}

中国和巴基斯坦,可以形成良好的互补

当然,千万不要相信一些反智的谣言,说中国人到了巴基斯坦大受欢迎,纯粹扯淡。你想疫情结束了,还好多人防着湖北人呢,还想着有个国家的人对中国人比中国人对中国人还好?

国家之间,从来没有什么因信仰而结盟,都是利益达成的共同体。所谓意识形态之争,都是骗鬼的,中国如果安心做裁缝,别说社会主义国家,是原始社会美国都愿意有友好关系。

现在的「巴铁」在我们建国初期并不「铁」,而且长期是亲美反华的。

印巴的外交有个前提就是两家从各种宗教矛盾、种族矛盾、历史矛盾根本解不开,在印巴分家的当天就迫不及待抄家伙,双方互相攻击,搞了个50万人的种族仇杀。如果说我们和印度有摩擦,巴基斯坦和印度那就是血海深仇,这个结,基本解不开

在分家上,印度比较占便宜,首先,继承了英国的「\zd{印度}」这一历史名称和政治遗产,并成为英联邦成员。「注:印度在历史上没有统一过,是英国创造出来的一个国家。」

而印度在初期,选择的外交策略是比较聪明的,采用的是不结盟策略,在美苏争霸的时代,不表明态度,给双方都留有幻想的空间,让美苏都想拉拢,同时,作为英联邦成员,又能得到西方世界的一些照顾。

而巴基斯坦就比较惨了,首先,在地缘外交上,无论是继承的「印度」的遗产,还是人口,都没有印度那么重要,如果让各个国家谈重视程度,一定是排在印度之后。

而巴基斯坦在穷兵黩武的印度的强大压力下,为了自身安全,在1958年加入了美方阵营,刚加入,就急不可待的进行表态,对边界和台湾问题上对中国进行指责,并且将军事基地借给美国,将苏联彻底得罪了。

但是,问题是,在西方世界的对世界的博弈中,巴基斯坦这颗棋子,无论是在国力、人口、经济上都比印度落后,注定是没有如印度那么受重视。

而且,印度的态度越是暧昧,美苏越是想争取,这个时候对巴基斯坦尴尬的是,在经济上得到了美国的支持,但是在军事和安全上没人搭理,如克什米尔问题,没人在意它的诉求。

而印度,美国希望来打造民主的样板,苏联希望争取来制衡美国,反而左右逢源。

1962年,中印冲突期间,美国加大对印度的军事援助,同时严令巴基斯坦保障印度的安全,对盟友不闻不问,对非结盟的印度百般袒护,彻底激怒了巴基斯坦,而中国对印度的打击让巴基斯坦乐开了花,终于开始慢慢的倒向中国,在联合国上支持恢复中国席位,给出了橄榄枝。

1965年8月,克什米尔游击队和印度军队出现火并,爆发印巴战争。1965年11月,印军突入巴基斯坦,在初期势如破竹,巴基斯坦已非常危险。

关键时刻,中国救了巴基斯坦的命!

9月15日,中国在中印边界开始大幅度兵力调动,并严令印度撤出巴基斯坦,9月18日,挺进中印国境线。面对双线可能爆发的作战,印度承受巨大的压力,从巴基斯坦撤军。此后,中国的军援、军工开始对巴基斯坦全面支持。

中巴关系,是比较完美的利益共同体

印度一直拿中国作为假想敌,同时,印巴又是死对手,双方借助彼此的力量制衡印度

巴基斯坦意味着中国到非洲、中东航程上可以比马六甲海峡缩短一半以上

还有最重要的一点,伊朗是中东对抗美国的钉子户,面对美国的共同压力,伊朗和俄罗斯走到了一起,组成联盟,共同对抗美国。

美国要控股中东的石油,是美元霸权重要的组成部分,美国希望埋进去的以色列这个钉子,加上与沙特的关系,统一中东的石油产地,进而,通过石油控制全球经济

最终,中俄伊走到了一起......

然后,巴基斯坦是连接中国到伊朗的重要管道,即巴基斯坦通道,是中国通向中东的必经之路.....

中巴的关系,美国心理和明镜一样,清楚的很,想尽一切办法来找茬,2011年在巴基斯坦击毙本拉登后,放出话来要废掉巴基斯坦的核武器库

是的,印度和巴基斯坦,都是拥核国家,当然能打到哪里,我不知道

美国话音刚落,中国马上给巴基斯坦送达50架枭龙战斗机

巴基斯坦,对中国来说,太重要了



\section{}

\zd{逼上梁山的俄罗斯}

和巴基斯坦不一样的是,俄罗斯是妥妥的被逼上梁山......

俄罗斯一直想回归西方,但是美国太需要俄罗斯了,需要俄罗斯作为一个敌人.........

戈尔巴乔夫时代,就对美国的民主自由深信不疑,叶利钦直接对美国投降,结果美国对俄罗斯没有缴枪不杀一说,摆明的态度就是要赶尽杀绝,要将已经分解的俄罗斯再次分解,直到普京上来才挽回颓势

可见「普京的复仇」

对于美国来说,事实上不会和任何国家真的有仇,只是自己作为全球的奴隶主,绝对不允许有势力来挑战自己的权威,同时,基于自己的国家属性,全球的区域战争与矛盾是必须要存在的,如果世界大同,全球都努力发展经济,美国就完蛋了,军火商怎么办,美元霸权怎么办.......

俄罗斯如果真的如法国想的那样,和西方世界和解,这意味着北约再无存在的必要,自己还有什么理由在欧洲驻军?

一旦失去北约,欧洲凭什么听美国的?德国法国还甘心被美国控制吗?你年年坑人家,真当人家傻,若有一个真的团结的欧洲,拼什么保障欧洲不会挑战美国霸权,不会和中国结盟?

所以俄罗斯也是一条不归路

在前15强的国家中,中俄必须是美国的敌人,不允许投降,美国已经定好了游戏规则,否则还能怎么制造全球不安定局面?

日本都躺地上了,你爱咋咋

印度天天献殷勤

还能到墨西哥剿匪禁毒去?



\section{}

\zd{中国,只有两个盟友}

但是,历经战火洗礼的中国,非常清楚,放眼全球,自己只能孤独的对抗霸权,所有的盟友都可以被收买,所有的利益都可以被交换,中国,从未指望过任何人

我们干了很多别人看着很蠢的事情,中国的粮食储备全球最高的,水稻存储占了全球的2/3还多,占国土面积1/10为数不多的土地,永远是先保障三大主粮的安全,而非为了经济利益去种作物

粮食储备多到一年不种田还可以支撑一年半

在工业上,我们建造了全球最系统,最完备的工业体系,美国硬生生的把华为逼成了全能的民族企业,先做基站通讯服务,再做手机,再做麒麟芯片,这一切,仿佛是在逆全球的分工体系,为什么要踩别人已经踩过的坑呢,这意味着我们的产品要付出更高的成本

我们希望开放和真诚的面向世界,但西方世界已经给了我们答案。

昨天的苏联,一直在告诉我们,与美国的战斗,不能输,且没有投降这么一说

输了,就是国家被分裂,经济被颠覆,人民流离失所..........

美国过来抢尸体,无数的美国盟友将纷沓而至,过来分一杯羹.......

\zd{中国,只有两个盟友,工业和农业}

我们笨拙的建设了最大的粮仓和最完备的工业体系,是在告诉西方:

我不会给一点点的机会,那怕万分之一,一百年前你们给予的,我们从未忘记......

我们相信人类命运共同体,相信病毒是全人类共同的敌人,希望在灾难面前守望相助,我希望与全世界和平发展

\zd{但我亦不惧怕全世界与我为敌!}

\end{document}

