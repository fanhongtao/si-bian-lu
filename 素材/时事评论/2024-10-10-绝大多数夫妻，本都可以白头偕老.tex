\documentclass[UTF8,11pt,oneside]{ctexart}

\def\articletitle{绝大多数夫妻,本都可以白头偕老}
\usepackage{CJKfntef}
\usepackage{float}

\usepackage{geometry}
\geometry{a4paper,left=2cm,right=2cm,top=2cm,bottom=1cm}

\usepackage{graphicx}

\usepackage{hyperref}
\hypersetup{colorlinks=true, linkcolor=red}

\linespread{1.6}

\usepackage{fancyhdr}
\usepackage{ifthen}
\pagestyle{fancy}
\fancyhf{}
\setlength{\headheight}{14pt}
\fancyhead[R]{\ifthenelse{\value{page}>1}{\thepage}{}}
\fancyhead[C]{\ifthenelse{\value{page}>1}{\articletitle}{}}
\renewcommand\headrulewidth{0pt}

\usepackage{tcolorbox}
\tcbuselibrary{skins}

\newcommand{\zd}[1]{\textbf{\textcolor[RGB]{123,12,0}{#1}}} % 重点

\newcommand{\yinyong}[1]{% 引用
    \begin{tcolorbox}[enhanced,
        frame hidden, interior hidden,
        before skip = 5mm, left skip=10mm,
        borderline west={5pt}{0pt}{gray!50}]
        #1
    \end{tcolorbox}
}

\newcommand{\dianping}[1]{% 点评
    \begin{tcolorbox}[enhanced,
        colframe=red!50!black,
        title=点评]
        #1
    \end{tcolorbox}
}

\newcommand{\xhx}[1]{%下划线(模拟微信中的划线功能,用于标注我个人认为的文章中精彩的地方)
    \CJKunderline*[thickness=1.5pt, format=\color[RGB]{84,216,140}]{#1}
}

\newcommand{\biaoti}[1]{% 标题
    \section*{#1}
}

\newcommand{\SetSectionType} {
    \ctexset{
        section={
            number = \chinese{section},
            aftername={、},
            format=\Large\bfseries,
        }
    }
}



\begin{document}

\begin{center}
    \LARGE{\articletitle\footnotemark}
\end{center}
\footnotetext{
    原文出自公众号“远方青木”的文章 《\href{https://mp.weixin.qq.com/s/1uhCzfEuW3IWzo0BKd3Hnw}{\articletitle}》
}

每一对夫妻,在宣誓结婚的那一刻,我相信他们都是真心希望自己能和对方白头偕老的。

但有很多人,并没有走到最后。

\zd{什么样的婚姻才能走到最后,什么样的婚姻不能?}

今天,我就和大家谈一谈\zd{婚姻的本质}。

你想知道怎么才能白头偕老,\zd{那就首先要搞明白你为什么要结婚。}

单身不香么?单身你想干嘛就干嘛,想结识多少异性就结识多少异性,不违法也不违道德。

但结了婚,你受到的限制就太大了,方方面面都受到限制。

同样的事放王思聪身上叫风流倜傥,放已婚人士身上叫十恶不赦,区别就是你领没领那张结婚证。

既然结婚后会损失这么多自由,那为什么人们还要结婚?

\zd{因为结婚会让你更幸福,所以你自愿放弃了一部分自由。}

如果你结婚只是为了性,或者“找个人陪陪自己”,那你离婚是早晚的事。

因为人性就是喜新厌旧的,无论男女都是,而一时的寂寞更不构成结婚的理由。

陪完了,就该撵走了。

能白头偕老的夫妻,在人生路上必然是互相扶持,相濡以沫。

说的有点虚,谈几个好理解的例子吧。

\zd{先谈谈什么样的穷人可以白头偕老,再谈谈什么样的富人可以白头偕老。}

离婚率最低的穷人夫妻是什么样的?

在现代,离婚率最低的夫妻,是小区门口的夫妻店。

夫妻小俩口开个包子铺,\textbf{男的蒸包子女的卖包子,男的和面女的拎水,男的去城郊批发市场大包小包进货,女的在店里看家。}

夫妻俩一起经营这个小店铺,谁都离不开谁,别说离婚,哪怕有一人生病出了意外,整个小家庭的生活水平都会断崖式崩塌。

这怎么出轨?

夫妻俩人只要智商正常,都会选择相濡以沫,白头偕老。

穷是穷了点,但真的是\zd{情比金坚。}

古代的穷人是怎么白头偕老的。

男的出门种地,重体力劳动,每天累死累活。

但那些收入仅仅只够家庭糊口而已。

要想生活的好一点,女的要在家织布,补贴一点家用。

虽然看起来织布没有种地累,但女的还要带家里的N个娃娃,还要负责照顾,还要负责做饭,一样是累的要死要活。

这就是古代穷人典型的\zd{男耕女织}的家庭模式。

如果男的出了意外不在了,整个家庭会瞬间陷入赤贫,孤儿寡母的日子极其难过,能讨几口饭吃把孩子养大已实属不易。

但如果女的出了意外不在了,整个家庭也会瞬间崩塌,家里的几个孩子谁带?家里的饭谁做?如果把家里照顾好了,那外面的地谁种?

孤儿寡母很难生活,一把屎一把尿把孩子带大的鳏夫,也很痛苦。

\zd{合则两利,分则两害,所以夫妻二人必须情比金坚,必须白头偕老。}

下面谈谈富人白头偕老的理由是什么。

现代的中国没有真正的富人阶层,出现了极大的断代,新生的富人阶层尚未形成自己的婚恋观,所以我就以古代富人举例子。

古代正妻的地位为什么那么高,为什么小妾哪怕貌美如花也不可能挑战正妻的地位?

首先,古代富家公子下聘,确实要花费千金才能娶来富家小姐,但富家小姐出阁时,一样会带来十里红妆。

这笔巨大的财富,是正妻地位的初步奠基石,也是两家结盟的标志,证明正妻背后的势力,关键时刻是可以做夫家后盾的。

然后,就是男主外女主内。

女主内指的是带孩子做饭么?

不,古代富人正妻不需要带孩子,也不需要做饭,那些都是佣人做。

但古代富人之家,动辄百余口人,其中八九十都是下人。

一位老爷,一位正妻,三四个小妾,七八个公子小姐,可能还有一两位健在的高堂,这些都要下人。

红楼梦里贾母一个人,就得八位侍女伺候,整个贾府得多少人?

这么多下人,都是需要正妻管理的,包括几个小妾也得收拾敲打。

你管过百人级别的公司么?

\zd{古代所谓的女主内,指的就是这种程度的管理,要平衡好府内各色人马的关系,要让下人的运转井井有条,这比当一家百人公司的CEO难多了。}

平时,正妻还得去结交其他大府上的正妻,走夫人路线,当家里老爷生意或官场上的助力。

很多权贵之女,从小就跟着嫡母耳濡目染这些,驾轻就熟。

但很多小妾,出身贫寒,除了美貌什么都没有,更不可能懂这些复杂的知识。

拿什么和正妻斗啊,美色这东西五年八年的也就没了,家里的老爷只要脑袋没毛病都知道怎么选。

古人云,娶妻娶贤,纳妾纳色,指的就是这个意思。

所谓的贤,指的就是贤内助。

真正出身名门的正妻,对丈夫的帮助实在是太大了,在古代重男轻女的社会背景下,权贵男性占尽优势,但这样级别的正妻,那也不是想娶就能娶得到的,极其稀缺。

所以虽然古代的宗族礼法对女性的限制很大,但权贵们的女儿,依然稳稳的掌控着一家之母的地位。

说到这里大家应该都品出味道了。

\xhx{无论古今,无论男女,支撑双方相濡以沫五六十年走到白头偕老这一步的,绝对不是美貌和性,而是互相扶持,互相需要。}

\zd{夫妻二人,必须要有一个共同且明确的人性目标,这个目标他们靠自己的力量都无法实现,迫切的需要彼此,才有可能走到最后。}

否则,等双方对彼此肉体厌倦的时候,稍微吵几架,就离婚了。

什么样的婚姻模式最危险?

男的有自己的独立工作,自己赚自己花。

女的有自己的独立工作,自己赚自己花。

然后如果再丁克。

早点离婚吧,离婚的时候你们双方都会发自内心的如释重负,感叹到终于自由了,单身万岁。

这样的家庭只有一种情况可以维持的下去,那就是男女双方有一个共同的伟大理想,非常艺术和梦幻的理想,彼此有灵魂上的共鸣,互相引为知己。

这概率比彩票还低,而且哪天你爱好换了,双方马上就不是知己了。

所以中产工薪阶层的离婚率是最高的,因为对彼此的依赖度最弱。

\zd{你不需要我,我也不需要你,那我们为什么还要吊死在一棵树上,放过彼此不好么。}

\zd{中产阶层想要维系婚姻到白头偕老,必须要找到一个彼此互相需要的点,这个点就是你们婚姻的核心,没有这个点,婚姻早晚完蛋。}

夫妻一起打拼,在生活的重压下苦苦支撑,彼此依偎取暖,这是最稳定的婚姻模式。

但对于很多上班的工薪族来说,这个不具备普遍适用性。

如果你们家找不到那个彼此需要的点,那就必须考虑最原始的婚姻家庭模式了。

没错,就是孩子,同时拥有你们俩人血脉的孩子。

\zd{养孩子越难,家庭就越稳固。}

没有孩子的时候,离婚了自然可以当一个单身贵族,但如果你想把孩子带大带好,那绝对不是一个人可以完成的任务。

此时,你们的人生就有了共同奋斗的目标,你们俩缺了一个都不行。

此处拒绝女拳警告,我只是说孩子存在的客观意义,没说其他的。

当然,这里还有个漏洞,就是夫妻双方都必须想倾尽全力的去养孩子,都想给孩子最好的生活,否则这个共同的人生目标依然不存在。

如果夫妻中有一方压根不在乎孩子的死活,那有孩子也没用。

所以如果你是中产阶级,那么你在筛选人生另一半的时候,有没有奋斗心,脾气好不好这些东西,其实都是次要的。

\xhx{你最应该关注的,是对方对小孩子的态度,是不是喜欢小孩,未来的人生计划中是怎么规划安排小孩的,有没有为了小孩努力奋斗一生的打算。}

另外,在生育孩子后,夫妻双方都要轮流承担照顾孩子的任务,在金钱和抚育上平分责任,绝不能单独由一方出力。

这样的婚姻,也很稳固。

而那些觉得小孩是累赘和负担,只希望自己开开心心活一辈子的,极端一点直接丁克的。

这种生活方式不是不可以,但如果你和这样的人结为夫妻,走不到最后的概率极大。

毕竟,家花永远没有野花香。

你们俩毫无共同的人生目标,也各自生活的都很好,没有携手共走人生路的必要性。

那何必要牺牲自由,受到婚姻的约束呢。

同床异梦,是必然的结果。

\zd{超等婚姻}:双方有共同的人生理念,且这个人生理念自己无法完成,必须和知己一起完成,在灵魂上互相共鸣。

案例:这种人基本只存在于小说中,我没见过。

\zd{第一等的婚姻}:双方有共同的人生事业,且这个事业无法自己完成,必须和对象一起完成,彼此扶持才能让自己的生活水平更好,离开对方生活水平会迅速掉落。

案例:街头夫妻店、男耕女织、男主外女主内,此类婚姻白首偕老的概率非常大,有没有孩子都会很恩爱。

\zd{第二等的婚姻}:双方有各自的事业,但拥有共同的生活目标,就是养育后代,双方都无力独自承担抚养孩子的任务。

案例:这是最常见的夫妻家庭模式,大部分人都是这样,白首偕老的概率也非常大,但存在少量家庭纯粹为了孩子而维系婚姻的情况。

\zd{第三等的婚姻}:夫妻A方依赖于B方,失去B方则生活水平急剧下降。但B方并不依赖于A方,A也不能给与B任何帮助。

案例:夫妻一方飞黄腾达,远远脱离了原有阶级,而另一方未能跟上时通常会变成这样的婚姻模式,A不管怎么做都很难保住婚姻,离婚概率极大,唯一选择是忍气吞声,只求名分,留一个名存实亡的婚姻。

\zd{第四等的婚姻}:双方无任何共同生活目标,生活独立,不需要对方的协作,结婚仅为了“找个人陪下自己”。

案例:怕被喷,不提了,不过这是双方自我感觉最良好,但地雷程度最大的婚姻模式,夫妻双方通常还算是社会精英。

\zd{不管你是男是女,不管你已婚未婚,了解以上婚姻的本质,对你拥有一个更幸福美满的婚姻生活,都是有极大帮助和指导意义的。}

千万不要被外面那些毒鸡汤所迷惑,看似正确实则害人,大家都是成年人了,应该有自己的独立思考判断能力。

\zd{我希望看到此文的所有人,都能白头偕老。}

\end{document}

