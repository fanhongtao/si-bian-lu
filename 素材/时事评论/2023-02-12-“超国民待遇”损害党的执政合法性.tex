\documentclass[UTF8, 11pt, oneside]{ctexart}

\usepackage{geometry}
\geometry{a4paper,left=2cm,right=2cm,top=2cm,bottom=1cm}

\usepackage{graphicx}

\usepackage{hyperref}
\hypersetup{colorlinks=true, linkcolor=red}

\linespread{1.6}

\def\articletitle{“超国民待遇”损害党的执政合法性}

\usepackage{fancyhdr}
\usepackage{ifthen}
\pagestyle{fancy}
\fancyhf{}
\setlength{\headheight}{14pt}
\fancyhead[R]{\ifthenelse{\value{page}>1}{\thepage}{}}
\fancyhead[C]{\ifthenelse{\value{page}>1}{\articletitle}{}}
\renewcommand\headrulewidth{0pt}

\usepackage{tcolorbox}
\tcbuselibrary{skins}

\begin{document}

北京师范大学于凤政教授这篇重文虽然在有些枝节数字上可能有所偏差,但整篇文章可谓黄钟大吕,振聋发聩!
尤其令人敬佩的是,外国留学生超国民待的问题直接涉及到教育部,
而于教授本身在教育部直属高校工作,却能不顾自身,毅然挺身而出,大胆直言!
于教授从教前是军人,实乃真正的学者!勇士!战士!

和爱国者!向于教授致敬!

\begin{center}
    \LARGE{\articletitle\footnotemark}
\end{center}
\footnotetext{
    原文出自微博用户“登山临水孰与归”的装载的文章《\href{https://m.weibo.cn/status/4868344300179580}{\articletitle}》
}
\begin{center}
    北京师范大学教授 \hspace*{0.5cm} 于风政
\end{center}

古今中外,治国理政,凡涉及到民众权益的事情,自然是本国人优先,外国人在后,
而本国国民一律平等,则是现代政治文明的基本准则与常识。

社会主义核心价值观里有“公正”、“平等”四字。
分配社会资源与社会福利,不是本国人优先,而是外国人优先,何来公正?
把国民分为三六九等,实行差别待遇,何来平等?

可惜,核心价值观讲了多年,学习运动轰轰烈烈,标语口号刷遍城乡每个角落,许多事情还是说归说,做归做。
我国某些群体长期享受“超国民待遇”,广大人民群众非常不满,而政府视而不见,充耳不闻,从不回应。
看来,社会主义核心价值还是停留在口头上。

所谓的“超国民待遇”,主要存在于三个群体,
一是在中国长期生活的外国人(或称为“久居外国人”),
二是台港澳居民,三是少数民族。

正式或非正式久居中国的外国人,是一个日渐扩大的群体,他们享有超出中国国民的特殊待遇,乃是不争的事实。
比如,他们可以在中国各地包括北京上海自由迁徒,不受户籍制度的限制;
他们的孩子可以随时就近入读公办学校,勿需交纳费用;
购买商品房,非户籍本国人需要五年连续交纳社保或纳税记录,外国人通常只需要一年;
他们可以自由生育,不受中国计划生育法律的限制。
他们还可以享受更周到、更体贴入微的政府服务,这也是人所共知的。

外国人享有的这些权利,超越了中国国民。
中国国民的流动须受户籍制度限制,农民工不能在工作地自由落户,
即便是博士、教授、科学家、企业家,户口也不能随意迁移,进入北京上海,更是比登天还难。
本人不能在工作地落户,家属子女老人就不能随迁,子女就没有资格进入公办学校。
买房需要五年以上连续交纳社会保和纳税记录。

中国人超生第三胎须交纳几十万罚款,黑人在广州生第四胎,副市长亲临医院祝贺和慰问。

绝大部分的久居外国人,不是凭专长与贡献,而仅凭外国人的身份,便可以享受到中国国民享受不到的待遇。

近几年,“超国民待遇”中屡屡成为舆论热点的是外国留学生政策。
中国政府为外国留学生提供的待遇太过优裕,而这个群体的整体素质又太差,
与政府为本国学生提供的保障相比,太过悬殊,难以令人理解与接受。

据说是为了实现2049年成为世界第一留学生大国的目标(一个匪亦所思的目标!),
我国通过各种政策、办法要求和鼓励各大学乃至职业技术学院大力招收外国留学生。
我国每年招揽的几十万外国留学生,绝大部分是来自非洲、中东、南亚等落后国家,
许多学生持伊斯兰信仰。

中央和地方政府为外国留学生中的多数人(主要是学位生)提供的待遇,
包括免除全部学费、提供最好的住宿条件且免住宿费、每月发放高达数千元的生活补助等等。
这笔来自中国纳税人的开支,可能每年达到数百亿元。

外国留学生都是优秀人才吗?他们毕业后会成为中国科技的生力军吗?否!
这些主要来自落后国家的留学生,大部人在他们自己的国家也不是最优秀的学生;
在中国的大学里,作为一个群体,他们甚至达不到我国大学生的中等水平,
而不管他们是本科生还是顶着博士生、硕士生的头衔。
我不敢说100\%的外国留学生都是垃圾生,但是,说优秀生凤毛麟角,决不会有一位大学校长否认。

与此形成鲜明对比的是,到今天为止,非常富裕、四处撒钱的我国政府却一直没有建立大学生学费减免制度。
再贫穷的家庭,孩子上大学也必须交学费,交住宿费,自己负担生活费(每月在千元以上)和其他各种开支。

2007年,在温家宝总理的推动下,国家建立了高校学生奖助学金制度,
设立励志助学金,来自贫困家庭的学生(比例不超过在校生20\%)每人每年补助2000元;
设立面向贫困生中的优秀生的励志奖学金,比例为在校生的3\%,每人每年奖励5000元。
资助和奖励的比例很小,额度很低,而且获得这些资格,
学生必须提供村、镇和县级民政部门的贫困家庭证明,还要在班级内进行公开评选,
在学校或学院公示,实际上就比哪个学生更贫困、更可怜。
这就必然地侵犯学生的隐私,伤害他们的自尊,
因此,许多有骨气的学生甚至不愿申请这些奖助学金。

中国并不是每个家庭都有能力负担一个大学生。
学生考取大学却因家庭负担不起而自杀的事件时有所闻,在校生因贫困而轻生的事件也屡见不鲜,
我本人就负责处理过这类的事件。

对外国人给予如此离谱的超国民待遇,在全世界近二百个国家中,绝对找不到第二个。
人们不禁要问:为什么我们的政府对外国人、对外国学生如此厚爱,对本国人、本国学生却如此“苛刻”?
这是在中国的土地上吗?这是中国人自己的政府吗?
虽然这样的质问过于激烈,我觉得应该问,而政府有义务作出回答。

享受“超国民待遇”的第二个群体是台港澳同胞。台湾是中国的一部分,香港澳门已经回归祖国。
台港澳的居民,也是中国的国民。但是,我们的政府并没有把他们与内地居民一视同仁,
而是事事处处把台港澳视为“境外”,把台港澳居民视为一个特殊群体,给予他们与外国人同等的待遇。

严格说来,台港澳居民虽然是中国国民,但他们并没有享受与内地居民同等公共服务和社会福利的权利,
因为他们并不履行作为一国国民应该履行的向中央政府纳税与当兵的基本义务,除非他们在内地工作并纳税。
政府提供的公共务和社会福利来自纳税人的缴税。
只有纳税人才能权利享受这些公共服务与社会福利。这是权利与义务相统一的原则。

在现实中,我们看到的恰恰相反。
在大陆居住和工作的台港澳居民,享有相兰多的特权。
有些地方的房屋限购,不仅为台港澳居民规定了远低于内地非户籍居民的条件,
广东的一些地方(如东莞)甚至规定,港澳居民只需要提供身份证明,便可购房一套,
不需要提交任何的社保与纳税记录。

我所在的珠海市高新区前不久发布2020年公办幼儿园招生公告。
公办幼儿园师资好,费用低,当然是稀缺资源。
招生公告对非户籍居民子女的报名条件是:珠海市高层次人才、博士或博士后子女,
其父母或监护人需要提供社保缴纳和工作合同等材料,
而台湾同胞、港澳人士的子女不需要任何条件,都可以报名。

我看到这个公告,感到无比的悲哀!
珠海近三分之二的人口是非户籍人口,他们中既有大批的企业家,
也有许多没有被认定(或根本没参与认定)为珠海市“高层次人才”的科学家、工程师、学者、企业管理人员,
更有几十万上百万农民工。
他们用自己的劳动、自己的纳税撑起珠海这片天,而到享受公共服务和社会福利的时候,
他们的贡献竟然不如台港澳居民的一张身份证!

我2002年来珠海工作,至今已满18年,
来珠海前我是北京师范大学的正教授、博士生导师、研究生院副院长、校党委研究生工作部部长,
来珠海后长期担任北师大珠海分校的副书记、副校长,
2006年以来一直担任珠海市人大常委会的法律顾问,每年纳税近2万元。
我的工作关系和户籍一直在北京。
对照一下这个幼儿园招生条件,要不是我有博士学位,我的子女也没有资格报读珠海的公办幼儿园。
这岂不荒谬?公正何在?公理何在?

其实,许多人没有查觉的,是中央政府对台港澳地区给予的整体性超国民待遇。
最不公正的,凡是中央政府以及广东、福建等地方政府制定的涉及台港澳地区的开放政策,
都是单向的,即大陆对台港澳开放,却不要求台港澳对大陆同等开放。
比如香港的医生、律师、会计师、建筑师可以来内地执业,
大陆的专业技术人员却不能到香港执业。
由政府主导的与台港澳地区的经济贸易服务,都是片面地对台港澳地区提供优惠。
比如,两岸服贸协议显著地让利于台湾。
中央政府曾积极推动扩大进口台湾的农产品、水产品,
而对同时期我国南地区农渔民遇到的同样的产品销售困难,却无特殊关照。

澳门、香港向珠海、深圳要土地,要水域,要生活物资供应,中央政府有求必应,片面照顾,
一味让利,全然不讲平等互利合作双赢,亦不顾及毗邻内地人民的利益和感受,
因此也惯成了港澳动辄向中央、向内地伸手要这要那并视其为理所当然的坏毛病。

享受“超国民待遇”的第三个群体是少数民族。
计划生育,对汉族强制推行“一胎化”,少数民族则可以生两胎、三胎,甚至完全没有限制,
有的民族多生还要重奖(如朝鲜族)。
这一政策不但对汉族群众不公,更造成边疆地区汉族人口比例的下降,
潜在地对国家安全产生重大影响。

少数民族世代享受中考、高考考试加分政策。
在贵州,享受高考加分的少数民族学生几乎达到全部考生的二分之一。
20分的加分,可把实际考分居全省第一名的学生挤到几十名以外,把考分居前100名的学生挤到千名以外,
而少数民族学生仅凭一个民族身份便可以通过加分把名次提前几十名、几百名甚至几千名。
出生在同一样地方,说同一种语言,读同一所学校,过同样的生活,
仅仅因民族不同(且不说不少少数民族是“汉改少”的假“少”),竟然享受到如此不同的政策待遇!

教育部对高考名额的投放,长期、显著地倾向于少数民族地区,
以致形成广东、江苏这样经济最为发达、对国家和其他地区贡献最大的地区的适龄青年入学率、
高考录取率、重点大学录取率等均远远低于青海、宁夏、新疆等少数民族地区的极不公平现象。

在公务员考试或事业单位人员录用考试中,少数民族同样享受加分待遇,
以致在汉族占人口多数的少数民族自治地方,录用的公务员却几乎是清一色的少数民族,
形成了对汉族的“逆淘汰"。

政府给穆斯林发放牛羊肉补贴,对不吃猪肉的国民群体如素食者、佛教徒并无任何类似补贴。
少数民族犯罪,过去有“两少一宽”,现在也并未一视同仁。
这不,浙江大学的一名哈萨克族学生犯了强奸罪,被判了1年6个月的刑,
浙大竟然因他是少数民族而仅仅给予“留校察看”处分!
如果是汉族学生,那一定是毫不犹豫开除学籍的。

政府对穆斯林的宗教信仰和宗教活动给予特别的宽容,也是尽人皆知的。

中国是一个多民族国家,民族平等乃是宪法准则,亦为各族人民真诚拥护和接受。
对少数民族的优惠政策和无原则的照顾,客观上把少数民族“捧”成了一个享有特权的"高贵”群体,
从而形成了对汉族人民的系统性、制度性逆向歧视。

汉族是我国的主体民族,新中国的建立与发展,主要靠汉族人民的斗争和贡献。
在涉及重大利益的问题上,过分照顾少数民族,而侵犯汉族人民的利益,
失去了基本的公平正义,自然要引起广大汉族群众的不满甚至愤怒。

为什么我们的政府敢于冒着绝大多数国民都很不高兴的危险,执意给那几个少数群体以“超国民待遇”呢?
依据可以说出许多条,但基础性的认知,是认定人是趋于利益的动物,故钱能通神。

撒钱可以展示中国的慷慨与友善,有利于促成由中国主导的“人类命运共同体”;
向台港澳居民让渡利益,可以使他们体会到党的伟大和社会主义祖国的关爱,
有利于凝聚民心民意,促进国家统一;
给予少数民族以特殊关照,可以展示以汉人为主体的政府不但不歧视少数民族,
反而要给他们更多的照顾,由此产生的感恩心情有利于增强少数民族对中华民族、中央政府的认同。

然而,这只是政府的一厢情愿,因为这种对人性的认知是片面的。
人与人之间、民族与民族之间、人和民族与政府之间、中央政府与个别地区之间,
他们的凝聚、和谐、融洽与团结,决不是具体可见的有限经济利益就可以决定的。
影响人际关系、政治关系、国际关系的,归根到底,还是价值观和道义。

近年来我国的国际关系、台港澳地区与中央政府的关系、国内各民族之间关系的发展变化,
已经有力地证明了这一点。
中国共产党传统的统战法宝并不以经济利益为核心,许多人不懂得这一点。

“得民心者得天下,失民心者失天下”,
这句中国人耳熟能详的老话,表达的乃是人类政治生活的一个真理。
国家政权、执政党的执政地位来自人民的认可,这在政治学上叫做“合法性”。
政治合法性越高,政权或执政党的地位就越稳固,反之,就越不稳固。
失去了政治合法性,即失去了人心,迟早要丢掉执政地位。

得民心,就要知民情,顺民意。
凡是人民希望的、需要的,就去做;
凡是人民反感的、反对的,就不要做;
做错了的,要勇于纠正。
只有真诚地倾听人民的呼声,回应人民的诉求,平息人民的不满,才能保持人民对党和政府的拥护与支持。
如果对绝大多数人民群众不满的事情视而不见,充耳不闻,
或者无动于衷,听之任之,或者固执己见,不愿改正错误,就会失去人心,即削弱党的护政合法性。

“超国民待遇”严重影响和削弱共产党的执政合法性,已经演变为一个严重的政治问题。
应当引起党的各级领导机关和各级政府的重视。

\end{document}

