\documentclass[UTF8, 11pt, oneside]{ctexart}

\usepackage{geometry}
\geometry{a4paper,left=2cm,right=2cm,top=2cm,bottom=1cm}

\usepackage{hyperref}
\hypersetup{colorlinks=true, linkcolor=red}

\usepackage[shortlabels]{enumitem}

\linespread{1.6}

\setlist[itemize]{nosep, left=\parindent}

\newcommand{\jg}{\vspace{2em}} % 间隔

\begin{document}

\begin{center}
    \LARGE{为什么有人会真心喜欢西方教?\footnotemark}
\end{center}
\footnotetext{
    原文出自微博号 “\href{https://m.weibo.cn/u/7549409992}{御风司水梁美丽}” 4 月 22 号的文章 “\href{https://m.weibo.cn/detail/4761059930671420}{为什么有人会真心喜欢西方教?}”
}

有朋友找我解惑,问我:“道长,我身边有一些朋友信仰西方教,有高知、教授。我一直很讨厌西方教,就想说服他们,现在我发现他们是真心喜欢西方宗教的那一套。我都有点动摇了,他们是真的喜欢吗?”

\jg

我说:相信我,我也认为他们是真的喜欢西方教,因为人生最大的遗憾莫过于仇人死的太早、恩人死的太晚。

西方价值观的根本是感恩神(造物主),中国价值观的根本是感恩人。信仰虚无缥缈的神(造物主)不需要感恩人,感恩神太容易,感恩人太累。

感恩神只需要动动嘴,感恩人需要实际行动。

\jg

对于自私的人来说,仇人死了他会愤愤不平,我大仇未报,怎么你就死了呢?恩人死了他会有解脱感,施恩的人至于死了,我不必再感恩他了。

感恩死人或者虚无缥缈的造物主只需要唱唱赞歌就可以表达,放下屠刀立地成佛;感恩活人需要的代价太大了,那么自私的人就会喜欢感恩死人。

\jg


那些所谓高知和教授,与其说说受到西方价值观影响,不如说他们本身就自私的人,知识越多越反动,恰恰是因为他们看的太明白了,才一切从自身利益出发,才真心喜欢不需要感恩人的西方教。

他们真心喜欢的不是西方的神,而是喜欢西方价值观里的那种彻底的自私自利。God窃取了上帝和神的名义,实际上,东西方文化里的神压根不是同一种玩意。

\jg

西方的God骨子里就是自私的,把人类作为自己的私有财产,凌驾于人类之上,不受人世间美好的道德和规范约束,肆意妄为、喜怒无常。

东方的神受到了约束和规范,必须有良好的德行才可能有神权上的正义性,拥有大神通没有好品行的就是为祸人类的魔头。

\jg

西方的God只是为了自己,东方的神是为了众生,西方人只不过是造物主的羔羊,中国人是自己命运的主宰。

当然,既然西方的God是自私自利的,西方的价值观的本质就是自私自利的,他们感恩造物主了,就不需要感恩人了,就没有负担,就没有了人性,就没有负罪感了。

\jg

想一想,当欧洲人踏上美洲大陆,印第安人帮助了这些欧洲人,然后欧洲人毫无人性、毫无负罪感的杀死了印第安人,抢夺了他们的土地,冠名曰“感恩节”。

在中国价值观里,这该是多么没有人性的行为,该是多么忘恩负义的行为,他们却堂而皇之宣传,毫无羞耻之心,这就是恶之花的绽放,这就是人类文明的耻辱,这就是最自私卑劣的行为。

\jg

假如当你帮助了我,我口口声声感谢上帝的原因是什么呢?

哪怕你帮助了我,我不需要感谢你,这是我应得的。

感谢你我需要背负上道德的枷锁,感谢上帝我可以获得最大利益,你的帮助与我没有关系。

\jg

也正因为西方文明中骨子里的自私,他们一贯感恩神灵,却吝啬于感恩祖宗、吝啬于感恩人,没有为人树立偶像的传统,甚至排斥偶像崇拜。

他们真心喜欢西方教,就是因为不需要感恩人就不用背负良知的束缚,一切都是神的安排,只要感恩神就够了,这样,就可以逃避自己的责任。

\jg

从本质上来说,感恩神让他们接受别人的帮助而毫无愧疚之心,甚至可以随意加害帮助自己的人。

但我们还有良知啊,我们愿意受到良知的束缚,愿意通过这种束缚来完善自我,让自己成为一个好人。

\jg

感恩人,是真正的感恩;感恩神,是虚假的交易。

感恩人,是真正的爱;感恩神,是自私的表现。

感恩人,是勇于承担;感恩神,是逃避责任。

\jg

我们崇拜偶像,立功立德立言的人就是我们的偶像,因为这种感恩,让我们越来越完美,让我们愿意成为他们那样的人,为中华民族为整个世界做自己的努力。

我们褒奖真知,而不是舍弃良知。我们褒奖善意,而不是舍弃良心。

\jg

当一个民族愿意被良知束缚而不是被自私放纵时,当一个民族选择温情于人而不是奉献于神时,当一个民族愿意承担感恩而不是逃避人性时,他必然是个成熟的文明,哪怕经历各种苦难,都会最终跨越坎坷,毕竟我们的征途是星辰大海。

如果神是一个最高的褒奖,那么,我们宁可把神的称号给真正的人,也不愿意把神的称号给虚无缥缈的造物主。

\jg

我们跟他们实际上是两类人,一类是彻底自私自利的人,一类是有点良知的普通人,我们都有自己的利益,但我们还有羞耻之心,我们还有良知,我们不会忘恩负义。

他们喜欢西方教可能是真心的,我们坚守自己的底线也是真心的,这都符合人类的本性。

\jg

从某种意义上来说,西方教占领的领域大,也是因为这个世界上,自私的人占了多数,卑劣的强盗文明和不知廉耻的感恩节还能受到吹捧。

我相信随着中国的逐渐强大,人类文明会迎来真正的曙光,中华民族伟大复兴给予人类文明的将是一次拨乱反正的机会。

\jg

你一直很讨厌西方教,是说明你内心不愿意做一个那么没有廉耻、没有道德、没有感恩心的人,你的内心是光明的,不是龌龊的。

做好人总是比做坏人难,我们就不做好人了吗?

\jg

无能的存在才会以灭世来威胁人类,卑鄙的存在才会拿人的子女来试探信仰,狂暴的存在才喜怒无常,愚昧的存在才说自己全知全能,想吃人的存在才会把人类当做美味的羔羊。

人类永不为奴,哪怕它是所谓的造物主,哪怕它利用了人性中的自私。

\jg

我们不是任何神或者造物主的棋子,人创造了神,不是神创造了人。

坚定信念,坚定文化自信,坚定文明自信,我们一定行。

他们真心喜欢西方教,是因为他们真心自私。

\jg

刚刚有网友说:自私自己怎么样都舒服,但对于人类来说是一种破坏和毁灭,无私很难很痛苦,但确是最崇高的精神,也是人类得以延续的根本。

是的,为了人类,我们必须要实现中华民族的伟大复兴!同志们,共勉!

\jg

这也是为什么中国接受了共产主义、西方世界视之为洪水猛兽的原因。共产主义思想是西方的异类,违背了他们自私自利、虚伪贪婪、精致利己主义的原则;唯有在中国,天下为公、大同世界、人类命运共同体的价值观下,才能生根发芽茁壮成长。

它来自于西方,生长与中国,这是人类文明志同道合的相遇!

\end{document}

