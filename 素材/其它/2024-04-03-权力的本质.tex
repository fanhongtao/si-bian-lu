\documentclass[UTF8, 11pt, oneside]{ctexart}

\usepackage{float}

\usepackage{geometry}
\geometry{a4paper,left=2cm,right=2cm,top=2cm,bottom=1cm}

\usepackage{graphicx}

\usepackage{hyperref}
\hypersetup{colorlinks=true, linkcolor=red}

\linespread{1.6}


\def\articletitle{权力的本质}

\usepackage{fancyhdr}
\usepackage{ifthen}
\pagestyle{fancy}
\fancyhf{}
\setlength{\headheight}{14pt}
\fancyhead[R]{\ifthenelse{\value{page}>1}{\thepage}{}}
\fancyhead[C]{\ifthenelse{\value{page}>1}{\articletitle}{}}
\renewcommand\headrulewidth{0pt}

\usepackage{tcolorbox}
\tcbuselibrary{skins}


\newcommand{\zd}[1]{\textbf{\textcolor[RGB]{123,12,0}{#1}}} % 重点

\newcommand{\yh}[1]{% 引用
    \begin{tcolorbox}[enhanced,
        frame hidden, interior hidden,
        before skip = 5mm, left skip=10mm,
        borderline west={5pt}{0pt}{gray!50}]
        #1
    \end{tcolorbox}
}

\newcommand{\biaoti}[1]{% 标题
    \section*{#1}
}

\begin{document}

\begin{center}
    \LARGE{\articletitle\footnotemark}
\end{center}
\footnotetext{
    原文出自公众号“远方青木”的文章 《\href{https://mp.weixin.qq.com/s/_HfWhFQzflDnXSI8i86kzg}{\articletitle}》
}

这世上人人都想获得权力,从政想当官,进公司想当高管,回了家都想争夺家庭主导权。

\zd{权力是个好东西,人人都想获得,但很多人的认知出现了问题,在一条自以为是的错误道路上越走越远。}

今天我就和大家聊一聊,我认为的权力本质。

有一条谣言传播的极其广泛,那就是当官靠的是领导提拔,所以想当官要学会拍马屁,把领导伺候舒服了才能当官。

信这个的人,没有一个能当官的,越是觉得点头哈腰就能获取权力的人就越不可能获取权力。

\zd{因为这些人认为权力来源于职位,只要有了职位就能得到权力,这个是大错特错的。}

别说那些流官职位,就算是封建时期的皇位,你以为坐上皇位就一定能拥有权力?

整个封建社会所有的官都是不稳固的,但皇位不是,王朝的开创者宁可牺牲一切都要从制度上确保自己的继承人一定可以坐稳皇位,掌控国家权力。

即便如此,很多皇帝依然被人架空。

\zd{因为权力的本质不来源于职位,只来源于赏罚二字,其中赏为主,罚为辅。}

惩罚是最简单获取权力的途径,也是最原始的途径,因为人类天然害怕遭遇损失,所以如果你拥有剥夺一个人财产乃至于生命的力量,让这个人产生不服从你会很惨的认知时,你就拥有了指挥这个人的权力。

暴力是权力的底层逻辑,所有的人类国家都是依托暴力建立起来的,但简单依托暴力建立起来的权力极其不稳固,人民的拥护度极低,现代社会里所有的军政府都没有好下场就是这个原因。

所以罚是权力的必要来源,但必须为辅。

\zd{赏为主的原因,是这世界上没有任何人类是服管的,但所有的人类都崇拜强者,其中男性比女性更崇拜强者,最核心的原因就是跟着强者有肉吃。}

如果跟随一个强者能让你获得比自己一个人单干更多的肉,那你就肯定会服从这位强者的命令,那这位强者就获得了指挥你的权力,而你心甘情愿。

举个例子,三国时期的曹操,在东汉末年不过是一个小官,没有皇位,甚至也不姓刘。

但跟着曹操有肉吃,曹操能带着大家打胜仗,不听曹操的就要输。

一仗、两仗、三仗。。。无数的战争打下来,曹操是战绩最好,地盘最大的那个,大量的人服了曹操,那曹操就获得了权力。

而汉献帝刘协没有能力向大家证明跟着他比跟着曹操强,所以汉献帝就威胁不了曹操的权力。

而刘备和曹操掌握权力的多寡,也是双方互相抢肉的过程中打出来的。

因此,历代王朝的开国皇帝都具备至高无上的权力,有没有皇位都不影响他的权力,因为整个王朝的地盘都是他自己亲手打下来的。

而皇N代如果不能向大臣们证明听他的有肉吃,那就可能会被人阳奉阴违,慢慢架空,能力差的皇帝甚至无法了解京师之外的任何事情,完全被手下当傻子哄。

\zd{现代社会当官也是一样,有了职位之后你会获得一个初始权力,但这个权力的保质期极短,所以新官上任必须三把火,因为新官必须尽快向手下展示自己的能力。}

刚才说过一个人只要能让别人认为服从他就会有肉吃,那他就会获得权力。

\zd{反过来,如果一个官让别人觉得服从他也没啥好处,那这个官就会逐渐失去权力。}

\zd{如果这个官还同时让人觉得不服从他也没啥坏处,那这个官会在极短的时间内彻底失去权力,手下纷纷阳奉阴违,政令不出自己办公室,整个部门的运转处于瘫痪状态。}

这个时候如果不管,那这个部门很快就会出事,然后这个官也就当到头了。

\zd{如果求助于提拔自己的领导,依靠领导的力量来解决这个问题,那这个官也就当到头了。}

因为领导提拔一个自己的下属当官,是希望这个下属帮自己分忧解难,而不是自己帮这个下属分忧解难。

所谓的点头哈腰和拍马屁,那只是表面的,只是因为领导要提拔肯定提拔服从度高的下属,不可能提拔一个不服自己管理的下属,否则自己反而会失去领导权力。

但提拔哪个人当下属,肯定不是因为这个人马屁拍的比别人好,因为这仅仅只是一个服从管理的态度表达,过及格线就行,剩下的还是要看能力。

这个能力,指的是你能不能压服那个部门,能不能镇得住那个职位。

当你看到一个马屁精被提拔了上去,然后负责管理的那个部门运转的井井有条,三五年都没出什么乱子,说的话手下人都听。

\zd{那就说明提拔的一点问题都没有,而你如果简单以为这个人只是靠拍马屁上位的,然后只天天偷学别人的马屁功夫,那你这辈子都上不了位。}

\zd{如果你觉得放头猪在那个领导岗位上都能管得好,我当领导了还有人敢不听我的话?下属难道不是天经地义100\%执行我命令的吗?}

\zd{那你肯定没当过领导,最小的领导都没当过,而且这辈子不可能当领导。}

\zd{但凡管过三个人,都不可能这么幼稚。}

反过来,如果一个人没有职位,但总能搞定别人搞不定的事情,解决别人不能解决的问题,能把蛋糕做大,能弄来很多的肉,那这个人就成了强者,具备了拥有权力的基础。

但强者并不一定具有权力,独狼式强者再强也不会有权力。

如果一个强者具备带着大家一起做大蛋糕的能力并且愿意分蛋糕,让别人认定跟着他有肉吃,那这个强者就一定会获取权力,无论目前是什么职位。

举个例子,当大量的人相信跟着毛主席或邓小平走自己的生活就能变好的时候,无论毛邓是什么职位,他都具有权力,免职也没用。

落实到个人头上,那就是一个人只要让周围的人相信跟着他之后,自己能在某一方面变得更好,那这个人身上就会自然而然散发出所谓的“领袖气质”。

想从芸芸众生中脱颖而出,记住三句法宝:

\yh{
    这件事就按我说的办,保证能赢,出了事我负责。
}

\zd{以上三句法宝只要说出口,任何人都会开始散发“领袖气质”,小女生尤其迷恋这个,如果说出的话多次实现了,那这个人就会获得权力,被众人拥护为领袖。}

以上方式是获得权力的煌煌大道,得到的权力纯粹又强大,但也有其他获取权力的方式,简单粗暴但根基不稳,那就是洗脑。

正常获取权力的方式是向其他人证明跟着自己有肉吃,不跟着自己会很惨。

\zd{而通过洗脑的方式获取权力是不证明,单纯让你相信跟着他有肉吃,不跟着他会很惨即可。}

比如说神权,不需要那么费劲的带着你吃肉,只需要虚构天堂和地狱,让你相信听从神权的能上天堂,不听从神权的会下地狱,那神权就能获取权力,类似套路的还有印度的种姓制。

而中国的封建君权,开国皇帝为确保自己的傻儿子也能掌控帝国,采用的是宣传君权神授,顺我者昌逆我者亡的那一套,然后愚民。

你不需要知道那么多,只需要知道听皇帝的会全家昌盛,不听皇帝的会死全家,那就行了。

举个现代的例子,美国就很擅长这一套,直接跳过证明过程,让人无脑相信听美国的一定会发财,不听美国的一定会倒霉。

不信你看各路公知,其思维模式是不是都是这个套路,这就是被洗脑后的表现。

美国确实通过对一些国家的拉踩战术来证明了顺我者昌逆我者亡,但大部分权力还是通过洗脑获得的,很多美国大片都设定成美国拯救世界,出现中国拯救世界的电影就急的跳脚,原因就是这个。

这种洗脑方式获取的权力,简单粗暴,但根基不稳,很容易被拆穿从而毁灭,非煌煌大道,但因为见效快被一些短视的掌权者所喜爱。

如果把权力落实到家庭,又是一种什么样的情况?

很多觉得只有当官和当老板才会有权力,这是错的,权力在人类社会无处不在,有人的地方就有权力,家庭里到处都充斥着权力。

家庭里到底该谁指挥谁,就完全遵循权力的本质。

\zd{谁能带别人吃肉,谁就天经地义的掌控权力,这里的肉指的是另一个人需要的东西,包括但不限于钱、性、安全等。}

有人需要钱,如果你能赚钱,那ta就一定会听你的,但如果ta不缺钱,对钱不怎么感兴趣,那你能赚钱就未必能获得权力,这一条多见于男对女。

有人需要性,特别饥渴,虽然性是双向的,但夫妻两人对性的需求度可能不一样高,这个时候需求度低的那个利用这点,成了能“带别人吃肉”的那个人,就有可能让另一人言听计从,从而掌控家庭权力,这一点多见于女对男。

还有更常见的是父母对小孩。

在孩子幼小的时候,父母天然拥有对孩子的权力,这不是因为孩子是自己生出来的所以具备权力,而是因为孩子需要的一切“肉”都只能由父母提供,所以父母是能带孩子“吃肉”的那个人,因此孩子就会服从父母的管理,父母就拥有了对孩子的领导权力。

等孩子越来越大,依靠自己每个月捕到的肉越来越多,最终可以自食其力的时候,父母对孩子的权力就会越来越小。

手里还能剩点权力,是因为孝道的传统观念,以及社会制度对不孝之人的惩罚所带来的。

所以等孩子拥有自食其力的能力,在社会上独立存活后,家庭的指挥权就会从老人手中转移到孩子手里,因此老人对孩子多是依靠亲情羁绊,而不是强硬管理。

还有其他几种家庭权力的获取方式,比如家暴和保护全家,这是古代男性依赖武力得到家庭权力的一大途径,现代社会基本没啥用了就不提了。

还有鼎鼎大名的PUA,其实就是洗脑,跳过证明过程,直接让别人相信跟着你会更好,不跟你会很惨,然后获得家庭权力,这个属于歪门邪路,得到的权力也极易破碎,很难持续存在。

以上就是在人类社会中获取权力的几大途径,我今天大概就说那么多。

如果你想当官,如果你想当高管,如果你想成为学校里面的风云人物,如果你想让老婆/老公听你的。

\zd{记住我刚才说的三大法宝,多说多做,那就是“这件事就按我说的办,保证能赢,出了事我负责。”}

\zd{敢说能实现,那你就肯定有权力,传说中的所谓领袖气质也会自动上门。}

\end{document}

