\documentclass[UTF8, 11pt, oneside]{ctexart}

\usepackage{geometry}
\geometry{a4paper,left=2cm,right=2cm,top=2cm,bottom=1cm}

\usepackage{hyperref}
\hypersetup{colorlinks=true, linkcolor=red}

\linespread{1.6}

\usepackage{varwidth}
% 偈子
% 注:当 “偈子” 跨页显示时,需要手工处理跨页。
\newenvironment{jizi}[1]{
    \vspace{-1em}\begin{center}
        #1%
        \begin{varwidth}[t]{\linewidth}
}{
        \end{varwidth}
    \end{center}
}


\begin{document}

\begin{center}
    \LARGE{六祖大师法宝坛经}
\end{center}

\begin{center}
    六祖惠能大师说 \hspace{3em} 门人释法海 集录
\end{center}


\section*{自序品第一}

时,大师至宝林,韶州韦刺史(名璩)与官僚入山请师,出于城中大梵寺讲堂,为众开缘说法。师升座次,刺史官僚三十余人,儒宗学士三十余人,僧尼道俗一千余人,同时作礼,愿闻法要。

大师告众曰:“善知识!菩提自性,本来清净,但用此心,直了成佛。

“善知识!且听惠能行由,得法事意。惠能严父,本贯范阳,左降流于岭南,作新州百姓。此身不幸,父又早亡,老母孤遗,移来南海,艰辛贫乏,于市卖柴。时,有一客买柴,使令送至客店;客收去,惠能得钱,却出门外,见一客诵经。惠能一闻经语,心即开悟,遂问:‘客诵何经?’客曰:‘《金刚经》。’复问:‘从何所来,持此经典?’客云:‘我从蕲州黄梅县东禅寺来。其寺是五祖忍大师在彼主化,门人一千有余。我到彼中礼拜,听受此经。大师常劝僧俗,但持《金刚经》,即自见性,直了成佛。’惠能闻说,宿昔有缘,乃蒙一客,取银十两与惠能,令充老母衣粮,教便往黄梅参礼五祖。

“惠能安置母毕,即便辞违。不经三十余日,便至黄梅,礼拜五祖。祖问曰:‘汝何方人?欲求何物?’惠能对曰:‘弟子是岭南新州百姓,远来礼师,惟求作佛,不求余物。’祖言:‘汝是岭南人,又是獦獠,若为堪作佛?’惠能曰:‘人虽有南北,佛性本无南北;獦獠身与和尚不同,佛性有何差别?’五祖更欲与语,且见徒众总在左右,乃令随众作务。惠能曰:‘惠能启和尚,弟子自心,常生智慧,不离自性,即是福田。未审和尚教作何务?’祖云:‘这獦獠根性大利!汝更勿言,著槽厂去。’

“惠能退至后院,有一行者,差惠能破柴踏碓。经八月余,祖一日忽见惠能,曰:‘吾思汝之见可用,恐有恶人害汝,遂不与汝言,汝知之否?’惠能曰:‘弟子亦知师意,不敢行至堂前,令人不觉。’

“祖一日唤诸门人总来:‘吾向汝说,世人生死事大,汝等终日只求福田,不求出离生死苦海;自性若迷,福何可救?汝等各去,自看智慧,取自本心般若之性,各作一偈,来呈吾看。若悟大意,付汝衣法,为第六代祖。火急速去,不得迟滞,思量即不中用;见性之人,言下须见。若如此者,轮刀上阵,亦得见之!’

“(喻利根者)众得处分,退而递相谓曰:‘我等众人,不须澄心用意作偈,将呈和尚,有何所益?神秀上座,现为教授师,必是他得。我辈谩作偈颂,枉用心力。’诸人闻语,总皆息心,咸言:‘我等已后依止秀师,何烦作偈?’

“神秀思惟:‘诸人不呈偈者,为我与他为教授师,我须作偈,将呈和尚。若不呈偈,和尚如何知我心中见解深浅?我呈偈意,求法即善,觅祖即恶,却同凡心夺其圣位奚别?若不呈偈,终不得法。大难!大难!’

“五祖堂前,有步廊三间,拟请供奉卢珍,画《楞伽经变相》,及《五祖血脉图》,流传供养。神秀作偈成已,数度欲呈,行至堂前,心中恍惚,遍身汗流,拟呈不得。前后经四日,一十三度呈偈不得。秀乃思惟:‘不如向廊下书著,从他和尚看见,忽若道好,即出礼拜,云是秀作;若道不堪,枉向山中数年,受人礼拜,更修何道!’是夜三更,不使人知,自执灯,书偈于南廊壁间,呈心所见。偈曰:

\begin{jizi}{“}
    身是菩提树, 心如明镜台; \\
    时时勤拂拭, 勿使惹尘埃。
\end{jizi}

“秀书偈了,便却归房,人总不知。秀复思惟:‘五祖明日见偈欢喜,即我与法有缘;若言不堪,自是我迷,宿业障重,不合得法。’圣意难测,房中思想,坐卧不安,直至五更。

“祖已知神秀入门未得,不见自性。天明,祖唤卢供奉来,向南廊壁间,绘画图相,忽见其偈,报言:‘供奉却不用画,劳尔远来。经云:“凡所有相,皆是虚妄。”但留此偈,与人诵持。依此偈修,免堕恶道;依此偈修,有大利益。’令门人炷香礼敬,尽诵此偈,即得见性。门人诵偈,皆叹善哉!

“祖,三更唤秀入堂,问曰:‘偈是汝作否?’秀言:‘实是秀作,不敢妄求祖位,望和尚慈悲,看弟子有少智慧否?’祖曰:‘汝作此偈,未见本性,只到门外,未入门内。如此见解,觅无上菩提,了不可得。无上菩提,须得言下识自本心,见自本性,不生不灭。于一切时中,念念自见,万法无滞,一真一切真,万境自如如。如如之心,即是真实。若如是见,即是无上菩提之自性也。汝且去,一两日思惟,更作一偈,将来吾看。汝偈若入得门,付汝衣法。’神秀作礼而出。又经数日,作偈不成,心中恍惚,神思不安,犹如梦中,行坐不乐。

“复两日,有一童子于碓坊过,唱诵其偈。惠能一闻,便知此偈未见本性,虽未蒙教授,早识大意。遂问童子曰:‘诵者何偈?’童子曰:‘尔这獦獠不知,大师言:“世人生死事大,欲得传付衣法,令门人作偈来看。若悟大意,即付衣法,为第六祖。”神秀上座,于南廊壁上,书无相偈,大师令人皆诵,依此偈修,免堕恶道;依此偈修,有大利益。’惠能曰:‘我亦要诵此,结来生缘。上人!我此踏碓,八个余月,未曾行到堂前。望上人引至偈前礼拜。’

“童子引至偈前礼拜。惠能曰:‘惠能不识字,请上人为读。’时,有江州别驾,姓张名日用,便高声读。惠能闻已,遂言:‘亦有一偈,望别驾为书。’别驾言:‘汝亦作偈?其事希有!’惠能向别驾言:‘欲学无上菩提,不得轻于初学。下下人有上上智,上上人有没意智。若轻人,即有无量无边罪。’别驾言:‘汝但诵偈,吾为汝书。汝若得法,先须度吾。勿忘此言。’惠能偈曰:

\begin{jizi}{“}
    菩提本无树, 明镜亦非台; \\
    本来无一物, 何处惹尘埃?
\end{jizi}

“书此偈已,徒众总惊,无不嗟讶,各相谓言:‘奇哉!不得以貌取人,何得多时,使他肉身菩萨。’祖见众人惊怪,恐人损害,遂将鞋擦了偈,曰:‘亦未见性。’众以为然。

“次日,祖潜至碓坊,见能腰石舂米,语曰:‘求道之人,为法忘躯,当如是乎!’乃问曰:‘米熟也未?’惠能曰:‘米熟久矣,犹欠筛在。’祖以杖击碓三下而去。

“惠能即会祖意,三鼓入室。祖以袈裟遮围,不令人见,为说《金刚经》。至‘应无所住而生其心’,惠能言下大悟,一切万法,不离自性。遂启祖言:‘何期自性,本自清净;何期自性,本不生灭;何期自性,本自具足;何期自性,本无动摇;何期自性,能生万法。’祖知悟本性,谓惠能曰:‘不识本心,学法无益;若识自本心,见自本性,即名丈夫、天人师、佛。’

“三更受法,人尽不知,便传顿教及衣钵,云:‘汝为第六代祖,善自护念,广度有情,流布将来,无令断绝。听吾偈曰:

\begin{jizi}{“‘}
    有情来下种, 因地果还生, \\
    无情亦无种, 无性亦无生。’
\end{jizi}

“祖复曰:‘昔达摩大师,初来此土,人未之信,故传此衣,以为信体,代代相承;法则以心传心,皆令自悟自解。自古,佛佛惟传本体,师师密付本心。衣为争端,止汝勿传;若传此衣,命如悬丝。汝须速去,恐人害汝。’惠能启曰:‘向甚处去?’

“祖云:‘逢怀则止,遇会则藏。’

“惠能三更领得衣钵,云:‘能本是南中人,素不知此山路,如何出得江口?’五祖言:‘汝不须忧,吾自送汝。’祖相送,直至九江驿。祖令上船,五祖把橹自摇。惠能言:‘请和尚坐。弟子合摇橹。’祖云:‘合是吾渡汝。’惠能云:‘迷时师度,悟了自度;度名虽一,用处不同。惠能生在边方,语音不正,蒙师传法,今已得悟,只合自性自度。’祖云:‘如是,如是。以后佛法,由汝大行。汝去三年,吾方逝世。汝今好去,努力向南,不宜速说,佛法难起。’

“惠能辞违祖已,发足南行。两月中间,至大庾岭。

“(五祖归,数日不上堂。众疑,诣问曰:‘和尚少病少恼否?’曰:‘病即无,衣法已南矣。’问:‘谁人传授?’曰:‘能者得之。’众乃知焉。)逐后数百人来,欲夺衣钵。

“一僧俗姓陈,名惠明,先是四品将军,性行粗糙,极意参寻,为众人先,趁及惠能。惠能掷下衣钵于石上,云:‘此衣表信,可力争耶?’能隐草莽中。惠明至,提掇不动,乃唤云:‘行者!行者!我为法来,不为衣来。’

“惠能遂出,坐盘石上。惠明作礼云:‘望行者为我说法。’惠能云:‘汝既为法而来,可屏息诸缘,勿生一念,吾为汝说。’明良久。惠能云:‘不思善,不思恶,正与么时,那个是明上座本来面目?’惠明言下大悟。复问云:‘上来密语密意外,还更有密意否?’惠能云:‘与汝说者,即非密也。汝若返照,密在汝边。’明曰:‘惠明虽在黄梅,实未省自己面目。今蒙指示,如人饮水,冷暖自知。今行者即惠明师也。’惠能曰:‘汝若如是,吾与汝同师黄梅,善自护持。’明又问:‘惠明今后向甚处去?’惠能曰:‘逢袁则止,遇蒙则居。’明礼辞。(明回至岭下,谓趁众曰:‘向陟崔嵬,竟无踪迹,当别道寻之。’趁众咸以为然。惠明后改道明,避师上字。)

“惠能后至曹溪,又被恶人寻逐,乃于四会,避难猎人队中,凡经一十五载,时与猎人随宜说法。猎人常令守网,每见生命,尽放之。每至饭时,以菜寄煮肉锅。或问,则对曰:‘但吃肉边菜。’

“一日思惟:‘时当弘法,不可终遁。’遂出至广州法性寺,值印宗法师讲《涅槃经》。时有风吹幡动,一僧曰:‘风动。’一僧曰:‘幡动。’议论不已。惠能进曰:‘不是风动,不是幡动,仁者心动。’一众骇然。

“印宗延至上席,征诘奥义。见惠能言简理当,不由文字。宗云:‘行者定非常人。久闻黄梅衣法南来,莫是行者否?’惠能曰:‘不敢。’宗于是作礼,告请传来衣钵,出示大众。宗复问曰:‘黄梅付嘱,如何指授?’惠能曰:‘指授即无;惟论见性,不论禅定解脱。’宗曰:‘何不论禅定解脱?’能曰:‘为是二法,不是佛法。佛法是不二之法。’宗又问:‘如何是佛法不二之法?’惠能曰:‘法师讲《涅槃经》,明佛性是佛法不二之法。如高贵德王菩萨白佛言:“犯四重禁、作五逆罪,及一阐提等,当断善根佛性否?”佛言:“善根有二:一者常,二者无常,佛性非常非无常,是故不断,名为不二;一者善,二者不善,佛性非善非不善,是名不二。蕴之与界,凡夫见二,智者了达,其性无二,无二之性即是佛性。”’

“印宗闻说,欢喜合掌,言:‘某甲讲经,犹如瓦砾;仁者论义,犹如真金。’于是为惠能剃发,愿事为师。惠能遂于菩提树下,开东山法门。

“惠能于东山得法,辛苦受尽,命似悬丝。今日得与使君、官僚、僧尼、道俗同此一会,莫非累劫之缘,亦是过去生中供养诸佛,同种善根,方始得闻如上顿教得法之因。教是先圣所传,不是惠能自智。愿闻先圣教者,各令净心,闻了各自除疑,如先代圣人无别。”

一众闻法,欢喜作礼而退。



\section*{般若品第二}

次日,韦使君请益。师升座,告大众曰:“总净心念摩诃般若波罗蜜多。”复云:“善知识!菩提般若之智,世人本自有之;只缘心迷,不能自悟,须假大善知识,示导见性。当知愚人智人,佛性本无差别,只缘迷悟不同,所以有愚有智。吾今为说摩诃般若波罗蜜法,使汝等各得智慧。志心谛听,吾为汝说。

“善知识!世人终日口念般若,不识自性般若,犹如说食不饱,口但说空,万劫不得见性,终无有益。

“善知识!摩诃般若波罗蜜是梵语,此言大智慧到彼岸。此须心行,不在口念。口念心不行,如幻如化、如露如电;口念心行,则心口相应,本性是佛,离性无别佛。何名摩诃?摩诃是大。心量广大,犹如虚空,无有边畔,亦无方圆大小,亦非青黄赤白,亦无上下长短,亦无瞋无喜,无是无非,无善无恶,无有头尾。诸佛刹土,尽同虚空。世人妙性本空,无有一法可得。自性真空,亦复如是。

“善知识!莫闻吾说空,便即著空。第一莫著空,若空心静坐,即著无记空。

“善知识!世界虚空,能含万物色像,日月星宿,山河大地,泉源溪涧,草木丛林,恶人善人,恶法善法,天堂地狱,一切大海,须弥诸山,总在空中。世人性空,亦复如是。

“善知识!自性能含万法是大,万法在诸人性中。若见一切人、恶之与善,尽皆不取不舍,亦不染著,心如虚空,名之为大,故曰摩诃。

“善知识!迷人口说,智者心行。又有迷人,空心静坐,百无所思,自称为大。此一辈人,不可与语,为邪见故。

“善知识!心量广大,遍周法界。用即了了分明,应用便知一切。一切即一,一即一切,去来自由,心体无滞,即是般若。

“善知识!一切般若智,皆从自性而生,不从外入。莫错用意,名为真性自用。一真一切真。心量大事,不行小道。口莫终日说空,心中不修此行。恰似凡人,自称国王,终不可得,非吾弟子。

“善知识!何名般若?般若者,唐言智慧也。一切处所,一切时中,念念不愚,常行智慧,即是般若行。一念愚即般若绝,一念智即般若生。世人愚迷,不见般若。口说般若,心中常愚。常自言:‘我修般若。’念念说空,不识真空。般若无形相,智慧心即是。若作如是解,即名般若智。

“何名波罗蜜?此是西国语,唐言到彼岸,解义离生灭。著境生灭起,如水有波浪,即名为此岸;离境无生灭,如水常通流,即名为彼岸,故号波罗蜜。

“善知识!迷人口念,当念之时,有妄有非。念念若行,是名真性。悟此法者,是般若法;修此行者,是般若行。不修即凡;一念修行,自身等佛。

“善知识!凡夫即佛,烦恼即菩提。前念迷即凡夫,后念悟即佛。前念著境即烦恼,后念离境即菩提。

“善知识!摩诃般若波罗蜜,最尊最上最第一,无住无往亦无来,三世诸佛从中出。当用大智慧,打破五蕴烦恼尘劳。如此修行,定成佛道,变三毒为戒定慧。

“善知识!我此法门,从一般若生八万四千智慧。何以故?为世人有八万四千尘劳。若无尘劳,智慧常现,不离自性。悟此法者,即是无念、无忆、无著,不起诳妄。用自真如性,以智慧观照,于一切法,不取不舍,即是见性成佛道。

“善知识!若欲入甚深法界及般若三昧者,须修般若行,持诵《金刚般若经》,即得见性。当知此经功德,无量无边,经中分明赞叹,莫能具说。此法门是最上乘,为大智人说,为上根人说。小根小智人闻,心生不信。何以故?譬如天龙下雨于阎浮提,城邑聚落,悉皆漂流,如漂草叶。若雨大海,不增不减。若大乘人,若最上乘人,闻说《金刚经》,心开悟解。故知本性自有般若之智,自用智慧常观照故,不假文字。譬如雨水,不从天有,元是龙能兴致,令一切众生、一切草木、有情无情,悉皆蒙润。百川众流,却入大海,合为一体。众生本性般若之智,亦复如是。

“善知识!小根之人,闻此顿教,犹如草木,根性小者,若被大雨,悉皆自倒,不能增长。小根之人,亦复如是。元有般若之智,与大智人更无差别,因何闻法不自开悟?缘邪见障重、烦恼根深。犹如大云,覆盖于日,不得风吹,日光不现。般若之智,亦无大小,为一切众生,自心迷悟不同。迷心外见,修行觅佛,未悟自性,即是小根。若开悟顿教,不执外修,但于自心,常起正见,烦恼尘劳,常不能染,即是见性。

“善知识!内外不住,去来自由,能除执心,通达无碍。能修此行,与《般若经》本无差别。

“善知识!一切修多罗及诸文字,大小二乘,十二部经,皆因人置。因智慧性,方能建立。若无世人,一切万法本自不有。故知万法本自人兴,一切经书,因人说有。缘其人中有愚有智,愚为小人,智为大人。愚者问于智人,智者与愚人说法,愚人忽然悟解心开,即与智人无别。

“善知识!不悟,即佛是众生;一念悟时,众生是佛。故知万法尽在自心,何不从自心中,顿见真如本性?《菩萨戒经》云:‘我本元自性清净,若识自心见性,皆成佛道。’《净名经》云:‘即时豁然,还得本心。’

“善知识!我于忍和尚处,一闻言下便悟,顿见真如本性。是以将此教法流行,令学道者顿悟菩提,各自观心,自见本性。若自不悟,须觅大善知识,解最上乘法者,直示正路。是善知识,有大因缘,所谓化导令得见性,一切善法,因善知识能发起故。三世诸佛、十二部经,在人性中本自具有,不能自悟,须求善知识指示方见。若自悟者,不假外求。若一向执谓须他善知识方得解脱者,无有是处。何以故?自心内有知识自悟。若起邪迷、妄念颠倒,外善知识虽有教授,救不可得;若起真正般若观照,一刹那间,妄念俱灭;若识自性,一悟即至佛地。

“善知识!智慧观照,内外明彻,识自本心。若识本心,即本解脱。若得解脱,即是般若三昧。般若三昧,即是无念。何名无念?若见一切法,心不染著,是为无念。用即遍一切处,亦不著一切处。但净本心,使六识出六门,于六尘中无染无杂,来去自由,通用无滞,即是般若三昧,自在解脱,名无念行。若百物不思,当令念绝,即是法缚,即名边见。

“善知识!悟无念法者,万法尽通;悟无念法者,见诸佛境界;悟无念法者,至佛地位。

“善知识!后代得吾法者,将此顿教法门,于同见同行,发愿受持,如事佛故,终身而不退者,定入圣位。然须传授,从上以来,默传分付,不得匿其正法。若不同见同行,在别法中,不得传付,损彼前人,究竟无益。恐愚人不解,谤此法门,百劫千生,断佛种性。

“善知识!吾有一无相颂,各须诵取。在家出家,但依此修;若不自修,惟记吾言,亦无有益。听吾颂曰:

\begin{jizi}{“}
    说通及心通, 如日处虚空, \\
    唯传见性法, 出世破邪宗。 \\
    法即无顿渐, 迷悟有迟疾, \\
    只此见性门, 愚人不可悉。 \\
    说即虽万般, 合理还归一, \\
    烦恼暗宅中, 常须生慧日。 \\
    邪来烦恼至, 正来烦恼除, \\
    邪正俱不用, 清净至无余。 \\
    菩提本自性, 起心即是妄, \\
    净心在妄中, 但正无三障。 \\
    世人若修道, 一切尽不妨, \\
    常自见己过, 与道即相当。 \\
    色类自有道, 各不相妨恼, \\
    离道别觅道, 终身不见道。 \\
    波波度一生, 到头还自懊, \\
    欲得见真道, 行正即是道。 \\
    自若无道心, 暗行不见道, \\
    若真修道人, 不见世间过。 \\
    若见他人非, 自非却是左, \\
    他非我不非, 我非自有过。 \\
    但自却非心, 打除烦恼破, \\
    憎爱不关心, 长伸两脚卧。 \\
    欲拟化他人, 自须有方便, \\
    勿令彼有疑, 即是自性现。 \\
    佛法在世间, 不离世间觉, \\
    离世觅菩提, 恰如求兔角。 \\
    正见名出世, 邪见名世间, %\\
\end{jizi}
\begin{jizi}{}
    邪正尽打却, 菩提性宛然。 \\
    此颂是顿教, 亦名大法船, \\
    迷闻经累劫, 悟则刹那间。”
\end{jizi}

师复曰:“今于大梵寺说此顿教,普愿法界众生,言下见性成佛。”

时,韦使君与官僚道俗,闻师所说,无不省悟。一时作礼,皆叹:“善哉!何期岭南有佛出世!”



\section*{决疑品第三}

一日,韦刺史为师设大会斋。斋讫,刺史请师升座,同官僚士庶,肃容再拜,问曰:“弟子闻和尚说法,实不可思议。今有少疑,愿大慈悲,特为解说。”

师曰:“有疑即问,吾当为说。”

韦公曰:“和尚所说,可不是达摩大师宗旨乎?”

师曰:“是。”

公曰:“弟子闻:达摩初化梁武帝,帝问云:‘朕一生造寺度僧、布施设斋,有何功德?’达摩言:‘实无功德。’弟子未达此理,愿和尚为说。”

师曰:“实无功德,勿疑先圣之言。武帝心邪,不知正法。造寺度僧、布施设斋,名为求福,不可将福便为功德。功德在法身中,不在修福。”师又曰:“见性是功,平等是德,念念无滞,常见本性,真实妙用,名为功德。内心谦下是功,外行于礼是德;自性建立万法是功,心体离念是德;不离自性是功,应用无染是德。若觅功德法身,但依此作,是真功德。若修功德之人,心即不轻,常行普敬。心常轻人,吾我不断,即自无功;自性虚妄不实,即自无德。为吾我自大,常轻一切故。善知识!念念无间是功,心行平直是德。自修性是功,自修身是德。善知识!功德须自性内见,不是布施供养之所求也。是以福德与功德别。武帝不识其理,非我祖师有过。”

刺史又问曰:“弟子常见僧俗念阿弥陀佛,愿生西方。请和尚说,得生彼否?愿为破疑。”

师言:“使君善听,惠能与说。世尊在舍卫城中,说西方引化。经文分明,去此不远。若论相说,里数有十万八千,即身中十恶八邪,便是说远。说远为其下根,说近为其上智。人有两种,法无两般。迷悟有殊,见有迟疾。迷人念佛求生于彼,悟人自净其心。所以佛言:‘随其心净即佛土净。’使君,东方人,但心净即无罪;虽西方人,心不净亦有愆。东方人造罪,念佛求生西方;西方人造罪,念佛求生何国?凡愚不了自性,不识身中净土,愿东愿西,悟人在处一般。所以佛言:‘随所住处恒安乐。’使君,心地但无不善,西方去此不遥;若怀不善之心,念佛往生难到。今劝善知识,先除十恶,即行十万;后除八邪,乃过八千。念念见性,常行平直,到如弹指,便睹弥陀。使君,但行十善,何须更愿往生?不断十恶之心,何佛即来迎请?若悟无生顿法,见西方只在刹那;不悟念佛求生,路遥如何得达?惠能与诸人,移西方于刹那间,目前便见。各愿见否?”

众皆顶礼云:“若此处见,何须更愿往生?愿和尚慈悲,便现西方,普令得见。”

师言:“大众!世人自色身是城,眼耳鼻舌是门,外有五门,内有意门。心是地,性是王。王居心地上,性在王在,性去王无。性在身心存,性去身坏。佛向性中作,莫向身外求。自性迷即是众生,自性觉即是佛。慈悲即是观音,喜舍名为势至,能净即释迦,平直即弥陀。人我是须弥,贪欲是海水,烦恼是波浪,毒害是恶龙,虚妄是鬼神,尘劳是鱼鳖,贪瞋是地狱,愚痴是畜生。善知识!常行十善,天堂便至。除人我,须弥倒;去贪欲,海水竭;烦恼无,波浪灭;毒害除,鱼龙绝。自心地上,觉性如来,放大光明,外照六门清净,能破六欲诸天;自性内照,三毒即除;地狱等罪,一时销灭。内外明彻,不异西方。不作此修,如何到彼?”

大众闻说,了然见性,悉皆礼拜,俱叹善哉!唱言:“普愿法界众生,闻者一时悟解。”

师言:“善知识!若欲修行,在家亦得,不由在寺。在家能行,如东方人心善;在寺不修,如西方人心恶。但心清净,即是自性西方。”

韦公又问:“在家如何修行?愿为教授。”

师言:“吾与大众说无相颂。但依此修,常与吾同处无别;若不依此修,剃发出家,于道何益?颂曰:

\begin{jizi}{“}
    心平何劳持戒, 行直何用修禅。 \\
    恩则孝养父母, 义则上下相怜。 \\
    让则尊卑和睦, 忍则众恶无喧。 \\
    若能钻木出火, 淤泥定生红莲。 \\
    苦口的是良药, 逆耳必是忠言。 \\
    改过必生智慧, 护短心内非贤。 \\
    日用常行饶益, 成道非由施钱。 \\
    菩提只向心觅, 何劳向外求玄。 \\
    听说依此修行, 西方只在目前。”
\end{jizi}

师复曰:“善知识!总须依偈修行,见取自性,直成佛道。时不相待,众人且散,吾归曹溪。众若有疑,却来相问。”

时,刺史官僚、在会善男信女,各得开悟,信受奉行。



\section*{定慧品第四}

师示众云:“善知识!我此法门,以定慧为本。大众勿迷,言定慧别。定慧一体,不是二。定是慧体,慧是定用。即慧之时定在慧,即定之时慧在定。若识此义,即是定慧等学。诸学道人,莫言先定发慧,先慧发定,定慧各别。作此见者,法有二相,口说善语,心中不善;空有定慧,定慧不等。若心口俱善、内外一如,定慧即等。自悟修行,不在于诤。若诤先后,即同迷人,不断胜负,却增我法,不离四相。善知识!定慧犹如何等?犹如灯光,有灯即光,无灯即暗,灯是光之体,光是灯之用;名虽有二,体本同一。此定慧法,亦复如是。”

师示众云:“善知识!一行三昧者,于一切处行住坐卧,常行一直心是也。《净名经》云:‘直心是道场,直心是净土。’莫心行谄曲,口但说直;口说一行三昧,不行直心。但行直心,于一切法勿有执著。迷人著法相、执一行三昧,直言:‘常坐不动,妄不起心,即是一行三昧。’作此解者,即同无情,却是障道因缘。善知识!道须通流,何以却滞?心不住法,道即通流;心若住法,名为自缚。若言常坐不动是,只如舍利弗宴坐林中,却被维摩诘诃。善知识!又有人教坐,看心观静,不动不起,从此置功。迷人不会,便执成颠,如此者众。如是相教,故知大错。”

师示众云:“善知识!本来正教,无有顿渐,人性自有利钝。迷人渐修,悟人顿契。自识本心,自见本性,即无差别,所以立顿渐之假名。

“善知识!我此法门,从上以来,先立无念为宗,无相为体,无住为本。无相者,于相而离相。无念者,于念而无念。无住者,人之本性,于世间善恶好丑,乃至冤之与亲,言语触刺欺争之时,并将为空,不思酬害。念念之中,不思前境。若前念今念后念,念念相续不断,名为系缚;于诸法上,念念不住,即无缚也。此是以无住为本。善知识!外离一切相,名为无相。能离于相,即法体清净。此是以无相为体。善知识!于诸境上,心不染,曰无念。于自念上,常离诸境,不于境上生心。若只百物不思,念尽除却,一念绝即死,别处受生,是为大错。学道者思之。若不识法意,自错犹可,更误他人;自迷不见,又谤佛经。所以立无念为宗。

“善知识!云何立无念为宗?只缘口说见性,迷人于境上有念,念上便起邪见,一切尘劳妄想从此而生。自性本无一法可得,若有所得,妄说祸福,即是尘劳邪见。故此法门立无念为宗。善知识!无者无何事?念者念何物?无者无二相,无诸尘劳之心。念者念真如本性。真如即是念之体,念即是真如之用。真如自性起念,非眼耳鼻舌能念。真如有性,所以起念。真如若无,眼耳色声当时即坏。善知识!真如自性起念,六根虽有见闻觉知,不染万境,而真性常自在。故经云:‘能善分别诸法相,于第一义而不动。’”



\section*{妙行品第五}

师示众云:“此门坐禅,元不著心,亦不著净,亦不是不动。若言著心,心元是妄,知心如幻,故无所著也。若言著净,人性本净,由妄念故,盖覆真如,但无妄想,性自清净。起心著净,却生净妄,妄无处所,著者是妄。净无形相,却立净相,言是工夫;作此见者,障自本性,却被净缚。善知识!若修不动者,但见一切人时,不见人之是非善恶过患,即是自性不动。善知识!迷人身虽不动,开口便说他人是非长短好恶,与道违背。若著心著净,即障道也。”

师示众云:“善知识!何名坐禅?此法门中,无障无碍,外于一切善恶境界,心念不起,名为坐;内见自性不动,名为禅。善知识!何名禅定?外离相为禅,内不乱为定。外若著相,内心即乱;外若离相,心即不乱。本性自净自定,只为见境思境即乱;若见诸境心不乱者,是真定也。善知识!外离相即禅,内不乱即定,外禅内定,是为禅定。《菩萨戒经》云:‘我本元自性清净。’善知识!于念念中,自见本性清净,自修自行,自成佛道。”



\section*{忏悔品第六}

时,大师见广韶洎四方士庶,骈集山中听法,于是升座,告众曰:“来!诸善知识,此事须从自性中起,于一切时,念念自净其心,自修自行,见自己法身,见自心佛,自度自戒始得,不假到此。既从远来,一会于此,皆共有缘。今可各各胡跪,先为传自性五分法身香,次授无相忏悔。”众胡跪。

师曰:“一、戒香:即自心中无非、无恶、无嫉妒、无贪瞋、无劫害,名戒香。

“二、定香:即睹诸善恶境相,自心不乱,名定香。

“三、慧香:自心无碍,常以智慧观照自性,不造诸恶;虽修众善,心不执著;敬上念下,矜恤孤贫,名慧香。

“四、解脱香:即自心无所攀缘,不思善、不思恶,自在无碍,名解脱香。

“五、解脱知见香:自心既无所攀缘善恶,不可沉空守寂,即须广学多闻,识自本心,达诸佛理,和光接物,无我无人,直至菩提,真性不易,名解脱知见香。

“善知识!此香各自内熏,莫向外觅。今与汝等授无相忏悔,灭三世罪,令得三业清净。善知识!各随我语,一时道:‘弟子等,从前念今念及后念,念念不被愚迷染,从前所有恶业愚迷等罪,悉皆忏悔,愿一时销灭,永不复起。弟子等,从前念今念及后念,念念不被憍诳染,从前所有恶业憍诳等罪,悉皆忏悔,愿一时销灭,永不复起。弟子等,从前念今念及后念,念念不被嫉妒染,从前所有恶业嫉妒等罪,悉皆忏悔,愿一时销灭,永不复起。’

“善知识!已上是为无相忏悔。云何名忏?云何名悔?忏者,忏其前愆,从前所有恶业,愚迷憍诳嫉妒等罪,悉皆尽忏,永不复起,是名为忏。悔者,悔其后过,从今以后,所有恶业,愚迷憍诳嫉妒等罪,今已觉悟,悉皆永断,更不复作,是名为悔。故称忏悔。凡夫愚迷,只知忏其前愆,不知悔其后过。以不悔故,前愆不灭,后过又生。前愆既不灭,后过复又生,何名忏悔?

“善知识!既忏悔已,与善知识发四弘誓愿,各须用心正听:自心众生无边誓愿度,自心烦恼无边誓愿断,自性法门无尽誓愿学,自性无上佛道誓愿成。善知识!大家岂不道,众生无边誓愿度。恁么道,且不是惠能度。善知识!心中众生,所谓邪迷心、诳妄心、不善心、嫉妒心、恶毒心,如是等心,尽是众生,各须自性自度,是名真度。何名自性自度?即自心中邪见烦恼愚痴众生,将正见度。既有正见,使般若智打破愚痴迷妄众生,各各自度。邪来正度,迷来悟度,愚来智度,恶来善度;如是度者,名为真度。又,烦恼无边誓愿断,将自性般若智,除却虚妄思想心是也。又,法门无尽誓愿学,须自见性,常行正法,是名真学。又,无上佛道誓愿成,既常能下心,行于真正,离迷离觉,常生般若,除真除妄,即见佛性,即言下佛道成。常念修行,是愿力法。

“善知识!今发四弘愿了,更与善知识授无相三归依戒。善知识!归依觉,两足尊;归依正,离欲尊;归依净,众中尊。从今日去,称觉为师,更不归依邪魔外道,以自性三宝常自证明。劝善知识,归依自性三宝。佛者,觉也;法者,正也;僧者,净也。自心归依觉,邪迷不生,少欲知足,能离财色,名两足尊。自心归依正,念念无邪见,以无邪见故,即无人我贡高,贪爱执著,名离欲尊。自心归依净,一切尘劳爱欲境界,自性皆不染著,名众中尊。

“若修此行,是自归依。凡夫不会,从日至夜,受三归戒。若言归依佛,佛在何处?若不见佛,凭何所归?言却成妄。善知识!各自观察,莫错用心。经文分明言自归依佛,不言归依他佛。自佛不归,无所依处。今既自悟,各须归依自心三宝。内调心性,外敬他人,是自归依也。

“善知识!既归依自三宝竟,各各志心,吾与说一体三身自性佛,令汝等见三身,了然自悟自性。总随我道:‘于自色身,归依清净法身佛;于自色身,归依圆满报身佛;于自色身,归依千百亿化身佛。’善知识!色身是舍宅,不可言归。向者三身佛,在自性中,世人总有;为自心迷,不见内性,外觅三身如来,不见自身中有三身佛。汝等听说,令汝等于自身中,见自性有三身佛。此三身佛,从自性生,不从外得。

“何名清净法身佛?世人性本清净,万法从自性生。思量一切恶事,即生恶行;思量一切善事,即生善行。如是诸法,在自性中,如天常清,日月常明,为浮云盖覆,上明下暗,忽遇风吹云散,上下俱明,万象皆现。世人性常浮游,如彼天云。善知识!智如日,慧如月,智慧常明,于外著境,被妄念浮云盖覆自性,不得明朗。若遇善知识,闻真正法,自除迷妄,内外明彻,于自性中万法皆现。见性之人,亦复如是。此名清净法身佛。

“善知识!自心归依自性,是归依真佛。自归依者,除却自性中不善心、嫉妒心、谄曲心、吾我心、诳妄心、轻人心、慢他心、邪见心、贡高心,及一切时中不善之行,常自见己过,不说他人好恶,是自归依。常须下心,普行恭敬,即是见性通达,更无滞碍,是自归依。

“何名圆满报身?譬如一灯能除千年暗,一智能灭万年愚。莫思向前,已过不可得;常思于后,念念圆明,自见本性。善恶虽殊,本性无二;无二之性,名为实性。于实性中,不染善恶,此名圆满报身佛。自性起一念恶,灭万劫善因;自性起一念善,得恒沙恶尽。直至无上菩提,念念自见,不失本念,名为报身。

“何名千百亿化身?若不思万法,性本如空;一念思量,名为变化。思量恶事,化为地狱;思量善事,化为天堂。毒害化为龙蛇,慈悲化为菩萨,智慧化为上界,愚痴化为下方。自性变化甚多,迷人不能省觉,念念起恶,常行恶道。回一念善,智慧即生,此名自性化身佛。

“善知识!法身本具,念念自性自见,即是报身佛。从报身思量,即是化身佛。自悟自修,自性功德,是真归依。皮肉是色身,色身是宅舍,不言归依也。但悟自性三身,即识自性佛。吾有一无相颂,若能诵持,言下令汝积劫迷罪,一时销灭。颂曰:

\begin{jizi}{“}
    迷人修福不修道, 只言修福便是道, \\
    布施供养福无边, 心中三恶元来造。 \\
    拟将修福欲灭罪, 后世得福罪还在, \\
    但向心中除罪缘, 各自性中真忏悔。 \\
    忽悟大乘真忏悔, 除邪行正即无罪, \\
    学道常于自性观, 即与诸佛同一类。 \\
    吾祖唯传此顿法, 普愿见性同一体, \\
    若欲当来觅法身, 离诸法相心中洗。 \\
    努力自见莫悠悠, 后念忽绝一世休, \\
    若悟大乘得见性, 虔恭合掌至心求。”
\end{jizi}

师言:“善知识!总须诵取,依此修行,言下见性,虽去吾千里,如常在吾边;于此言下不悟,即对面千里,何勤远来。珍重!好去。”

一众闻法,靡不开悟,欢喜奉行。



\section*{机缘品第七}

师自黄梅得法,回至韶州曹侯村,人无知者。有儒士刘志略,礼遇甚厚。志略有姑为尼,名无尽藏,常诵《大涅槃经》。师暂听,即知妙义,遂为解说。尼乃执卷问字,师曰:“字即不识,义即请问。”尼曰:“字尚不识,焉能会义?”师曰:“诸佛妙理,非关文字。”尼惊异之,遍告里中耆德云:“此是有道之士,宜请供养。”有魏(魏一作晋)武侯玄孙曹叔良及居民,竞来瞻礼。时,宝林古寺,自隋末兵火已废,遂于故基重建梵宇,延师居之。俄成宝坊。

师住九月余日,又为恶党寻逐。师乃遁于前山,被其纵火焚草木,师隐身挨入石中得免。石今有师趺坐膝痕及衣布之纹,因名避难石。师忆五祖怀会止藏之嘱,遂行隐于二邑焉。

僧法海,韶州曲江人也。初参祖师,问曰:“即心即佛,愿垂指谕。”师曰:“前念不生即心,后念不灭即佛;成一切相即心,离一切相即佛。吾若具说,穷劫不尽。听吾偈曰:

\begin{jizi}{“}
    即心名慧, 即佛乃定, \\
    定慧等持, 意中清净。 \\
    悟此法门, 由汝习性, \\
    用本无生, 双修是正。”
\end{jizi}

法海言下大悟,以偈赞曰:

\begin{jizi}{“}
    即心元是佛, 不悟而自屈; \\
    我知定慧因, 双修离诸物。”
\end{jizi}

僧法达,洪州人,七岁出家,常诵《法华经》。来礼祖师,头不至地。师诃曰:“礼不投地,何如不礼?汝心中必有一物,蕴习何事耶?”曰:“念《法华经》已及三千部。”师曰:“汝若念至万部,得其经意,不以为胜,则与吾偕行。汝今负此事业,都不知过。听吾偈曰:

\begin{jizi}{“}
    礼本折慢幢, 头奚不至地? \\
    有我罪即生, 忘功福无比。”
\end{jizi}

师又曰:“汝名什么?”曰:“法达。”师曰:“汝名法达,何曾达法?”复说偈曰:

\begin{jizi}{“}
    汝今名法达, 勤诵未休歇, \\
    空诵但循声, 明心号菩萨。 \\
    汝今有缘故, 吾今为汝说, \\
    但信佛无言, 莲花从口发。”
\end{jizi}

达闻偈,悔谢曰:“而今而后,当谦恭一切。弟子诵《法华经》,未解经义,心常有疑。和尚智慧广大,愿略说经中义理。”师曰:“法达!法即甚达,汝心不达;经本无疑,汝心自疑。汝念此经,以何为宗?”达曰:“学人根性暗钝,从来但依文诵念,岂知宗趣?”师曰:“吾不识文字,汝试取经诵一遍,吾当为汝解说。”

法达即高声念经,至《譬喻品》,师曰:“止!此经元来以因缘出世为宗,纵说多种譬喻,亦无越于此。何者因缘?经云:‘诸佛世尊,唯以一大事因缘故出现于世。’一大事者,佛之知见也。世人外迷著相,内迷著空;若能于相离相、于空离空,即是内外不迷。若悟此法,一念心开,是为开佛知见。佛,犹觉也。分为四门:开觉知见、示觉知见、悟觉知见、入觉知见。若闻开示,便能悟入,即觉知见,本来真性而得出现。汝慎勿错解经意,见他道开示悟入,自是佛之知见,我辈无分。若作此解,乃是谤经毁佛也。彼既是佛,已具知见,何用更开?汝今当信,佛知见者,只汝自心,更无别佛。盖为一切众生,自蔽光明,贪爱尘境,外缘内扰,甘受驱驰。便劳他世尊,从三昧起,种种苦口,劝令寝息,莫向外求,与佛无二,故云:‘开佛知见。’吾亦劝一切人,于自心中,常开佛之知见。世人心邪,愚迷造罪,口善心恶,贪瞋嫉妒,谄佞我慢,侵人害物,自开众生知见。若能正心,常生智慧,观照自心,止恶行善,是自开佛之知见。汝须念念开佛知见,勿开众生知见。开佛知见,即是出世;开众生知见,即是世间。汝若但劳劳执念,以为功课者,何异犛牛爱尾?”

达曰:“若然者,但得解义,不劳诵经耶?”师曰:“经有何过,岂障汝念?只为迷悟在人,损益由己。口诵心行,即是转经;口诵心不行,即是被经转。听吾偈曰:

\begin{jizi}{“}
    心迷法华转, 心悟转法华; \\
    诵经久不明, 与义作仇家。 \\
    无念念即正, 有念念成邪; \\
    有无俱不计, 长御白牛车。”
\end{jizi}

达闻偈,不觉悲泣,言下大悟,而告师曰:“法达从昔已来,实未曾转法华,乃被法华转。”再启曰:“经云:‘诸大声闻乃至菩萨,皆尽思共度量,不能测佛智。’今令凡夫但悟自心,便名佛之知见,自非上根,未免疑谤。又经说三车,羊鹿牛车与白牛之车,如何区别?愿和尚再垂开示。”

师曰:“经意分明,汝自迷背。诸三乘人,不能测佛智者,患在度量也。饶伊尽思共推,转加悬远。佛本为凡夫说,不为佛说。此理若不肯信者,从他退席,殊不知坐却白牛车,更于门外觅三车。况经文明向汝道:‘唯一佛乘,无有余乘。’若二若三,乃至无数方便,种种因缘,譬喻言词,是法皆为一佛乘故。汝何不省,三车是假,为昔时故;一乘是实,为今时故。只教汝去假归实,归实之后,实亦无名。应知所有珍财,尽属于汝,由汝受用,更不作父想,亦不作子想,亦无用想,是名持《法华经》。从劫至劫,手不释卷,从昼至夜,无不念时也。”

达蒙启发,踊跃欢喜,以偈赞曰:

\begin{jizi}{“}
    经诵三千部, 曹溪一句亡, \\
    未明出世旨, 宁歇累生狂。 \\
    羊鹿牛权设, 初中后善扬, \\
    谁知火宅内, 元是法中王。”
\end{jizi}

师曰:“汝今后方可名念经僧也。”达从此领玄旨,亦不辍诵经。

僧智通,寿州安丰人,初看《楞伽经》约千余遍,而不会三身四智,礼师求解其义。师曰:“三身者,清净法身,汝之性也;圆满报身,汝之智也;千百亿化身,汝之行也。若离本性,别说三身,即名有身无智;若悟三身无有自性,即明四智菩提。听吾偈曰:

\begin{jizi}{“}
    自性具三身, 发明成四智, \\
    不离见闻缘, 超然登佛地。 \\
    吾今为汝说, 谛信永无迷, \\
    莫学驰求者, 终日说菩提。”
\end{jizi}

通再启曰:“四智之义,可得闻乎?”师曰:“既会三身,便明四智,何更问耶?若离三身,别谈四智,此名有智无身。即此有智,还成无智。”复说偈曰:

\begin{jizi}{“}
    大圆镜智性清净, 平等性智心无病, \\
    妙观察智见非功, 成所作智同圆镜。 \\
    五八六七果因转, 但用名言无实性, \\
    若于转处不留情, 繁兴永处那伽定。”
\end{jizi}

(如上转识为智也。教中云:转前五识为成所作智,转第六识为妙观察智,转第七识为平等性智,转第八识为大圆镜智。虽六七因中转,五八果上转,但转其名而不转其体也。)

通顿悟性智,遂呈偈曰:

\begin{jizi}{“}
    三身元我体, 四智本心明, \\
    身智融无碍, 应物任随形。 \\
    起修皆妄动, 守住匪真精, \\
    妙旨因师晓, 终亡染污名。”
\end{jizi}

僧智常,信州贵溪人,髫年出家,志求见性。一日参礼,师问曰:“汝从何来?欲求何事?”曰:“学人近往洪州白峰山,礼大通和尚,蒙示见性成佛之义,未决狐疑,远来投礼,伏望和尚慈悲指示。”师曰:“彼有何言句?汝试举看。”曰:“智常到彼,凡经三月,未蒙示诲。为法切故,一夕独入丈室,请问:‘如何是某甲本心本性?’大通乃曰:‘汝见虚空否?’对曰:‘见。’彼曰:‘汝见虚空有相貌否?’对曰:‘虚空无形,有何相貌?’彼曰:‘汝之本性,犹如虚空,了无一物可见,是名正见;无一物可知,是名真知。无有青黄长短,但见本源清净,觉体圆明,即名见性成佛,亦名如来知见。’学人虽闻此说,犹未决了,乞和尚开示。”师曰:“彼师所说,犹存见知,故令汝未了。吾今示汝一偈:

\begin{jizi}{“}
    不见一法存无见, 大似浮云遮日面, \\
    不知一法守空知, 还如太虚生闪电。 \\
    此之知见瞥然兴, 错认何曾解方便, \\
    汝当一念自知非, 自己灵光常显现。”
\end{jizi}

常闻偈已,心意豁然。乃述偈曰:

\begin{jizi}{“}
    无端起知见, 著相求菩提, \\
    情存一念悟, 宁越昔时迷。 \\
    自性觉源体, 随照枉迁流, \\
    不入祖师室, 茫然趣两头。”
\end{jizi}

智常一日问师曰:“佛说三乘法,又言最上乘。弟子未解,愿为教授。”师曰:“汝观自本心,莫著外法相。法无四乘,人心自有等差。见闻转诵是小乘;悟法解义是中乘;依法修行是大乘;万法尽通,万法俱备,一切不染,离诸法相,一无所得,名最上乘。乘是行义,不在口争。汝须自修,莫问吾也。一切时中,自性自如。”常礼谢,执侍终师之世。

僧志道,广州南海人也。请益曰:“学人自出家,览《涅槃经》十载有余,未明大意,愿和尚垂诲。”师曰:“汝何处未明?”曰:“诸行无常,是生灭法;生灭灭已,寂灭为乐。于此疑惑。”师曰:“汝作么生疑?”曰:“一切众生皆有二身,谓色身、法身也。色身无常,有生有灭;法身有常,无知无觉。经云:‘生灭灭已,寂灭为乐’者,不审何身寂灭?何身受乐?若色身者,色身灭时,四大分散,全然是苦,苦不可言乐;若法身寂灭,即同草木瓦石,谁当受乐?又,法性是生灭之体,五蕴是生灭之用。一体五用,生灭是常。生则从体起用,灭则摄用归体。若听更生,即有情之类,不断不灭;若不听更生,则永归寂灭,同于无情之物。如是,则一切诸法被涅槃之所禁伏,尚不得生,何乐之有?”

师曰:“汝是释子,何习外道断常邪见,而议最上乘法?据汝所说,即色身外别有法身,离生灭求于寂灭。又推涅槃常乐,言有身受用,斯乃执吝生死,耽著世乐。汝今当知,佛为一切迷人,认五蕴和合为自体相,分别一切法为外尘相。好生恶死,念念迁流,不知梦幻虚假,枉受轮回。以常乐涅槃,翻为苦相,终日驰求。佛愍此故,乃示涅槃真乐,刹那无有生相,刹那无有灭相,更无生灭可灭,是则寂灭现前。当现前时,亦无现前之量,乃谓常乐。此乐无有受者,亦无不受者,岂有一体五用之名?何况更言涅槃禁伏诸法,令永不生?斯乃谤佛毁法。听吾偈曰:

\begin{jizi}{“}
    无上大涅槃, 圆明常寂照, \\
    凡愚谓之死, 外道执为断。 \\
    诸求二乘人, 目以为无作, \\
    尽属情所计, 六十二见本。 \\
    妄立虚假名, 何为真实义? \\
    惟有过量人, 通达无取舍。 \\
    以知五蕴法, 及以蕴中我, \\
    外现众色像, 一一音声相, \\
    平等如梦幻, 不起凡圣见, \\
    不作涅槃解, 二边三际断。 \\
    常应诸根用, 而不起用想, \\
    分别一切法, 不起分别想。 \\
    劫火烧海底, 风鼓山相击, \\
    真常寂灭乐, 涅槃相如是。 \\
    吾今强言说, 令汝舍邪见, \\
    汝勿随言解, 许汝知少分。”
\end{jizi}

志道闻偈大悟,踊跃作礼而退。

行思禅师,生吉州安城刘氏。闻曹溪法席盛化,径来参礼,遂问曰:“当何所务,即不落阶级?”师曰:“汝曾作什么来?”曰:“圣谛亦不为。”师曰:“落何阶级?”曰:“圣谛尚不为,何阶级之有?”师深器之,令思首众。一日,师谓曰:“汝当分化一方,无令断绝。”思既得法,遂回吉州青原山,弘法绍化(谥“弘济禅师”)。

怀让禅师,金州杜氏子也。初谒嵩山安国师,安发之曹溪参叩。让至礼拜,师曰:“甚处来?”曰:“嵩山。”师曰:“什么物?恁么来?”曰:“说似一物即不中。”师曰:“还可修证否?”曰:“修证即不无,污染即不得。”师曰:“只此不污染,诸佛之所护念。汝既如是,吾亦如是。西天般若多罗谶,汝足下出一马驹,踏杀天下人。应在汝心,不须速说(一本无西天以下二十七字)。”让豁然契会,遂执侍左右一十五载,日臻玄奥。后往南岳,大阐禅宗,敕谥“大慧禅师”。

永嘉玄觉禅师,温州戴氏子。少习经论,精天台止观法门,因看《维摩经》发明心地。偶师弟子玄策相访,与其剧谈,出言暗合诸祖。策云:“仁者得法师谁?”曰:“我听方等经论,各有师承。后于《维摩经》悟佛心宗,未有证明者。”策云:“威音王已前即得,威音王已后,无师自悟,尽是天然外道。”曰:“愿仁者为我证据。”策云:“我言轻。曹溪有六祖大师,四方云集,并是受法者。若去,则与偕行。”

觉遂同策来参,绕师三匝,振锡而立。师曰:“夫沙门者,具三千威仪、八万细行。大德自何方而来,生大我慢?”觉曰:“生死事大,无常迅速。”师曰:“何不体取无生,了无速乎?”曰:“体即无生,了本无速。”师曰:“如是,如是!”玄觉方具威仪礼拜,须臾告辞。师曰:“返太速乎?”曰:“本自非动,岂有速耶?”师曰:“谁知非动?”曰:“仁者自生分别。”师曰:“汝甚得无生之意。”曰:“无生岂有意耶?”师曰:“无意谁当分别?”曰:“分别亦非意。”师曰:“善哉!少留一宿。”时谓一宿觉。后著《证道歌》,盛行于世。谥曰“无相大师”,时称为“真觉”焉。

禅者智隍,初参五祖,自谓已得正受,庵居长坐,积二十年。师弟子玄策,游方至河朔,闻隍之名,造庵问云:“汝在此作什么?”隍曰:“入定。”策云:“汝云入定,为有心入耶?无心入耶?若无心入者,一切无情,草木瓦石,应合得定;若有心入者,一切有情,含识之流,亦应得定。”隍曰:“我正入定时,不见有有无之心。”策云:“不见有有无之心,即是常定。何有出入?若有出入,即非大定。”隍无对,良久,问曰:“师嗣谁耶?”策云:“我师曹溪六祖。”隍云:“六祖以何为禅定?”策云:“我师所说,妙湛圆寂,体用如如,五阴本空,六尘非有,不出不入,不定不乱。禅性无住,离住禅寂;禅性无生,离生禅想。心如虚空,亦无虚空之量。”

隍闻是说,径来谒师。师问云:“仁者何来?”隍具述前缘。师云:“诚如所言。汝但心如虚空,不著空见,应用无碍,动静无心,凡圣情忘,能所俱泯,性相如如,无不定时也。”隍于是大悟,二十年所得心,都无影响。其夜,河北士庶,闻空中有声云:“隍禅师今日得道。”隍后礼辞,复归河北,开化四众。

一僧问师云:“黄梅意旨,甚么人得?”师云:“会佛法人得。”僧云:“和尚还得否?”师云:“我不会佛法。”

师一日欲濯所授之衣,而无美泉。因至寺后五里许,见山林郁茂,瑞气盘旋,师振锡卓地,泉应手而出,积以为池,乃跪膝浣衣石上。忽有一僧来礼拜,云:“方辩是西蜀人,昨于南天竺国,见达摩大师,嘱方辩速往唐土:‘吾传大迦叶正法眼藏及僧伽梨,现传六代于韶州曹溪,汝去瞻礼’。方辩远来,愿见我师传来衣钵。”师乃出示,次问:“上人攻何事业?”曰:“善塑。”师正色曰:“汝试塑看。”辩罔措。过数日,塑就真相,可高七寸,曲尽其妙。师笑曰:“汝只解塑性,不解佛性。”师舒手摩方辩顶,曰:“永为人天福田。”(师仍以衣酬之。辩取衣分为三,一披塑像,一自留,一用棕裹瘗地中。誓曰:“后得此衣,乃吾出世,住持于此,重建殿宇。”宋嘉祐八年,有僧惟先,修殿掘地,得衣如新。像在高泉寺,祈祷辄应。)

有僧举卧轮禅师偈云:

\begin{jizi}{“}
    卧轮有伎俩, 能断百思想, \\
    对境心不起, 菩提日日长。”
\end{jizi}

师闻之,曰:“此偈未明心地,若依而行之,是加系缚。”因示一偈曰:

\begin{jizi}{“}
    惠能没伎俩, 不断百思想, \\
    对境心数起, 菩提作么长。”
\end{jizi}


\section*{顿渐品第八}

时,祖师居曹溪宝林,神秀大师在荆南玉泉寺。于时两宗盛化,人皆称南能北秀,故有南北二宗顿渐之分,而学者莫知宗趣。师谓众曰:“法本一宗,人有南北;法即一种,见有迟疾。何名顿渐?法无顿渐,人有利钝,故名顿渐。”

然秀之徒众,往往讥南宗祖师不识一字,有何所长?秀曰:“他得无师之智,深悟上乘,吾不如也。且吾师五祖亲传衣法,岂徒然哉!吾恨不能远去亲近,虚受国恩。汝等诸人,毋滞于此,可往曹溪参决。”一日,命门人志诚曰:“汝聪明多智,可为吾到曹溪听法。若有所闻,尽心记取,还为吾说。”志诚禀命至曹溪,随众参请,不言来处。时祖师告众曰:“今有盗法之人,潜在此会。”志诚即出礼拜,具陈其事。师曰:“汝从玉泉来,应是细作。”对曰:“不是。”师曰:“何得不是?”对曰:“未说即是,说了不是。”师曰:“汝师若为示众?”对曰:“常指诲大众,住心观净,长坐不卧。”师曰:“住心观净,是病非禅;长坐拘身,于理何益?听吾偈曰:

\begin{jizi}{“}
    生来坐不卧, 死去卧不坐, \\
    一具臭骨头, 何为立功课?”
\end{jizi}

志诚再拜曰:“弟子在秀大师处,学道九年,不得契悟。今闻和尚一说,便契本心。弟子生死事大,和尚大慈,更为教示。”师云:“吾闻汝师教示学人戒定慧法,未审汝师说戒定慧行相如何?与吾说看。”诚曰:“秀大师说,诸恶莫作名为戒,诸善奉行名为慧,自净其意名为定。彼说如此,未审和尚以何法诲人?”师曰:“吾若言有法与人,即为诳汝,但且随方解缚,假名三昧。如汝师所说戒定慧,实不可思议。吾所见戒定慧又别。”志诚曰:“戒定慧只合一种,如何更别?”师曰:“汝师戒定慧接大乘人,吾戒定慧接最上乘人。悟解不同,见有迟疾。汝听吾说,与彼同否?吾所说法,不离自性,离体说法,名为相说,自性常迷。须知一切万法,皆从自性起用,是真戒定慧法。听吾偈曰:

\begin{jizi}{“}
    心地无非自性戒, 心地无痴自性慧, \\
    心地无乱自性定, 不增不减自金刚, \\
    身去身来本三昧。”
\end{jizi}

诚闻偈,悔谢,乃呈一偈曰:

\begin{jizi}{“}
    五蕴幻身, 幻何究竟? \\
    回趣真如, 法还不净。”
\end{jizi}

师然之。复语诚曰:“汝师戒定慧,劝小根智人;吾戒定慧,劝大根智人。若悟自性,亦不立菩提涅槃,亦不立解脱知见。无一法可得,方能建立万法。若解此意,亦名佛身,亦名菩提涅槃,亦名解脱知见。见性之人,立亦得、不立亦得,去来自由,无滞无碍,应用随作,应语随答,普见化身,不离自性,即得自在神通,游戏三昧,是名见性。”志诚再启师曰:“如何是不立义?”师曰:“自性无非、无痴、无乱,念念般若观照,常离法相,自由自在,纵横尽得,有何可立?自性自悟,顿悟顿修,亦无渐次,所以不立一切法。诸法寂灭,有何次第?”

志诚礼拜,愿为执侍,朝夕不懈(诚,吉州太和人也)。

僧志彻,江西人,本姓张,名行昌,少任侠。自南北分化,二宗主虽亡彼我,而徒侣竞起爱憎。时北宗门人,自立秀师为第六祖,而忌祖师传衣为天下闻,乃嘱行昌来刺师。师心通,预知其事,即置金十两于座间。时,夜暮,行昌入祖室,将欲加害,师舒颈就之。行昌挥刃者三,悉无所损。师曰:“正剑不邪,邪剑不正。只负汝金,不负汝命。”行昌惊仆,久而方苏,求哀悔过,即愿出家。师遂与金,言:“汝且去,恐徒众翻害于汝。汝可他日易形而来,吾当摄受。”行昌禀旨宵遁,后投僧出家,具戒精进。

一日,忆师之言,远来礼觐。师曰:“吾久念汝,汝来何晚?”曰:“昨蒙和尚舍罪,今虽出家苦行,终难报德,其惟传法度生乎?弟子常览《涅槃经》,未晓常无常义,乞和尚慈悲,略为解说。”师曰:“无常者,即佛性也;有常者,即一切善恶诸法分别心也。”

曰:“和尚所说,大违经文。”师曰:“吾传佛心印,安敢违于佛经?”

曰:“经说佛性是常,和尚却言无常;善恶诸法乃至菩提心,皆是无常,和尚却言是常。此即相违,令学人转加疑惑。”师曰:“《涅槃经》,吾昔听尼无尽藏读诵一遍,便为讲说,无一字一义不合经文,乃至为汝,终无二说。”曰:“学人识量浅昧,愿和尚委曲开示。”

师曰:“汝知否?佛性若常,更说什么善恶诸法,乃至穷劫,无有一人发菩提心者,故吾说无常,正是佛说真常之道也。又,一切诸法若无常者,即物物皆有自性,容受生死,而真常性有不遍之处,故吾说常者,正是佛说真无常义。佛比为凡夫外道,执于邪常,诸二乘人于常计无常,共成八倒。故于《涅槃》了义教中,破彼偏见,而显说真常、真乐、真我、真净。汝今依言背义,以断灭无常及确定死常,而错解佛之圆妙最后微言,纵览千遍,有何所益?”行昌忽然大悟,说偈曰:

\begin{jizi}{“}
    因守无常心, 佛说有常性, \\
    不知方便者, 犹春池拾砾。 \\
    我今不施功, 佛性而现前, \\
    非师相授与, 我亦无所得。”
\end{jizi}

师曰:“汝今彻也,宜名志彻。”彻礼谢而退。

有一童子,名神会,襄阳高氏子,年十三,自玉泉来参礼。师曰:“知识远来艰辛,还将得本来否?若有本则合识主,试说看。”会曰:“以无住为本,见即是主。”师曰:“这沙弥争合取次语。”会乃问曰:“和尚坐禅,还见不见?”师以柱杖打三下,云:“吾打汝痛不痛?”对曰:“亦痛亦不痛。”师曰:“吾亦见亦不见。”神会问:“如何是亦见亦不见?”师云:“吾之所见,常见自心过愆,不见他人是非好恶,是以亦见亦不见。汝言:‘亦痛亦不痛。’如何?汝若不痛,同其木石;若痛,则同凡夫,即起恚恨。汝向前见不见是二边,痛不痛是生灭。汝自性且不见,敢尔弄人?”神会礼拜悔谢。师又曰:“汝若心迷不见,问善知识觅路。汝若心悟,即自见性,依法修行。汝自迷不见自心,却来问吾见与不见。吾见自知,岂待汝迷?汝若自见,亦不待吾迷。何不自知自见,乃问吾见与不见?”神会再礼百余拜,求谢过愆,服勤给侍,不离左右。

一日,师告众曰:“吾有一物,无头无尾,无名无字,无背无面。诸人还识否?”神会出曰:“是诸佛之本源,神会之佛性。”师曰:“向汝道:‘无名无字’,汝便唤作本源佛性。汝向去有把茆盖头,也只成个知解宗徒。”

祖师灭后,会入京洛,大弘曹溪顿教,著《显宗记》,盛行于世(是为荷泽禅师)。

师见诸宗难问,咸起恶心,多集座下,愍而谓曰:“学道之人,一切善念恶念,应当尽除;无名可名,名于自性;无二之性,是名实性。于实性上建立一切教门,言下便须自见。”诸人闻说,总皆作礼,请事为师。



\section*{护法品第九}

神龙元年上元日,则天、中宗诏云:“朕请安、秀二师宫中供养。万机之暇,每究一乘。二师推让云:‘南方有能禅师,密授忍大师衣法,传佛心印,可请彼问。’今遣内侍薛简,驰诏迎请,愿师慈念,速赴上京。”师上表辞疾,愿终林麓。薛简曰:“京城禅德皆云:‘欲得会道,必须坐禅习定。若不因禅定而得解脱者,未之有也。’未审师所说法如何?”

师曰:“道由心悟,岂在坐也?经云:‘若言如来若坐若卧,是行邪道。’何故?无所从来,亦无所去,无生无灭,是如来清净禅。诸法空寂,是如来清净坐。究竟无证,岂况坐耶。”简曰:“弟子回京,主上必问。愿师慈悲,指示心要,传奏两宫及京城学道者。譬如一灯燃百千灯,冥者皆明,明明无尽。”师云:“道无明暗,明暗是代谢之义;明明无尽,亦是有尽,相待立名。故《净名经》云:‘法无有比,无相待故。’”简曰:“明喻智慧,暗喻烦恼,修道之人,倘不以智慧照破烦恼,无始生死凭何出离?”师曰:“烦恼即是菩提,无二无别。若以智慧照破烦恼者,此是二乘见解,羊鹿等机。上智大根,悉不如是。”

简曰:“如何是大乘见解?”师曰:“明与无明,凡夫见二;智者了达,其性无二。无二之性,即是实性。实性者,处凡愚而不减,在贤圣而不增,住烦恼而不乱,居禅定而不寂。不断不常,不来不去,不在中间,及其内外,不生不灭,性相如如,常住不迁,名之曰道。”简曰:“师说不生不灭,何异外道?”师曰:“外道所说不生不灭者,将灭止生,以生显灭,灭犹不灭,生说不生。我说不生不灭者,本自无生,今亦不灭,所以不同外道。汝若欲知心要,但一切善恶都莫思量,自然得入清净心体,湛然常寂,妙用恒沙。”简蒙指教,豁然大悟,礼辞归阙,表奏师语。

其年九月三日,有诏奖谕师曰:“师辞老疾,为朕修道,国之福田。师若净名,托疾毗耶,阐扬大乘,传诸佛心,谈不二法。薛简传师指授如来知见,朕积善余庆,宿种善根,值师出世,顿悟上乘,感荷师恩,顶戴无已,并奉磨衲袈裟及水晶钵,敕韶州剌史修饰寺宇,赐师旧居为国恩寺焉。”



\section*{付嘱品第十}

师一日唤门人法海、志诚、法达、神会、智常、智通、志彻、志道、法珍、法如等,曰:“汝等不同余人,吾灭度后,各为一方师。吾今教汝说法,不失本宗:先须举三科法门,动用三十六对,出没即离两边,说一切法,莫离自性。忽有人问汝法,出语尽双,皆取对法,来去相因。究竟二法尽除,更无去处。

“三科法门者,阴、界、入也。阴是五阴:色、受、想、行、识是也。入是十二入,外六尘:色、声、香、味、触、法;内六门:眼、耳、鼻、舌、身、意是也。界是十八界:六尘、六门、六识是也。自性能含万法,名含藏识;若起思量,即是转识,生六识,出六门,见六尘,如是一十八界,皆从自性起用。自性若邪,起十八邪;自性若正,起十八正。若恶用即众生用,善用即佛用。

“用由何等?由自性有对法。外境无情五对:天与地对,日与月对,明与暗对,阴与阳对,水与火对,此是五对也。法相语言十二对:语与法对,有与无对,有色与无色对,有相与无相对,有漏与无漏对,色与空对,动与静对,清与浊对,凡与圣对,僧与俗对,老与少对,大与小对,此是十二对也。自性起用十九对:长与短对,邪与正对,痴与慧对,愚与智对,乱与定对,慈与毒对,戒与非对,直与曲对,实与虚对,险与平对,烦恼与菩提对,常与无常对,悲与害对,喜与瞋对,舍与悭对,进与退对,生与灭对,法身与色身对,化身与报身对,此是十九对也。”

师言:“此三十六对法,若解用,即道贯一切经法,出入即离两边。自性动用,共人言语,外于相离相,内于空离空。若全著相,即长邪见;若全执空,即长无明。执空之人有谤经,‘直言不用文字。’既云不用文字,人亦不合语言。只此语言,便是文字之相。又云:‘直道不立文字。’即此‘不立’两字,亦是文字。见人所说,便即谤他言著文字。汝等须知,自迷犹可,又谤佛经,不要谤经,罪障无数。若著相于外,而作法求真,或广立道场,说有无之过患,如是之人,累劫不得见性。但听依法修行,又莫百物不思,而于道性窒碍。若听说不修,令人反生邪念。但依法修行,无住相法施。汝等若悟,依此说、依此用、依此行、依此作,即不失本宗。若有人问汝义,问有将无对,问无将有对,问凡以圣对,问圣以凡对。二道相因,生中道义。如一问一对,余问一依此作,即不失理也。设有人问:‘何名为暗?’答云:‘明是因,暗是缘,明没即暗,以明显暗,以暗显明,来去相因,成中道义’。余问悉皆如此。汝等于后传法,依此转相教授,勿失宗旨。”

师于太极元年壬子延和七月(是年五月改延和。八月玄宗即位,方改元先天。次年遂改开元。他本作先天者非),命门人往新州国恩寺建塔,仍令促工,次年夏末落成。七月一日,集徒众曰:“吾至八月,欲离世间。汝等有疑,早须相问,为汝破疑,令汝迷尽。吾若去后,无人教汝。”法海等闻,悉皆涕泣。惟有神会,神情不动,亦无涕泣。师云:“神会小师,却得善不善等,毁誉不动,哀乐不生;余者不得。数年山中,竟修何道?汝今悲泣,为忧阿谁?若忧吾不知去处,吾自知去处;吾若不知去处,终不预报于汝。汝等悲泣,盖为不知吾去处;若知吾去处,即不合悲泣。法性本无生灭去来,汝等尽坐,吾与汝说一偈,名曰‘真假动静偈’。汝等诵取此偈,与吾意同,依此修行,不失宗旨。”众僧作礼,请师说偈,偈曰:

\begin{jizi}{“}
    一切无有真, 不以见于真, \\
    若见于真者, 是见尽非真。 \\
    若能自有真, 离假即心真, \\
    自心不离假, 无真何处真? %\\
\end{jizi}
\begin{jizi}{}
    有情即解动, 无情即不动, \\
    若修不动行, 同无情不动。 \\
    若觅真不动, 动上有不动, \\
    不动是不动, 无情无佛种。 \\
    能善分别相, 第一义不动, \\
    但作如此见, 即是真如用。 \\
    报诸学道人, 努力须用意, \\
    莫于大乘门, 却执生死智。 \\
    若言下相应, 即共论佛义; \\
    若实不相应, 合掌令欢喜。 \\
    此宗本无诤, 诤即失道意, \\
    执逆诤法门, 自性入生死。”
\end{jizi}

时,徒众闻说偈已,普皆作礼,并体师意,各各摄心,依法修行,更不敢诤,乃知大师不久住世。法海上座再拜问曰:“和尚入灭之后,衣法当付何人?”师曰:“吾于大梵寺说法,以至于今,钞录流行,目曰《法宝坛经》,汝等守护,递相传授,度诸群生。但依此说,是名正法。今为汝等说法,不付其衣。盖为汝等信根淳熟,决定无疑,堪任大事。然据先祖达摩大师付授偈意,衣不合传。偈曰:

\begin{jizi}{“‘}
    吾本来兹土, 传法救迷情, \\
    一花开五叶, 结果自然成。’”
\end{jizi}

师复曰:“诸善知识!汝等各各净心,听吾说法。若欲成就种智,须达一相三昧、一行三昧。若于一切处而不住相,于彼相中不生憎爱,亦无取舍,不念利益成坏等事,安闲恬静,虚融澹泊,此名一相三昧。若于一切处,行住坐卧,纯一直心,不动道场,真成净土,此名一行三昧。若人具二三昧,如地有种,含藏长养,成熟其实。一相一行,亦复如是。我今说法,犹如时雨,普润大地。汝等佛性,譬诸种子,遇兹沾洽,悉得发生。承吾旨者,决获菩提。依吾行者,定证妙果。听吾偈曰:

\begin{jizi}{“}
    心地含诸种, 普雨悉皆萌, \\
    顿悟花情已, 菩提果自成。”
\end{jizi}

师说偈已,曰:“其法无二,其心亦然,其道清净,亦无诸相。汝等慎勿观静,及空其心。此心本净,无可取舍,各自努力,随缘好去。”尔时,徒众作礼而退。

大师,七月八日,忽谓门人曰:“吾欲归新州,汝等速理舟楫。”大众哀留甚坚。师曰:“诸佛出现,犹示涅槃。有来必去,理亦常然。吾此形骸,归必有所。”众曰:“师从此去,早晚可回?”师曰:“叶落归根,来时无口。”又问曰:“正法眼藏,传付何人?”师曰:“有道者得,无心者通。”又问:“后莫有难否?”师曰:“吾灭后五六年,当有一人来取吾首。听吾记曰:‘头上养亲,口里须餐,遇满之难,杨柳为官。’”又云:“吾去七十年,有二菩萨从东方来,一出家、一在家,同时兴化,建立吾宗,缔缉伽蓝,昌隆法嗣。”问曰:“未知从上佛祖,应现已来,传授几代?愿垂开示。”师云:“古佛应世,已无数量,不可计也。今以七佛为始,过去庄严劫,毗婆尸佛、尸弃佛、毗舍浮佛;今贤劫,拘留孙佛、拘那含牟尼佛、迦叶佛、释迦文佛。是为七佛。

“释迦文佛首传第一摩诃迦叶尊者、第二阿难尊者、第三商那和修尊者、第四优波毱多尊者、第五提多迦尊者、第六弥遮迦尊者、第七婆须蜜多尊者、第八佛驮难提尊者、第九伏驮蜜多尊者、第十胁尊者、十一富那夜奢尊者、十二马鸣大士、十三迦毗摩罗尊者、十四龙树大士、十五迦那提婆尊者、十六罗睺罗多尊者、十七僧伽难提尊者、十八伽耶舍多尊者、十九鸠摩罗多尊者、二十阇耶多尊者、二十一婆修盘头尊者、二十二摩拏罗尊者、二十三鹤勒那尊者、二十四师子尊者、二十五婆舍斯多尊者、二十六不如蜜多尊者、二十七般若多罗尊者、二十八菩提达摩尊者(此土是为初祖)、二十九慧可大师、三十僧璨大师、三十一道信大师、三十二弘忍大师。惠能是为三十三祖。从上诸祖,各有禀承。汝等向后,递代流传,毋令乖误。”

大师,先天二年癸丑岁,八月初三日(是年十二月改元开元),于国恩寺斋罢,谓诸徒众曰:“汝等各依位坐,吾与汝别。”法海白言:“和尚留何教法,令后代迷人得见佛性?”师言:“汝等谛听!后代迷人,若识众生,即是佛性;若不识众生,万劫觅佛难逢。吾今教汝识自心众生,见自心佛性。欲求见佛,但识众生,只为众生迷佛,非是佛迷众生。自性若悟,众生是佛;自性若迷,佛是众生。自性平等,众生是佛;自性邪险,佛是众生。汝等心若险曲,即佛在众生中;一念平直,即是众生成佛。我心自有佛,自佛是真佛。自若无佛心,何处求真佛?汝等自心是佛,更莫狐疑。外无一物而能建立,皆是本心生万种法。故经云:‘心生种种法生,心灭种种法灭。’吾今留一偈,与汝等别,名‘自性真佛偈’。后代之人,识此偈意,自见本心,自成佛道。偈曰:

\begin{jizi}{“}
    真如自性是真佛, 邪见三毒是魔王, \\
    邪迷之时魔在舍, 正见之时佛在堂。 \\
    性中邪见三毒生, 即是魔王来住舍, \\
    正见自除三毒心, 魔变成佛真无假。 \\
    法身报身及化身, 三身本来是一身, \\
    若向性中能自见, 即是成佛菩提因。 \\
    本从化身生净性, 净性常在化身中, \\
    性使化身行正道, 当来圆满真无穷。 \\
    淫性本是净性因, 除淫即是净性身, \\
    性中各自离五欲, 见性刹那即是真。 \\
    今生若遇顿教门, 忽悟自性见世尊, \\
    若欲修行觅作佛, 不知何处拟求真? \\
    若能心中自见真, 有真即是成佛因, \\
    不见自性外觅佛, 起心总是大痴人。 \\
    顿教法门今已留, 救度世人须自修, \\
    报汝当来学道者, 不作此见大悠悠。”
\end{jizi}

师说偈已,告曰:“汝等好住,吾灭度后,莫作世情悲泣雨泪,受人吊问,身著孝服,非吾弟子,亦非正法。但识自本心,见自本性,无动无静,无生无灭,无去无来,无是无非,无住无往。恐汝等心迷,不会吾意,今再嘱汝,令汝见性。吾灭度后,依此修行,如吾在日;若违吾教,纵吾在世,亦无有益。”复说偈曰:

\begin{jizi}{“}
    兀兀不修善, 腾腾不造恶, \\
    寂寂断见闻, 荡荡心无著。”
\end{jizi}

师说偈已,端坐至三更,忽谓门人曰:“吾行矣!”奄然迁化。于时,异香满室,白虹属地,林木变白,禽兽哀鸣。

十一月,广韶新三郡官僚,洎门人僧俗,争迎真身,莫决所之。乃焚香祷曰:“香烟指处,师所归焉。”时香烟直贯曹溪。十一月十三日,迁神龛并所传衣钵而回。次年七月出龛,弟子方辩以香泥上之。门人忆念取首之记,仍以铁叶漆布,固护师颈入塔。忽于塔内白光出现,直上冲天,三日始散。

韶州奏闻,奉敕立碑,纪师道行。师春秋七十有六,年二十四传衣,三十九祝发,说法利生三十七载,得旨嗣法者四十三人,悟道超凡者莫知其数。达摩所传信衣(西域屈眴布也),中宗赐磨衲宝钵,及方辩塑师真相,并道具等,主塔侍者尸之,永镇宝林道场。留传《坛经》,以显宗旨,兴隆三宝,普利群生者。

\textbf{六祖大师法宝坛经(终)}

\end{document}

