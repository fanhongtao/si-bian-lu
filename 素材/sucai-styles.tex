\usepackage{CJKfntef}
\usepackage{float}

\usepackage{geometry}
\geometry{a4paper,left=2cm,right=2cm,top=2cm,bottom=1cm}

\usepackage{graphicx}

\usepackage{hyperref}
\hypersetup{colorlinks=true, linkcolor=red}

\linespread{1.6}

\usepackage{fancyhdr}
\usepackage{ifthen}
\pagestyle{fancy}
\fancyhf{}
\setlength{\headheight}{14pt}
\fancyhead[R]{\ifthenelse{\value{page}>1}{\thepage}{}}
\fancyhead[C]{\ifthenelse{\value{page}>1}{\articletitle}{}}
\renewcommand\headrulewidth{0pt}

\usepackage{tcolorbox}
\tcbuselibrary{skins}

\newcommand{\zd}[1]{\textbf{\textcolor[RGB]{123,12,0}{#1}}} % 重点

\newcommand{\yinyong}[1]{% 引用
    \begin{tcolorbox}[enhanced,
        frame hidden, interior hidden,
        before skip = 5mm, left skip=10mm,
        borderline west={5pt}{0pt}{gray!50}]
        #1
    \end{tcolorbox}
}

\newcommand{\dianping}[1]{% 点评
    \begin{tcolorbox}[enhanced,
        colframe=red!50!black,
        title=点评]
        #1
    \end{tcolorbox}
}

\newcommand{\xhx}[1]{%下划线(模拟微信中的划线功能,用于标注我个人认为的文章中精彩的地方)
    \CJKunderline*[thickness=1.5pt, format=\color[RGB]{84,216,140}]{#1}
}

\newcommand{\biaoti}[1]{% 标题
    \section*{#1}
}

\newcommand{\SetSectionType} {
    \ctexset{
        section={
            number = \chinese{section},
            aftername={、},
            format=\Large\bfseries,
        }
    }
}

