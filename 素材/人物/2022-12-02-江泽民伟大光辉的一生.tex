\documentclass[UTF8, 11pt, oneside]{ctexart}
% \pagestyle{headings}

\usepackage{geometry}
\geometry{a4paper,left=2cm,right=2cm,top=2cm,bottom=1cm}

\usepackage{hyperref}
\hypersetup{colorlinks=true, linkcolor=red}

\usepackage[shortlabels]{enumitem}

\linespread{1.6}

\setlist[itemize]{nosep, left=\parindent}

\usepackage{fancyhdr}
\usepackage{ifthen}
\pagestyle{fancy}
\fancyhf{}
\setlength{\headheight}{14pt}
\fancyhead[R]{\ifthenelse{\value{page}>1}{\thepage}{}}
\fancyhead[C]{\ifthenelse{\value{page}>1}{江泽民伟大光辉的一生}{}}
\renewcommand\headrulewidth{0pt}


\begin{document}

\begin{center}
    \LARGE{江泽民伟大光辉的一生\footnotemark}
\end{center}
\footnotetext{
    \href{https://www.ccps.gov.cn/}{中共中央党校} 2022 年 12 月 2 号的文章 《\href{https://www.ccps.gov.cn/zytt/tstt/202212/t20221202_155924.shtml}{江泽民伟大光辉的一生}》
}

江泽民是全党全军全国各族人民公认的享有崇高威望的卓越领导人,伟大的马克思主义者,伟大的无产阶级革命家、政治家、军事家、外交家,久经考验的共产主义战士,中国特色社会主义伟大事业的杰出领导者,党的第三代中央领导集体的核心,“三个代表”重要思想的主要创立者。

江泽民的一生,是光辉的战斗的一生。在70多年的革命生涯中,他对共产主义理想坚贞不渝,对党和人民无限忠诚,矢志不移为党和人民事业而奋斗。

党的十三届四中全会以后,在国内外形势十分复杂、世界社会主义出现严重曲折的严峻考验面前,江泽民带领党的中央领导集体,紧紧依靠全党全军全国各族人民,捍卫了中国特色社会主义伟大事业,成功把中国特色社会主义推向二十一世纪,建立了永不磨灭的功勋,赢得了全党全军全国各族人民衷心爱戴和国际社会广泛赞誉。

1926年8月17日,江泽民出生于江苏省扬州市一个爱国知识分子家庭。江泽民从小受到爱国主义思想和民主革命思想的启蒙,同时在诗书世家的氛围中深受中华优秀传统文化的熏陶。江泽民早年就读于扬州东关小学和扬州中学,在扬州中学求学期间家乡被日本侵略军占领,他常去梅花岭明代爱国名将史可法墓凭吊,吟诵史公祠的楹联“数点梅花亡国泪,二分明月故臣心”,抒发悲愤心情。

1943年,江泽民考入南京中央大学电机系,积极参加进步学生抗日爱国活动。抗日战争胜利后,江泽民转到上海交通大学电机系学习。1946年4月,江泽民加入中国共产党,成为一名共产主义战士。江泽民积极从事党的地下工作,参加矛头直指国民党反动统治的反饥饿、反内战、反迫害的爱国学生运动,掩护革命同志。

1947年,江泽民从上海交通大学电机系毕业后,到上海粮服实验工厂工作,历任工程师、工务科长、电务工场主任、动力车间主任等职。在此期间,他在工人群众中并在青年会夜校职业青年中从事革命宣传工作,1949年组织工人群众开展护厂活动,迎接上海解放。

中华人民共和国成立初期,江泽民先后担任上海益民食品一厂第一副厂长、上海制皂厂第一副厂长、第一机械工业部上海第二设计分局电器专业科科长等职。在益民食品厂期间,他负责研制和创立了“光明牌”食品品牌;为了支援抗美援朝,他组织生产了专门供应中国人民志愿军的罐头食品。在第二设计分局期间,他主持了新中国第一台国产汽轮发电机设计工作。

1954年9月,江泽民奉调参加兴建长春第一汽车制造厂,1955年4月到莫斯科斯大林汽车制造厂实习,1956年5月回国后在长春第一汽车制造厂任动力处副处长、副总动力师和动力分厂厂长等职。1961年,煤炭供应紧张,江泽民任动力锅炉改烧原油的大型工程总指挥,工程获得成功。

1962年,江泽民调任第一机械工业部上海电器科学研究所副所长,负责该所科研领导工作,主持完成了当时国家急需的JO2小型异步电机系列的设计任务。1964年和1965年,江泽民先后作为中国代表团副团长参加在日本、法国举行的国际电工委员会年会,并考察国外电气科学技术发展情况。

1966年5月,江泽民调任武汉热工机械研究所所长、代理党委书记,9月任党委书记,组织原子能发电设备的设计工作。“文化大革命”开始后,江泽民受到冲击。1970年底,江泽民调到第一机械工业部工作,1971年任中国第一机械工业部派驻罗马尼亚专家组总组长,负责领导中国援助的11个工厂建设工作。1973年回国,江泽民先后任第一机械工业部外事局副局长、局长。1973年、1974年,江泽民先后任中国代表团副团长、团长,参加在联邦德国和罗马尼亚举行的国际电工委员会年会。1976年,江泽民率团前往巴基斯坦,考察中国援建工厂建设情况。1978年,江泽民任中国机械工业代表团秘书长,访问欧洲六国,对加速提升我国机械工业技术水平、加快产品更新换代、提高提供成套技术装备能力、扩大机械产品出口等问题进行了深入思考,提出了对策和建议。

1978年12月召开的中共十一届三中全会,果断结束“以阶级斗争为纲”,实现党和国家工作中心战略转移,开启了改革开放和社会主义现代化建设新时期,实现了新中国成立以来党的历史上具有深远意义的伟大转折。1980年,江泽民担任国家进出口管理委员会、国家外国投资管理委员会副主任兼秘书长、党组成员,参与制定扩大对外贸易、引进国外先进技术和设备、吸收利用外资等方面的政策,同时分管国家对广东、福建两省采取特殊政策和灵活措施的具体贯彻工作,并参加筹建经济特区。兴办经济特区是党和国家为推进改革开放和社会主义现代化建设进行的伟大创举,也是一个以往缺乏经验、需要从头摸索的新事物。1980年,江泽民先后率团到泰国、斯里兰卡、马来西亚、新加坡、菲律宾、香港、墨西哥、爱尔兰等国家和地区考察,了解出口加工区、自由贸易区、边境经济区的情况,从中研究吸取可供中国举办经济特区借鉴的经验。同年8月,江泽民在五届全国人大常委会第十五次会议上作关于在广东、福建两省设置经济特区和《广东省经济特区条例》的说明,为会议审议通过相关议案提供了重要依据。

1982年5月,江泽民任电子工业部第一副部长、党组副书记,1983年任部长、党组书记。他深入生产第一线调查研究,提出电子工业“打基础、上水平、抓质量、求效益、翻三番、超十年”的发展方针,组织领导电子工业结构调整和技术改造,加强集成电路、计算机、通信以及系统工程等重点项目的科研开发和生产工作,使电子工业更好为国家经济建设和国防建设服务。1982年9月,江泽民在中共十二大上当选为中央委员。

1985年,江泽民出任上海市市长、中共上海市委副书记,1987年任中共上海市委书记。他紧紧抓住经济建设这个中心,坚持四项基本原则,坚持改革开放,在加强物质文明建设的同时十分重视社会主义精神文明建设,全力促进上海的改革发展稳定。他领导制订上海经济发展规划和城市建设规划,提出在20世纪末把上海建设成为开放型、多功能、产业结构合理、科学技术先进,具有高度文明的社会主义现代化城市的发展方针。他积极支持开发开放浦东,强调要把浦东建设成为国际化、枢纽化、现代化的世界一流的新市区。为广泛听取各方面意见,充分调动各方面积极性,实现上海经济社会发展战略目标,他倡导建立了民主党派双月座谈会、新闻界理论界双月座谈会等制度。他注重以科技创新推动经济发展,建立了重大工程决策专家论证制度。为了扩大上海对外开放,他提出相互了解、相互信任、互惠互利、长远考虑的发展对外经济交流的四项原则。在他倡导下,上海市政府采取每年必须为人民办几件实事的做法,努力解决广大群众衣食住行方面的实际问题,改善群众生活。1987年11月,在中共十三届一中全会上,江泽民当选为中共中央政治局委员。1989年春夏之交我国发生严重政治风波,他拥护和执行党中央关于旗帜鲜明反对动乱、捍卫社会主义国家政权、维护人民根本利益的正确决策,紧紧依靠广大党员、干部、群众,有力维护上海稳定。

1989年6月,在中共十三届四中全会上,江泽民当选为中共中央政治局常委、中央委员会总书记。他在会上坚定指出:“党的十一届三中全会以来的路线和基本政策没有变,必须继续贯彻执行。在这个最基本的问题上,我要十分明确地讲两句话:一句是坚定不移,毫不动摇;一句是全面执行,一以贯之。”同年11月,中共十三届五中全会决定江泽民为中共中央军事委员会主席。1990年3月,在七届全国人大三次会议上,江泽民当选为中华人民共和国中央军事委员会主席。

1992年10月,在中国共产党第十四次全国代表大会上,江泽民作题为《加快改革开放和现代化建设步伐,夺取有中国特色社会主义事业的更大胜利》的报告,提出确立邓小平建设有中国特色社会主义理论在全党的指导地位,确定中国经济体制改革的目标是建立社会主义市场经济体制,强调抓住机遇,加快发展,集中精力把经济建设搞上去。在中共十四届一中全会上,江泽民当选为中共中央政治局常委、中央委员会总书记,全会决定江泽民为中共中央军事委员会主席。1993年3月,在八届全国人大一次会议上,江泽民当选为中华人民共和国主席和中华人民共和国中央军事委员会主席。

1997年9月,在中国共产党第十五次全国代表大会上,江泽民作题为《高举邓小平理论伟大旗帜,把建设有中国特色社会主义事业全面推向二十一世纪》的报告,着重阐述邓小平理论的历史地位和指导意义,提出党在社会主义初级阶段的基本纲领,明确公有制为主体、多种所有制经济共同发展是我国社会主义初级阶段的一项基本经济制度,强调依法治国、建设社会主义法治国家,明确我国改革开放和现代化建设跨世纪发展的宏伟目标。在中共十五届一中全会上,江泽民当选为中共中央政治局常委、中央委员会总书记,全会决定江泽民为中共中央军事委员会主席。1998年3月,在九届全国人大一次会议上,江泽民当选为中华人民共和国主席和中华人民共和国中央军事委员会主席。

2002年11月,在中国共产党第十六次全国代表大会上,江泽民作题为《全面建设小康社会,开创中国特色社会主义事业新局面》的报告,提出全面建设小康社会的奋斗目标,阐述全面贯彻“三个代表”重要思想的根本要求。中共十六届一中全会决定江泽民为中共中央军事委员会主席。2003年3月,在十届全国人大一次会议上,江泽民当选为中华人民共和国中央军事委员会主席。

中共十三届四中全会以后,江泽民在领导建设中国特色社会主义的实践中,团结带领全党全国各族人民,坚持党的基本理论、基本路线、基本纲领,加深了对什么是社会主义、怎样建设社会主义和建设什么样的党、怎样建设党的认识,形成了“三个代表”重要思想,丰富和发展了中国特色社会主义理论。“三个代表”重要思想,既坚持了马克思主义基本原理,又反映了世界和中国发展变化对党和国家工作的新要求,以新的思想、观点、论断,继承、丰富、发展了马克思列宁主义、毛泽东思想、邓小平理论。

江泽民对中国特色社会主义发展道路问题作出深刻思考,提出必须把发展作为党执政兴国的第一要务,不断开创现代化建设新局面。他提出,全党工作的大局是抓住机遇、深化改革、扩大开放、促进发展、保持稳定,这是必须长期坚持的方针。要全面把握党的基本路线的全部内容,把经济建设这个中心同四项基本原则、改革开放这两个基本点统一于建设中国特色社会主义的伟大实践,贯穿于现代化建设整个过程。他指出,发展是硬道理,这是我们必须始终坚持的一个战略思想,财大才能气粗,落后就要挨打。只要我们的经济实力、国防实力、民族凝聚力不断增强,就可以“任凭风浪起,稳坐钓鱼船”。他深刻分析了党和国家面临的新形势,敏锐指出,综观全局,21世纪头20年,对我国来说,是一个必须紧紧抓住并且可以大有作为的重要战略机遇期,一定要有主动精神和忧患意识,抓住机遇而不可丧失机遇,开拓进取而不可因循守旧,集中全国人民智慧和力量,聚精会神搞建设,一心一意谋发展,不断促进社会主义物质文明、政治文明、精神文明协调发展。建设中国特色社会主义应该是我国经济、政治、文化全面发展的进程,是我国社会主义物质文明、政治文明、精神文明全面建设的进程。他提出,我们建设中国特色社会主义各项事业,我们进行的一切工作,既要着眼于人民现实的物质文化生活需要,同时又要着眼于促进人民素质的提高,也就是要努力促进人的全面发展,推进人的全面发展同推进经济、政治、文化发展和改善人民物质文化生活是互为前提和基础的。他强调,要正确认识和处理改革发展稳定的关系,改革是动力,发展是目标,稳定是前提,要把改革的力度、发展的速度和社会可承受的程度统一起来,把不断改善人民生活作为处理改革发展稳定关系的重要结合点,在社会稳定中推进改革发展,通过改革发展促进社会稳定。

江泽民提出21世纪头20年是全面建设小康社会的阶段,是对中国特色社会主义发展阶段和发展战略的丰富和发展,符合我国国情,符合人民愿望,有利于最广泛最充分调动一切积极因素为实现中华民族的伟大复兴而奋斗。在20世纪90年代,江泽民就对全面建设小康社会、实现第三步战略目标进行了前瞻性的战略思考。他在中共十五大报告中初步勾画了实现第三步战略目标的蓝图,指出展望下世纪,我们的目标是,第一个十年实现国民生产总值比2000年翻一番,使人民的小康生活更加宽裕,形成比较完善的社会主义市场经济体制;再经过十年的努力,到建党一百年时,使国民经济更加发展,各项制度更加完善;到世纪中叶建国一百年时,基本实现现代化,建成富强民主文明的社会主义国家。中共十五届五中全会进一步提出,从新世纪开始,我国将进入全面建设小康社会、加快推进社会主义现代化的新的发展阶段。在中共十六大上,他深刻阐述了全面建设小康社会的奋斗目标,强调我们要在本世纪头20年,集中力量,全面建设惠及十几亿人口的更高水平的小康社会,使经济更加发展、民主更加健全、科教更加进步、文化更加繁荣、社会更加和谐、人民生活更加殷实。这是实现现代化建设第三步战略目标必经的承上启下的发展阶段,也是完善社会主义市场经济体制和扩大对外开放的关键阶段。

关于中国特色社会主义的根本任务,江泽民指出,必须把集中力量发展社会生产力摆在首要地位,不断促进先进生产力发展,这是党始终站在时代前列、保持先进性的根本体现和根本要求。他强调,人是生产力中最具有决定性的力量,包括知识分子在内的工人阶级,广大农民,始终是推动我国先进生产力发展和社会全面进步的根本力量;在社会变革中出现的民营科技企业的创业人员和技术人员、受聘于外资企业的管理技术人员、个体户、私营企业主、中介组织的从业人员、自由职业人员等社会阶层,都是中国特色社会主义事业建设者。他提出,人才资源是第一资源,各级党委和政府都要着眼于党和国家事业长远发展和人才总体需要,紧紧抓住培养人才、吸引人才、用好人才三个环节,大力实施人才战略。他高度重视科学技术在推动社会生产力发展中的重要作用,强调我们要牢记一条道理,这就是没有强大的科技实力,就没有社会主义现代化。他提出大力推进知识创新和科技创新,建设国家知识创新体系,增强自主创新能力,实现技术发展跨越。他强调,要不断完善社会主义的生产关系和上层建筑,不断为生产力的解放和发展打开更广阔的通途。

江泽民结合新的实践,进一步深化了对我国社会主义改革的理论思考。他指出,认真总结苏联解体、东欧剧变的教训,以及我们发生“文化大革命”这样严重曲折的教训,深刻分析它们的原因,可以得出两条结论:一是必须坚持社会主义;二是必须进行社会主义改革,探索符合本国实际的社会主义发展道路。他强调,创新是一个民族进步的灵魂,是一个国家兴旺发达的不竭动力,必须坚定不移推进各方面改革,改革要从实际出发,整体推进,重点突破,循序渐进,注重制度建设和创新。20世纪90年代,他以中国共产党人坚持理论创新、与时俱进的巨大勇气,确立了社会主义市场经济体制的改革目标和基本框架,确立了社会主义初级阶段公有制为主体、多种所有制经济共同发展的基本经济制度和按劳分配为主体、多种分配方式并存的分配制度,开创全面改革开放新局面。1992年6月,他提出对高度集中的计划经济体制进行根本性的改革势在必行,不然就不可能实现我国的现代化,根据邓小平南方谈话精神,他明确提出使用“社会主义市场经济体制”这个提法,为中共十四大召开做了重要的思想理论准备,中共十四大正式把建立社会主义市场经济体制确立为中国经济体制改革的目标。中共十四届三中全会通过的《关于建立社会主义市场经济体制若干问题的决定》,制定了建立社会主义市场经济体制的总体规划。到20世纪末,我国已初步建立了社会主义市场经济体制的基本框架。江泽民指出,公有制为主体、多种所有制经济共同发展,是我国社会主义初级阶段的一项基本经济制度;公有制实现形式可以而且应当多样化;股份制是现代企业的一种资本组织形式,资本主义可以用,社会主义也可以用。必须毫不动摇巩固和发展公有制经济,必须毫不动摇鼓励、支持和引导非公有制经济发展。他强调,改革国有资产管理体制是深化经济体制改革的重大任务,要在坚持国家所有的前提下充分发挥中央和地方两个积极性。他指出,国有企业是我国国民经济的支柱,国有企业改革是我国经济体制改革的中心环节,要着眼于搞好整个国有经济,抓好大的,放活小的,对国有企业实施战略性改组。他指出,建立现代企业制度是国有企业改革的方向,要按照“产权清晰、权责明确、政企分开、管理科学”的要求,对国有大中型企业实行规范的公司制改革,使企业成为适应市场的法人实体和竞争主体。他提出,充分发挥市场机制的作用和加强宏观调控都是建立社会主义市场经济体制的基本要求,要加快健全和完善宏观调控体系,深化金融、财政、计划体制改革,完善宏观调控手段和协调机制,实施适度从紧的财政政策和货币政策,注意掌握调控力度。他强调,理顺分配关系事关广大群众的切身利益和积极性的发挥,要完善按劳分配为主体、多种分配方式并存的分配制度,坚持效率优先、兼顾公平,建立健全同经济发展水平相适应的社会保障体系,完善城镇职工基本养老保险制度和基本医疗保险制度,健全失业保险制度和城市居民最低生活保障制度,探索建立农村养老、医疗保险和最低生活保障制度。

进入20世纪90年代,江泽民紧紧把握经济全球化不断加快的趋势,强调中国要发展、要进步、要富强,就必须对外开放,加强同世界各国的经济、科技、文化的交流合作,吸收和借鉴一切先进的东西。他提出,我们要进一步完善有关政策,继续坚定不移扩大对外开放,不断丰富对外开放的形式和内容,不断提高对外开放的质量和水平,完善全方位、多层次、宽领域的对外开放格局。加入世界贸易组织是党从我国经济发展和改革开放的需要出发作出的重大战略决策,标志着我国对外开放进入了一个新的阶段。江泽民阐明了中国加入世界贸易组织的原则:第一,中国加入世界贸易组织是中国经济发展和改革开放的需要,同样世界贸易组织也需要中国,没有12亿多人口的中国参加,世界贸易组织是不完整的,也不利于世界经济的发展;第二,中国是一个发展中国家,社会生产力还不发达,只能以发展中国家的条件加入世界贸易组织;第三,中国加入世界贸易组织,其权利和义务一定要平衡,中国不会接受过高的、超出中国承受能力的要价。遵照这些指导原则,中国在加入世界贸易组织谈判过程中始终掌握主动权,于2001年12月正式成为世界贸易组织成员。江泽民强调,进一步吸引外商直接投资,提高利用外资质量和水平,进一步扩大商品和服务贸易,实施市场多元化战略,坚持以质取胜,经济特区要增创新优势、更上一层楼。他提出,实施“走出去”战略是对外开放新阶段的重大举措,“引进来”和“走出去”是对外开放的两个轮子,必须同时转动起来。他强调,中国发展和进步离不开世界各国文明成果,在对外开放的过程中,必须始终注意维护国家主权和经济社会安全,注意防范和化解国际风险冲击,处理好扩大对外开放和坚持自力更生的关系,把立足点放在依靠自己力量的基础上。

关于中国特色社会主义经济建设,江泽民提出,我们是发展中国家,要实现现代化,缩小同发达国家的差距,关键在于要走出一条既有较高速度又有较好效益的国民经济发展路子,实现经济增长方式从粗放型向集约型转变,保持国民经济持续快速健康发展。他强调,正确处理速度和效益的关系必须更新发展思路,实现经济增长方式从粗放型向集约型转变;提高经济运行质量和效益,关键是解决结构不合理问题,必须对经济结构进行战略性调整,扩大内需是我国经济发展长期的、基本的立足点。他提出,要走新型工业化道路,坚持以信息化带动工业化,以工业化促进信息化,走出一条科技含量高、经济效益好、资源消耗低、环境污染少、人力资源优势得到充分发挥的新型工业化路子,大力实施科教兴国战略和可持续发展战略。他强调,农业、农村、农民问题始终是一个关系我们党和国家全局的根本性问题,统筹城乡经济社会发展、建设现代农业、发展农村经济、增加农民收入是全面建设小康社会的重大任务。世纪之交,他向全党全国人民明确提出,要不失时机实施西部大开发战略。他指出,发展经济的根本目的是提高全国人民生活水平和质量。就业是民生之本,扩大就业是我国当前和今后长时期重大而艰巨的任务。

关于中国特色社会主义政治建设,江泽民提出,发展社会主义民主政治,建设社会主义政治文明,是全面建设小康社会的重要目标,建设社会主义政治文明是我国改革开放和社会主义现代化建设发展的必然要求,是中国共产党领导人民坚持和发展人民民主长期实践的必然结论,进一步深化了中国共产党对中国特色社会主义事业的规律性认识。他强调,发展社会主义民主政治,最根本的是要把坚持党的领导、人民当家作主、依法治国有机统一起来,坚持和完善社会主义民主制度,健全民主制度,丰富民主形式,扩大公民有序的政治参与,保证人民依法实行民主选举、民主决策、民主管理、民主监督,享有广泛的权利和自由,尊重和保障人权。要坚持和完善人民代表大会制度。坚持和完善共产党领导的多党合作和政治协商制度,巩固和发展最广泛的爱国统一战线,全面贯彻党的民族政策,坚持和完善民族区域自治制度,巩固和发展平等团结互助的社会主义民族关系,促进各民族共同繁荣进步。要全面贯彻党的宗教信仰自由政策,依法管理宗教事务,积极引导宗教与社会主义社会相适应,坚持独立自主自办的原则,坚决抵御境外利用宗教进行渗透。要认真贯彻党的侨务政策。他强调,扩大基层民主是发展社会主义民主的基础性工作。他提出,要实行依法治国,建设社会主义法治国家,加强社会主义法制建设,坚持有法可依、有法必依、执法必严、违法必究,使国家各项工作逐步走上法制化轨道。他提出,政治体制改革是社会主义政治制度的自我完善和发展,要坚持从我国国情出发,总结自己的实践经验,同时借鉴人类政治文明有益成果,绝不照搬西方政治制度的模式。要着重加强制度建设,实现社会主义民主政治制度化、规范化、程序化。他提出,要高度重视和认真做好维护社会稳定工作,努力维护安定团结的社会政治局面。

关于中国特色社会主义文化建设,江泽民提出,有中国特色社会主义的文化是凝聚和激励全国各族人民的重要力量,是综合国力的重要标志,在当代中国,发展先进文化就是发展有中国特色社会主义的文化,就是建设社会主义精神文明。必须坚持马克思列宁主义、毛泽东思想、邓小平理论在意识形态领域的指导地位,用“三个代表”重要思想统领社会主义文化建设,坚持为人民服务、为社会主义服务的方向和百花齐放、百家争鸣的方针,弘扬主旋律,提倡多样化。要坚持以科学的理论武装人、以正确的舆论引导人、以高尚的精神塑造人、以优秀的作品鼓舞人,大力发展先进文化,支持健康有益文化,努力改造落后文化,坚决抵制腐朽文化。他强调,民族精神是一个民族赖以生存和发展的精神支撑,面对世界范围各种思想文化的相互激荡,必须把弘扬和培育民族精神作为文化建设极为重要的任务,纳入国民教育全过程,纳入精神文明建设全过程,使全体人民始终保持昂扬向上的精神状态。要坚持不懈加强社会主义法制建设、依法治国,同时要坚持不懈加强社会主义道德建设、以德治国,把法制建设与道德建设紧密结合起来,把依法治国与以德治国紧密结合起来。

江泽民担任中央军委主席期间,对国防和人民军队建设提出许多重要思想,创立了江泽民国防和军队建设思想。他提出,建立巩固的国防是我国现代化建设的战略任务,是维护国家安全统一和全面建设小康社会的重要保障。要坚持以毛泽东军事思想、邓小平新时期军队建设思想为指导,按照政治合格、军事过硬、作风优良、纪律严明、保障有力的总要求,紧紧围绕打得赢、不变质两个历史性课题,坚定不移走中国特色的精兵之路,加强人民军队革命化现代化正规化建设。要始终把思想政治建设摆在我军各项建设的首位,永葆人民军队性质、本色、作风。他强调,党对人民军队的绝对领导是我军永远不变的军魂,要毫不动摇地坚持党领导人民军队的根本原则和制度。他提出要迎接世界新军事变革挑战,积极推进中国特色军事变革。20世纪90年代,中共中央和中央军委确立新时期积极防御的军事战略方针,在战略指导上实行重大调整,把军事斗争准备的基点由应付一般条件下的局部战争转到打赢现代技术特别是高技术条件下的局部战争上来,此后进一步提出把军事斗争准备的基点放到打赢信息化条件下的局部战争上来,明确了新形势下我军军事斗争准备的目标和任务,抓住了我军建设的主要矛盾,正确解决了我军建设和改革的发展方向问题。他提出,军队、武警部队和政法机关一律不再从事经商活动。他强调,要坚持国防建设与经济建设协调发展的方针,在经济发展的基础上推进国防和军队现代化,尽快形成自己的高技术武器装备的“杀手锏”。

江泽民指出,完成祖国统一大业,是中华民族的根本利益所在,是全中国人民包括台湾同胞、港澳同胞、海外侨胞的共同愿望。他领导推进中国政府对香港、澳门恢复行使主权的一系列准备工作。1997年7月1日实现香港回归,1999年12月20日实现澳门回归。他先后赴香港、澳门参加中英政府、中葡政府举行的香港、澳门交接仪式。香港和澳门回归祖国,丰富了“一国两制”的理论和实践。香港、澳门回归祖国后,他强调,在任何情况下都必须全面正确贯彻“一国两制”方针,严格按照香港基本法、澳门基本法办事,全力支持香港、澳门两个特别行政区行政长官和政府的工作,广泛团结港澳各界人士,共同维护和促进香港和澳门繁荣、稳定、发展。在中央政府坚定支持下,两个特别行政区政府沉着应对亚洲金融危机及内外经济环境变化带来的冲击和影响,团结各界人士,妥善处理了一系列社会和经济问题,保持了社会稳定和经济发展。江泽民把握解决台湾问题大局,推动两岸双方达成体现一个中国原则的“九二共识”,推进两岸协商谈判。1995年1月,他发表了《为促进祖国统一大业的完成而继续奋斗》的重要讲话,提出了现阶段发展两岸关系、推进祖国和平统一进程的八项主张,强调坚持一个中国的原则是实现和平统一的基础和前提,我们不承诺放弃使用武力决不是针对台湾同胞,而是针对外国势力干涉中国统一和搞“台湾独立”的图谋的。讲话既体现完成祖国统一大业的坚定决心,又充分考虑到台湾同胞愿望和台湾实际情况,引起海内外高度关注和积极反响。他提出文攻武备总方略,领导开展反分裂、反“台独”重大斗争。他郑重指出,国家要统一,民族要复兴,台湾问题不可能无限期地拖延下去;我们坚信,通过全体中华儿女共同努力,祖国的完全统一就一定能够早日实现。

面对国际局势跌宕起伏,江泽民深刻洞察世界形势发展总趋势,提出了一系列外交和国际战略思想,丰富了中国特色社会主义外交理论和实践。他指出,和平与发展仍是当今时代的主题,世界多极化和经济全球化在曲折中发展给世界的和平与发展带来了机遇和有利条件,新的世界大战在可预见的时期内打不起来,争取较长时期的和平国际环境和良好周边环境是可以实现的,但不公正不合理的国际政治经济旧秩序没有根本改变,影响和平与发展的不确定因素在增加,传统安全威胁和非传统安全威胁的因素相互交织,霸权主义和强权政治有新的表现,南北差距进一步扩大,世界还很不安宁,人类面临着许多严峻挑战。他强调,和平与发展仍然是当今时代的主题,不管国际风云如何变幻,中国始终不渝奉行独立自主的和平外交政策,中国外交政策的宗旨是维护世界和平、促进共同发展。他提出,中国把加强同发展中国家的团结合作作为对外政策的基本立足点,改善和发展同发达国家的关系,继续加强睦邻友好,坚持与邻为善、以邻为伴,加强区域合作,把同周边国家的交流和合作推向新水平,增强同第三世界的团结和合作,加强相互帮助和支持。他提出,要积极推动世界走向多极化,推进国际关系民主化,尊重世界多样性,加强文明交流互鉴。他强调,中国将继续积极参与多边外交活动,在联合国和其他国际及区域性组织中发挥作用,支持发展中国家维护自身正当权益。他提出建立适应时代需要的新安全观,核心是互信、互利、平等、协作。他推动成立的上海合作组织是第一个由中国参与推动建立并以中国城市命名的地区性合作组织,它所倡导的互信、互利、平等、协商、尊重多样文明、谋求共同发展的“上海精神”在当代国际关系中产生了重要影响。

江泽民高度重视党的自身建设,号召全党同志坚持从新的实际出发,以改革的精神研究和解决党的建设面临的重大理论和现实问题,使党始终保持先进性和纯洁性,充满创造力、凝聚力、战斗力。他提出,治国必先治党,治党务必从严,完整提出提高领导水平和执政水平、增强拒腐防变和抵御风险的能力两大历史性课题,要求全党认真研究和解决。他强调,在我们这样一个多民族的发展中大国,要把全体人民意志和力量凝聚起来,全面建设小康社会,加快推进社会主义现代化,必须毫不放松加强和改善党的领导,全面推进党的建设新的伟大工程,保证我们党始终是中国工人阶级的先锋队,同时是中国人民和中华民族的先锋队,始终是中国特色社会主义事业的领导核心,始终代表中国先进生产力的发展要求,代表中国先进文化的前进方向,代表中国最广大人民的根本利益。他强调,必须把党的思想理论建设摆在更加突出的位置。要加强党的执政能力建设,提高党的领导水平和执政水平。要坚持和健全民主集中制,增强党的活力和团结统一。要建设高素质领导干部队伍,形成朝气蓬勃、奋发有为的领导层,领导干部要讲学习、讲政治、讲正气。要做好基层党建工作,增强党的阶级基础和扩大党的群众基础。要加强和改进党的作风建设,深入开展反腐败斗争。他强调,推进党的作风建设,核心是保持党同人民群众的血肉联系,我们党的最大政治优势是密切联系群众,党执政后的最大危险是脱离群众。坚决反对和防止腐败是全党一项重大的政治任务,要坚持标本兼治、综合治理的方针,逐步加大治本力度,加强教育,发展民主,健全法制,强化监督,创新体制,把反腐败寓于各项重要政策措施之中,从源头上预防和解决腐败问题。他强调,只要全党同志始终保持共产党人的蓬勃朝气、昂扬锐气和浩然正气,永远同人民群众心连心,我们党的执政基础就坚如磐石。

江泽民强调,建设有中国特色社会主义全部工作的出发点和落脚点,是为了不断实现好、维护好、发展好最广大人民根本利益。制定和贯彻党的方针政策,基本着眼点是要代表最广大人民根本利益,正确反映和兼顾不同方面群众利益,使全体人民朝着共同富裕的方向稳步前进。他指出,必须始终坚持党的群众路线,一切为了群众,一切依靠群众,从群众中来,到群众中去,尊重人民群众创造,倾听人民群众呼声,反映人民群众意愿,集中人民群众智慧和力量去发展各项事业,在整个改革开放和社会主义现代化建设的过程中都要努力使工人、农民、知识分子和其他群众共同享受到经济社会发展成果。他要求各级领导干部牢记全心全意为人民服务的宗旨,树立强烈的公仆意识,深怀爱民之心,恪守为民之责,善谋富民之策,多办利民之事,时刻都要把人民群众安危冷暖放在心上,关心群众疾苦,努力为群众办实事、办好事。

江泽民担任党和国家主要领导职务之际,我国正面临着国内外形势十分复杂、世界社会主义出现严重曲折的严峻考验。他团结带领党的中央领导集体,紧紧依靠全党全军全国各族人民,从容应对一系列关系我国主权和安全的国际突发事件,战胜在政治、经济领域和自然界出现的困难和风险,特别是成功应对了亚洲金融危机的冲击、夺取了1998年抗洪抢险斗争的全面胜利,保证了改革开放和社会主义现代化建设的航船始终沿着正确方向破浪前进。

2004年9月,中共十六届四中全会决定同意江泽民辞去中共中央军事委员会主席的职务。2005年3月十届全国人大三次会议第二次全体会议,通过了关于接受江泽民辞去中华人民共和国中央军事委员会主席职务的请求的决定。从领导岗位上退下来以后,江泽民坚决拥护和支持党中央工作,关心中国特色社会主义伟大事业,坚定支持党风廉政建设和反腐败斗争。

江泽民亲自主持编辑和逐篇审定《江泽民文选》第一卷、第二卷、第三卷。《江泽民文选》主要收入了江泽民从二十世纪八十年代末至二十一世纪初具有代表性和独创性的重要著作,为我们更深入地学习领会“三个代表”重要思想、继续推进中国特色社会主义伟大事业和党的建设新的伟大工程提供了重要教材。

江泽民创立的“三个代表”重要思想,是对马克思列宁主义、毛泽东思想、邓小平理论的继承和发展,反映了当代世界和中国的发展变化对党和国家工作的新要求,是加强和改进党的建设、推进我国社会主义自我完善和发展的强大理论武器,是中国共产党集体智慧的结晶,是党必须长期坚持的指导思想,是党和人民的宝贵精神财富。

今天,我们比历史上任何时期都更接近、更有信心和能力实现中华民族伟大复兴的目标。新征程上,全党全军全国各族人民要在以习近平同志为核心的党中央坚强领导下,更加紧密地团结起来,高举中国特色社会主义伟大旗帜,全面贯彻习近平新时代中国特色社会主义思想,深刻领悟“两个确立”的决定性意义,增强“四个意识”、坚定“四个自信”、做到“两个维护”,自信自强、守正创新,踔厉奋发、勇毅前行,万众一心为全面建设社会主义现代化国家、全面推进中华民族伟大复兴而不懈奋斗。

\end{document}

